\documentclass{article}

\title{C1 Set Theory}
\author{Robin Adams}

\usepackage{amsmath}
\usepackage{amssymb}
\usepackage{amsthm}
\let\proof\relax
\let\endproof\relax
\let\qed\relax
\usepackage{pf2}
\usepackage[all]{xy}

\newtheorem{axiom}{Axiom}
\newtheorem{axs}[axiom]{Axiom Schema}
\newtheorem{lm}[axiom]{Lemma}
\newtheorem{proposition}[axiom]{Proposition}
\newtheorem{props}[axiom]{Proposition Schema}
\newtheorem{thm}[axiom]{Theorem}
\newtheorem{cor}{Corollary}[axiom]
\theoremstyle{definition}
\newtheorem{definition}[axiom]{Definition}

\begin{document}
    \maketitle

    \section{Primitive Notions}

    Let there be \emph{sets}.

    Let there be a binary relation called \emph{membership}, $\in$. When $x \in y$ holds, we say $x$ is a
    \emph{member} or \emph{element} of $y$. We write $x \notin y$ iff $x$ is not a member of $y$.

    \section{The Axioms}

    \begin{axiom}[Extensionality]
        If two sets have exactly the same members, then they are equal.
    \end{axiom}

    As a consequence of this axiom, we may identify a set $A$ with the class $\{ x : x \in A \}$. The use of
    the symbols $\in$ and $=$ is consistent.

    \begin{definition}
    We say that a class $\mathbf{A}$ \emph{is a set} iff there exists a set $A$ such that $A = \mathbf{A}$.
    That is, the class $\{ x : P(x) \}$ is a set iff
    \[ \exists A. \forall x (x \in A \leftrightarrow P(x)) \enspace . \]
    Otherwise, $\mathbf{A}$ is a \emph{proper class}.
    \end{definition}

    \begin{definition}[Subset]
        If $A$ is a set and $\mathbf{B}$ is a class, we say $A$ is a \emph{subset} of $\mathbf{B}$
        iff $A \subseteq \mathbf{B}$.
    \end{definition}

    \begin{definition}[Power Set]
        For any set $A$, the \emph{power set} of $A$, $\mathcal{P} A$, is the class of all subsets of $A$.
    \end{definition}
\end{document}
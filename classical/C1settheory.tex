\documentclass{article}

\title{C1 Set Theory}
\author{Robin Adams}

\usepackage{amsmath}
\usepackage{amssymb}
\usepackage{amsthm}
\let\proof\relax
\let\endproof\relax
\let\qed\relax
\usepackage{pf2}
\usepackage[all]{xy}

\newtheorem{axiom}{Axiom}
\newtheorem{axs}[axiom]{Axiom Schema}
\newtheorem{lemma}[axiom]{Lemma}
\newtheorem{proposition}[axiom]{Proposition}
\newtheorem{props}[axiom]{Proposition Schema}
\newtheorem{theorem}[axiom]{Theorem}
\newtheorem{corollary}{Corollary}[axiom]
\theoremstyle{definition}
\newtheorem{definition}[axiom]{Definition}

\newcommand{\dom}{\ensuremath{\operatorname{dom}}}
\newcommand{\fld}{\ensuremath{\operatorname{fld}}}
\newcommand{\inv}[1]{\ensuremath{{#1}^{-1}}}
\newcommand{\ran}{\ensuremath{\operatorname{ran}}}

\begin{document}
    \maketitle

    \section{Primitive Notions}

    Let there be \emph{sets}.

    Let there be a binary relation called \emph{membership}, $\in$. When $x \in y$ holds, we say $x$ is a
    \emph{member} or \emph{element} of $y$. We write $x \notin y$ iff $x$ is not a member of $y$.

    \section{The Axioms}

    \begin{axiom}[Extensionality]
        If two sets have exactly the same members, then they are equal.
    \end{axiom}

    As a consequence of this axiom, we may identify a set $A$ with the class $\{ x : x \in A \}$. The use of
    the symbols $\in$ and $=$ is consistent.

    \begin{definition}
    We say that a class $\mathbf{A}$ \emph{is a set} iff there exists a set $A$ such that $A = \mathbf{A}$.
    That is, the class $\{ x : P(x) \}$ is a set iff
    \[ \exists A. \forall x (x \in A \leftrightarrow P(x)) \enspace . \]
    Otherwise, $\mathbf{A}$ is a \emph{proper class}.
    \end{definition}

    \begin{definition}[Subset]
        If $A$ is a set and $\mathbf{B}$ is a class, we say $A$ is a \emph{subset} of $\mathbf{B}$
        iff $A \subseteq \mathbf{B}$.
    \end{definition}

    \begin{axiom}[Empty Set]
        The empty class is a set, called the \emph{empty set}.
    \end{axiom}

    \begin{axiom}[Pairing]
        For any objects $a$ and $b$, the class $\{a,b\}$ is a set, called a \emph{pair set}.
    \end{axiom}

    \begin{definition}[Union]
        For any class of sets $\mathbf{A}$, the \emph{union} $\bigcup \mathbf{A}$ is the class
        $\{ x : \exists A \in \mathbf{A}. x \in A \}$.

        We write $\bigcup_{P[x_1, \ldots, x_n]} t[x_1, \ldots, x_n]$ for $\bigcup \{ t[x_1, \ldots, x_n]
        : P[x_1, \ldots, x_n] \}$.
    \end{definition}

    \begin{proposition}
        If $\mathbf{A} \subseteq \mathbf{B}$ then $\bigcup \mathbf{A} \subseteq \bigcup \mathbf{B}$.
    \end{proposition}

    \begin{proof}
        \pf\ Easy. \qed
    \end{proof}

    \begin{axiom}[Union]
        For any set $A$, the union $\bigcup A$ is a set.
    \end{axiom}

    \begin{proposition}
        For any sets $A$ and $B$, the class $A \cup B$ is a set.
    \end{proposition}

    \begin{proof}
        \pf\ It is $\bigcup \{ A, B \}$. \qed
    \end{proof}

    \begin{props}
        For any objects $a_1$, \ldots, $a_n$, the class $\{ a_1, \ldots, a_n \}$ is a set.
    \end{props}

    \begin{proof}
        \pf\ By repeated application of the Pairing and Union axioms. \qed
    \end{proof}

    \begin{definition}[Power Set]
        For any set $A$, the \emph{power set} of $A$, $\mathcal{P} A$, is the class of all subsets of $A$.
    \end{definition}

    \begin{axiom}[Power Set]
        For any set $A$, the class $\mathcal{P} A$ is a set.
    \end{axiom}

    \begin{axiom}[Subset, Aussonderung]
        For any class $\mathbf{A}$ and set $B$, if $\mathbf{A} \subseteq B$ then $\mathbf{A}$ is a set.
    \end{axiom}

    \begin{proposition}
        For any set $A$ and class $\mathbf{B}$, the intersection $A \cap \mathbf{B}$ is a set.
    \end{proposition}

    \begin{proof}
        \pf\ By the Subset Axiom since it is a subclass of $A$. \qed
    \end{proof}

    \begin{proposition}
        For any set $A$ and class $\mathbf{B}$, the relative complement $A - \mathbf{B}$ is a set.
    \end{proposition}

    \begin{proof}
        \pf\ By the Subset Axiom since it is a subclass of $A$. \qed
    \end{proof}

    \begin{theorem}
        The universal class $\mathbf{V}$ is a proper class.
    \end{theorem}

    \begin{proof}
        \pf
        \step{1}{\assume{$\mathbf{V}$ is a set.}}
        \step{2}{\pflet{$R = \{ x : x \notin x \}$}}
        \step{3}{$R$ is a set.}
        \begin{proof}
            \pf\ By the Subset Axiom.
        \end{proof}
        \step{4}{$R \in R$ if and only if $R \notin R$}
        \qedstep
        \begin{proof}
            \pf\ This is a contradiction.
        \end{proof}
        \qed
    \end{proof}
    
    \begin{definition}[Intersection]
        For any class of sets $\mathbf{A}$, the \emph{intersection} $\bigcap \mathbf{A}$ is the class
        $\{ x : \forall A \in \mathbf{A}. x \in A \}$.

        We write $\bigcap_{P[x_1, \ldots, x_n]} t[x_1, \ldots, x_n]$ for $\bigcap \{ t[x_1, \ldots, x_n]
        : P[x_1, \ldots, x_n] \}$.
    \end{definition}

    \begin{proposition}
        For any nonempty class of sets $\mathbf{A}$, the class $\bigcap \mathbf{A}$ is a set.
    \end{proposition}

    \begin{proof}
        \pf\ Pick $A \in \mathbf{A}$. Then $\bigcap \mathbf{A} \subseteq A$. \qed
    \end{proof}
    
    \begin{proposition}
        If $\mathbf{A} \subseteq \mathbf{B}$ then $\bigcap \mathbf{B} \subseteq \bigcap \mathbf{A}$.
    \end{proposition}

    \begin{proof}
        \pf\ Easy. \qed
    \end{proof}

    \begin{proposition}
        For any set $A$ and class of sets $\mathbf{B}$, we have
        \[ A \cup \bigcap \mathbf{B} = \bigcap \{ A \cup X \mid X \in \mathbf{B} \} \]
    \end{proposition}

    \begin{proof}
        \pf\ Easy. \qed
    \end{proof}

    \begin{proposition}
        For any set $A$ and class of sets $\mathbf{B}$, we have
        \[ A \cap \bigcup \mathbf{B} = \bigcup \{ A \cap X \mid X \in \mathbf{B} \} \]
    \end{proposition}

    \begin{proof}
        \pf\ Easy. \qed
    \end{proof}

    \begin{proposition}
        For any set $C$ and class of sets $\mathbf{A}$, we have
        \[ C - \bigcup \mathbf{A} = \bigcap \{ C - X \mid X \in \mathbf{A} \} \enspace . \]
    \end{proposition}

    \begin{proof}
        \pf\ Easy. \qed
    \end{proof}

    \begin{proposition}
        For any set $C$ and class of sets $\mathbf{A}$, we have
        \[ C - \bigcap \mathbf{A} = \bigcup \{ C - X \mid X \in \mathbf{A} \} \enspace . \]
    \end{proposition}

    \begin{proof}
        \pf\ Easy. \qed
    \end{proof}

    \section{Ordered Pairs}

    \begin{definition}[Ordered Pair]
        For any objects $a$ and $b$, the \emph{ordered pair} $(a,b)$ is $\{ \{ a \}, \{a, b \} \}$.
        We call $a$ its \emph{first coordinate} and $b$ its \emph{second coordinate}.
    \end{definition}

    \begin{theorem}
        For any objects $(a,b)$, we have $(a,b) = (c,d)$ if and only if $a = c$ and $b = d$.
    \end{theorem}

    \begin{proof}
        \pf
        \step{1}{If $(a,b) = (c,d)$ then $a = c$ and $b = d$}
        \begin{proof}
            \step{a}{\assume{$(a,b) = (c,d)$}}
            \step{b}{$a = c$}
            \begin{proof}
                \pf\ Since $\{a\} = \bigcap (a,b) = \bigcap (c,d) = \{c\}$.
            \end{proof}
            \step{c}{$\{a,b\} = \{c,d\}$}
            \begin{proof}
                \pf\ $\{a,b\} = \bigcup (a,b) = \bigcup (c,d) = \{c,d\}$.
            \end{proof}
            \step{cc}{$b = c$ or $b = d$}
            \step{d}{\case{$b = c$}}
            \begin{proof}
                \step{i}{$a = b$}
                \step{ii}{$\{c,d\} = \{a\}$}
                \step{iii}{$b = d$}
            \end{proof}
            \step{e}{\case{$b = d$}}
            \begin{proof}
                \pf\ We have $a = c$ and $b = d$ as required.
            \end{proof}
        \end{proof}
        \step{2}{If $a = c$ and $b = d$ then $(a,b) = (c,d)$}
        \begin{proof}
            \pf\ Trivial.
        \end{proof}
        \qed
    \end{proof}

    \begin{definition}[Cartesian Product]
        The \emph{Cartesian product} of classes $\mathbf{A}$ and $\mathbf{B}$ is the class
        \[ \mathbf{A} \times \mathbf{B} = \{ (x,y) : x \in \mathbf{A}, y \in \mathbf{B} \} \enspace . \]
    \end{definition}

    \begin{lemma}
        For any objects $x$ and $y$ and set $C$, if $x \in C$ and $y \in C$ then $(x,y) \in \mathcal{PP} C$.
    \end{lemma}

    \begin{proof}
        \pf\ Easy. \qed
    \end{proof}

    \begin{corollary}
        For any sets $A$ and $B$, the Cartesian product $A \times B$ is a set.
    \end{corollary}

    \begin{proof}
        \pf\ By the Subset Axiom applied to $\mathcal{PP}(A \cup B)$. \qed
    \end{proof}

    \begin{lemma}
        If $(x,y) \in \mathbf{A}$ then $x, y \in \bigcup \bigcup \mathbf{A}$.
    \end{lemma}

    \begin{proof}
        \pf\ Easy. \qed
    \end{proof}

    \section{Relations}

    \begin{definition}[Relation]
        A \emph{relation} is a class of ordered pairs. It is \emph{small} iff it is a set.

        When $\mathbf{R}$ is a relation, we write $x \mathbf{R} y$ for $(x,y) \in \mathbf{R}$.
    \end{definition}

    \begin{definition}[Domain]
        The \emph{domain} of a class $\mathbf{R}$ is $\dom \mathbf{R} = \{ x : \exists y. (x,y) \in \mathbf{R} \}$.
    \end{definition}

    \begin{definition}[Range]
        The \emph{range} of a class $\mathbf{R}$ is $\ran \mathbf{R} = \{ y : \exists x. (x,y) \in \mathbf{R} \}$.
    \end{definition}

    \begin{definition}[Field]
        The \emph{field} of a class $\mathbf{R}$ is $\fld \mathbf{R} = \dom \mathbf{R} \cup \ran \mathbf{R}$.
    \end{definition}

    \begin{proposition}
        If $R$ is a set then $\dom R$, $\ran R$ and $\fld R$ are sets.
    \end{proposition}

    \begin{proof}
        \pf\ Apply the Subset Axiom to $\bigcup \bigcup R$. \qed
    \end{proof}
    
    \begin{definition}[Single-Rooted]
        A class $\mathbf{R}$ is \emph{single-rooted} iff, for all $y \in \ran \mathbf{R}$,
        there is only one $x$ such that $x \mathbf{R} y$.
    \end{definition}

    \begin{definition}[Inverse]
        The \emph{inverse} of a class $\mathbf{F}$ is the class $\mathbf{F}^{-1} = \{ (y,x)
        \mid (x,y) \in \mathbf{F} \}$.
    \end{definition}

    \begin{theorem}
        For any class $\mathbf{F}$, we have $\dom \mathbf{F}^{-1} = \ran \mathbf{F}$
        and $\ran \mathbf{F}^{-1} = \dom \mathbf{F}$.
    \end{theorem}

    \begin{proof}
        \pf\ Easy. \qed
    \end{proof}

    \begin{theorem}
        For a relation $\mathbf{F}$, $(\mathbf{F}^{-1})^{-1} = \mathbf{F}$.
    \end{theorem}

    \begin{proof}
        \pf\ Easy. \qed
    \end{proof}

    \begin{definition}[Composition]
        The \emph{composition} of classes $\mathbf{F}$ and $\mathbf{G}$ is the class
        $\mathbf{G} \circ \mathbf{F} = \{ (x,z) \mid \exists y. (x,y) \in \mathbf{F} \wedge (y,z) \in \mathbf{G} \}$.
    \end{definition}

    \begin{theorem}
        For any classes $\mathbf{F}$ and $\mathbf{G}$, $\inv{(\mathbf{F} \circ \mathbf{G})} =
        \inv{\mathbf{G}} \circ \inv{\mathbf{F}}$.
    \end{theorem}

    \begin{proof}
        \pf\ Easy. \qed
    \end{proof}

    \begin{definition}[Restriction]
        The \emph{restriction} of the class $\mathbf{F}$ to the class $\mathbf{A}$ is the class
        $\mathbf{F} \restriction \mathbf{A} = \{ (x,y) : x \in A \wedge (x,y) \in \mathbf{F} \}$.
    \end{definition}

    \begin{definition}[Image]
        The \emph{image} of the class $\mathbf{A}$ under the class $\mathbf{F}$ is the class
        $\mathbf{F}(\mathbf{A}) = \{ y : \exists x \in \mathbf{A}. (x,y) \in \mathbf{F} \}$.
    \end{definition}
    
    \begin{theorem}
        \[ \mathbf{F}(\mathbf{A} \cup \mathbf{B}) = \mathbf{F}(\mathbf{A}) \cup \mathbf{F}(\mathbf{B}) \]
    \end{theorem}

    \begin{proof}
        \pf\ Easy. \qed
    \end{proof}

    \begin{theorem}
        \[ \mathbf{F}(\bigcup \mathbf{A}) = \bigcup \{ \mathbf{F}(X) : X \in \mathbf{A} \} \]
    \end{theorem}

    \begin{proof}
        \pf\ Easy. \qed
    \end{proof}
    
    \begin{theorem}
        \[ \mathbf{F}(\mathbf{A} \cap \mathbf{B}) \subseteq \mathbf{F}(\mathbf{A}) \cap \mathbf{F}(\mathbf{B}) \]
        Equality holds if $\mathbf{F}$ is single-rooted.
    \end{theorem}

    \begin{proof}
        \pf\ Easy. \qed
    \end{proof}

    \begin{theorem}
        \[ \mathbf{F}(\bigcap \mathbf{A}) \subseteq \bigcap \{ \mathbf{F}(X) : X \in \mathbf{A} \} \]
        Equality holds if $\mathbf{F}$ is single-rooted.
    \end{theorem}

    \begin{proof}
        \pf\ Easy. \qed
    \end{proof}

    \begin{theorem}
        \[ \mathbf{F}(\mathbf{A}) - \mathbf{F}(\mathbf{B}) \subseteq \mathbf{F}(\mathbf{A} - \mathbf{B}) \]
        Equality holds if $\mathbf{F}$ is single-rooted.
    \end{theorem}

    \begin{proof}
        \pf\ Easy. \qed
    \end{proof}

    \begin{definition}[Reflexive]
        A binary relation $\mathbf{R}$ on $\mathbf{A}$ is \emph{reflexive} on $\mathbf{A}$ if and only if
        $\forall x \in \mathbf{A}. x\mathbf{R}x$.
    \end{definition}

    \begin{definition}[Symmetric]
        A binary relation $\mathbf{R}$ is \emph{symmetric} iff, whenever $x\mathbf{R}y$, then $y\mathbf{R}x$.
    \end{definition}

    \begin{definition}[Transitive]
        A binary relation $\mathbf{R}$ is \emph{transitive} iff, whenever $x\mathbf{R}y$ and $y\mathbf{R}z$,
        then $x\mathbf{R}z$.
    \end{definition}

    \section{$n$-ary Relations}

    \begin{definition}
        Given objects $a$, $b$, $c$, define the \emph{ordered triple}
        $(a,b,c)$ to be $((a,b),c)$.

        Define $(a,b,c,d) = ((a,b,c),d)$, etc.

        Define the \emph{1-tuple} $(a)$ to be $a$.
    \end{definition}

    \begin{definition}[$n$-ary Relation]
        Given a class $\mathbf{A}$, an \emph{$n$-ary relation} on $\mathbf{A}$ is a class of ordered
        $n$-tuples, all of whose components are in $\mathbf{A}$.
    \end{definition}

    \section{Functions}

    \begin{definition}[Function]
        A \emph{function} is a relation $\mathbf{F}$ such that, for all $x \in \dom \mathbf{F}$,
        there is only one $y$ such that $x \mathbf{F} y$. We call this unique $y$ the \emph{value}
        of $\mathbf{F}$ at $x$ and denote it by $\mathbf{F}(x)$.

        We say $\mathbf{F}$ is a function \emph{from} $\mathbf{A}$ \emph{into} $\mathbf{B}$,
        or $\mathbf{F}$ \emph{maps} $\mathbf{A}$ into $\mathbf{B}$, and write $\mathbf{F} : \mathbf{A}
        \rightarrow \mathbf{B}$, iff $\mathbf{F}$ is a function, $\dom \mathbf{F} = \mathbf{A}$,
        and $\ran \mathbf{F} \subseteq \mathbf{B}$.

        If, in addition, $\ran \mathbf{F} = \mathbf{B}$, we say $\mathbf{F}$ is a function from $\mathbf{A}$
        \emph{onto} $\mathbf{B}$.
    \end{definition}

    \begin{theorem}
        For a class $\mathbf{F}$, $\inv{\mathbf{F}}$ is a function if and only if $\mathbf{F}$ is single-rooted.
    \end{theorem}

    \begin{proof}
        \pf\ Easy. \qed
    \end{proof}

    \begin{theorem}
        A relation $\mathbf{F}$ is a function if and only if $\inv{\mathbf{F}}$ is single-rooted.
    \end{theorem}

    \begin{proof}
        \pf\ Easy. \qed
    \end{proof}

    \begin{theorem}
        For any function $\mathbf{G}$ and classes $\mathbf{A}$ and $\mathbf{B}$,
        \begin{align*}
            \inv{\mathbf{G}}(\bigcup \mathbf{A}) & = \bigcup \{ \inv{\mathbf{G}}(X) : X \in \mathbf{A} \} \\
            \inv{\mathbf{G}}(\bigcap \mathbf{A}) & = \bigcap \{ \inv{\mathbf{G}}(X) : X \in \mathbf{A} \} &
            (\text{if } \mathbf{A} \neq \emptyset) \\
            \inv{\mathbf{G}}(\mathbf{A} - \mathbf{B}) & = \inv{\mathbf{G}}(\mathbf{A}) - \inv{\mathbf{G}}(\mathbf{B})
        \end{align*}
    \end{theorem}

    \begin{proof}
        \pf\ Easy. \qed
    \end{proof}

    \begin{theorem}
        Assume that $\mathbf{F}$ and $\mathbf{G}$ are functions. Then $\mathbf{F} \circ \mathbf{G}$
        is a function, its domain is $\{ x \in \dom \mathbf{G} : \mathbf{G}(x) \in \dom \mathbf{F} \}$,
        and for $x$ in its domain,
        \[ (\mathbf{F} \circ \mathbf{G})(x) = \mathbf{F}(\mathbf{G}(x)) \enspace . \]
    \end{theorem}

    \begin{proof}
        \pf\ Easy. \qed
    \end{proof}

    \begin{definition}[One-to-one]
        A function $\mathbf{F}$ is \emph{one-to-one} or an \emph{injection} iff it is single-rooted.
    \end{definition}

    \begin{theorem}
        Let $\mathbf{F}$ be a one-to-one function. For $x \in \dom \mathbf{F}$, $\inv{\mathbf{F}}(\mathbf{F}(x)) = x$.
    \end{theorem}

    \begin{proof}
        \pf\ Easy. \qed
    \end{proof}

    \begin{theorem}
        Let $\mathbf{F}$ be a one-to-one function. For $y \in \ran \mathbf{F}$, $\mathbf{F}(\inv{\mathbf{F}}(y)) = y$.
    \end{theorem}

    \begin{proof}
        \pf\ Easy. \qed
    \end{proof}

    \begin{definition}[Identity Function]
        For any class $\mathbf{A}$, the \emph{identity} function on $\mathbf{A}$ is $\mathrm{id}_\mathbf{A} =
        \{ (x,x) \mid x \in \mathbf{A} \}$.
    \end{definition}

    \begin{theorem}
        Let $F : A \rightarrow B$. Assume $A \neq \emptyset$. Then $F$ has a left inverse (i.e. there exists
        $G : B \rightarrow A$ such that $G \circ F = \mathrm{id}_A$) if and only if $F$ is one-to-one.
    \end{theorem}

    \begin{proof}
        \pf
        \step{1}{If $F$ is one-to-one then $F$ has a left inverse.}
        \begin{proof}
            \step{a}{\assume{$F$ is one-to-one.}}
            \step{b}{$\inv{F} : \ran F \rightarrow A$}
            \step{c}{\pick\ $a \in A$}
            \step{d}{Define $G : B \rightarrow A$ by:
            \[ G(x) = \begin{cases}
                \inv{F}(x) & \text{if } x \in \ran F \\
                a & \text{if } x \in B - \ran F
            \end{cases} \]}
            \step{f}{$\forall x \in A. G(F(x)) = x$}
        \end{proof}
        \step{2}{If $F$ has a left inverse then $F$ is one-to-one.}
        \begin{proof}
            \step{a}{\assume{$F$ has a left inverse $G$.}}
            \step{b}{\pflet{$x, y \in A$ with $F(x) = F(y)$}}
            \step{c}{$x = y$}
            \begin{proof}
                \pf\ $x = G(F(x)) = G(F(y)) = y$.
            \end{proof}
        \end{proof}
        \qed
    \end{proof}

    \begin{definition}[Binary Operation]
        A \emph{binary operation} on a set $A$ is a function from $A \times A$ into $A$.
    \end{definition}

    \section{The Axiom of Choice}

    \begin{axiom}[Choice]
        For any relation $R$ there exists a function $H \subseteq R$ with $\dom H = \dom R$.
    \end{axiom}

    \begin{theorem}
        Let $F : A \rightarrow B$. Then $F$ has a right inverse if and only if $F$ maps $A$ onto $B$.
    \end{theorem}

    \begin{proof}
        \pf
        \step{1}{If $F$ has a right inverse then $F$ maps $A$ onto $B$.}
        \begin{proof}
            \pf\ If $H : B \rightarrow A$ is a right inverse, then for any $y$ in $B$, we have
            $y = F(H(y))$.
        \end{proof}
        \step{2}{If $F$ maps $A$ onto $B$ then $F$ has a right inverse.}
        \begin{proof}
            \step{a}{\assume{$F$ maps $A$ onto $B$.}}
            \step{b}{\pick\ a function $H$ with $H \subseteq \inv{F}$ and $\dom H = \dom \inv{F}$}
            \begin{proof}
                \pf\ By the Axiom of Choice.
            \end{proof}
            \step{c}{$\dom H = B$}
            \begin{proof}
                \pf\ $\dom H = \dom \inv{F} = \ran F = B$ by \stepref{a}.
            \end{proof}
            \step{d}{For all $y \in B$ we have $F(H(y)) = y$}
            \begin{proof}
                \step{i}{\pflet{$y \in B$}}
                \step{ii}{$(y,H(y)) \in \inv{F}$}
                \step{iii}{$F(H(y)) = y$}
            \end{proof}
        \end{proof}
        \qed
    \end{proof}

    \section{Sets of Functions}

    \begin{definition}
        Let $A$ be a set and $\mathbf{B}$ be a class. Then $\mathbf{B}^A$ is the class of all functions
        $A \rightarrow \mathbf{B}$.
    \end{definition}

    \section{Dependent Products}

    \begin{definition}
        Let $I$ be a set and $H_i$ a set for all $i \in I$. Define
        \[ \prod_{i \in I} H_i = \{ f : \text{$f$ is a function}, \dom f = I, \forall i \in I. f(i) \in H_i \} \enspace . \]
    \end{definition}

    \begin{theorem}
        The Axiom of Choice is equivalent to the statement: For any set $I$ and any function $H$ with domain
        $I$, if $H(i) \neq \emptyset$ for all $i \in I$, then $\prod_{i \in I} H(i) \neq \emptyset$
    \end{theorem}

    \begin{proof}
        \pf
        \step{1}{If the Axiom of Choice is true then, for any set $I$ and any function $H$ with domain
        $I$, if $H(i) \neq \emptyset$ for all $i \in I$, then $\prod_{i \in I} H(i) \neq \emptyset$.}
        \begin{proof}
            \step{a}{\assume{The Axiom of Choice.}}
            \step{b}{\pflet{$I$ be a set.}}
            \step{c}{\pflet{$H$ be a function with domain $I$.}}
            \step{d}{\assume{$H(i) \neq \emptyset$ for all $i \in I$.}}
            \step{e}{\pflet{$R = \{ (i,x) : i \in I, x \in H(i) \}$}}
            \step{f}{\pick\ a function $F \subseteq R$ with $\dom F = \dom R$ \prove{$F \in \prod_{i \in I} H(i)$}}
            \begin{proof}
                \pf\ By the Axiom of Choice.
            \end{proof}
            \step{g}{$\dom H = I$}
            \begin{proof}
                \pf\ We have $\dom R = I$ since for all $i \in I$ there exists $x$ such that $x \in H(i)$.
            \end{proof}
            \step{h}{$\forall i \in I. F(i) \in H(i)$}
            \begin{proof}
                \pf\ Since $iRF(i)$.
            \end{proof}
        \end{proof}
        \step{2}{If, for any set $I$ and any function $H$ with domain
        $I$, if $H(i) \neq \emptyset$ for all $i \in I$, then $\prod_{i \in I} H(i) \neq \emptyset$,
        then the Axiom of Choice is true.}
        \begin{proof}
            \step{a}{\assume{For any set $I$ and any function $H$ with domain
            $I$, if $H(i) \neq \emptyset$ for all $i \in I$, then $\prod_{i \in I} H(i) \neq \emptyset$}}
            \step{b}{\pflet{$R$ be a relation}}
            \step{c}{\pflet{$I = \dom R$}}
            \step{d}{Define the function $H$ with domain $I$ by: for $i \in I$, $H(i) = \{ y : iRy \}$}
            \step{e}{$H(i) \neq \emptyset$ for all $i \in I$}
            \step{f}{\pick\ $F \in \prod_{i \in I} H(i)$}
            \begin{proof}
                \pf\ By \stepref{a}
            \end{proof}
            \step{g}{$F$ is a function}
            \step{h}{$F \subseteq R$}
            \begin{proof}
                \pf\ For all $i \in I$ we have $F(i) \in H(i)$ and so $iRF(i)$.
            \end{proof}
            \step{i}{$\dom F = \dom R$}
        \end{proof}
        \qed
    \end{proof}

    \begin{theorem}
        The following are equivalent.
        \begin{enumerate}
            \item The Axiom of Choice.
            \item Let $\mathcal{A}$ be a set such that (a) every member of $\mathcal{A}$ is a nonempty set,
            and (b) any two distinct members of $\mathcal{A}$ are disjoint. Then there exists a set $C$
            such that, for all $B \in \mathcal{A}$, we have $C \cap B$ is a singleton.
            \item For any set $A$, there exists a function $F : \mathcal{P} A - \{ \emptyset \} \rightarrow
            A$ such that $F(X) \in X$ for all $X \in \mathcal{P} A - \{ \emptyset \}$.
        \end{enumerate}
    \end{theorem}

    \begin{proof}
        \pf
        \step{1}{$1 \Rightarrow 2$}
        \begin{proof}
        \pf\ Let $\mathcal{A}$ be a set matching the two condtions. By the Multiplicative Axiom, pick a function
        $f \in \prod_{B \in \mathcal{A}} B$. Let $C = \ran f$. Then $C \cap B = \{f(B)\}$ for all $B \in
        \mathcal{A}$.
        \end{proof}
        \step{2}{$2 \Rightarrow 3$}
        \begin{proof}
            \step{a}{\assume{2}}
            \step{b}{\pflet{$A$ be a set.}}
            \step{c}{\pflet{$\mathcal{A} = \{ \{B\} \times B : B \in \mathcal{P} A - \{ \emptyset \} \}$}}
            \step{d}{\pick\ a set $C$ such that $C \cap (\{ B \} \times B)$ is a singleton
            for all $B \in \mathcal{P} A - \{ \emptyset \}$}
            \step{e}{\pflet{$F = C \cap \bigcup \mathcal{A}$}}
            \step{f}{$F : \mathcal{P} A - \{ \emptyset \} \rightarrow A$ is a function and $F(X) \in X$
            for all $X$}
        \end{proof}
        \step{3}{$3 \Rightarrow 1$}
        \begin{proof}
            \step{a}{\assume{3}}
            \step{b}{\pflet{$R$ be a relation}}
            \step{c}{\pick\ a choice function $G$ for $\ran R$}
            \step{d}{Define $F : \dom R \rightarrow \ran R$ by $F(x) = G(R(x))$}
            \step{e}{$F \subseteq R$}
        \end{proof}
        \qed
    \end{proof}
    \section{Equivalence Relations}

    \begin{definition}[Equivalence Relation]
        An \emph{equivalence relation} on $\mathbf{A}$ is a binary relation on $\mathbf{A}$
        that is reflexive on $\mathbf{A}$, symmetric and transitive.
    \end{definition}

    \begin{theorem}
        If $\mathbf{R}$ is a symmetric and transitive relation then $\mathbf{R}$ is an equivalence relation
        on $\fld \mathbf{R}$.
    \end{theorem}

    \begin{proof}
        \pf
        \step{1}{\pflet{$x \in \fld \mathbf{R}$}}
        \step{2}{\pick\ $y$ such that either $x\mathbf{R}y$ or $y\mathbf{R}x$}
        \step{3}{$x\mathbf{R}y$ and $y\mathbf{R}x$}
        \begin{proof}
            \pf\ Since $\mathbf{R}$ is symmetric.
        \end{proof}
        \step{4}{$x\mathbf{R}x$}
        \begin{proof}
            \pf\ Since $\mathbf{R}$ is transitive.
        \end{proof}
        \qed
    \end{proof}

    \begin{definition}[Equivalence Class]
        If $\mathbf{R}$ is an equivalence relation and $x \in \fld \mathbf{R}$, the \emph{equivalence class}
        of $x$ modulo $\mathbf{R}$ is
        \[ [x]_{\mathbf{R}} = \{ t : x \mathbf{R} t \} \enspace . \]
    \end{definition}

    \begin{lemma}
        Assume that $\mathbf{R}$ is an equivalence relation on $\mathbf{A}$ and that $x$ and $y$ belong to
        $\mathbf{A}$. Then
        \[ [x]_{\mathbf{R}} = [y]_{\mathbf{R}} \text{ iff } x \mathbf{R} y \enspace . \]
    \end{lemma}

    \begin{proof}
        \pf
        \step{1}{If $[x]_{\mathbf{R}} = [y]_{\mathbf{R}}$ then $x\mathbf{R}y$}
        \begin{proof}
            \step{a}{\assume{$[x]_{\mathbf{R}} = [y]_{\mathbf{R}}$}}
            \step{b}{$y \in [y]_{\mathbf{R}}$}
            \begin{proof}
                \pf\ Since $\mathbf{R}$ is reflexive on $\mathbf{A}$.
            \end{proof}
            \step{c}{$y \in [x]_{\mathbf{R}}$}
            \step{d}{$x\mathbf{R}y$}
        \end{proof}
        \step{2}{If $x\mathbf{R}y$ then $[x]_{\mathbf{R}} = [y]_{\mathbf{R}}$}
        \begin{proof}
            \step{a}{\assume{$x \mathbf{R} y$}}
            \step{b}{$[y]_{\mathbf{R}} \subseteq [x]_{\mathbf{R}}$}
            \begin{proof}
                \step{i}{\pflet{$z \in [y]_{\mathbf{R}}$}}
                \step{ii}{$y\mathbf{R}z$}
                \step{iii}{$x\mathbf{R}z$}
                \begin{proof}
                    \pf\ Since $\mathbf{R}$ is transitive.
                \end{proof}
                \step{iv}{$z \in [x]_{\mathbf{R}}$}
            \end{proof}
            \step{c}{$y\mathbf{R}x$}
            \begin{proof}
                \pf\ Since $\mathbf{R}$ is symmetric.
            \end{proof}
            \step{d}{$[x]_{\mathbf{R}} \subseteq [y]_{\mathbf{R}}$}
            \begin{proof}
                \pf\ Similar.
            \end{proof}
        \end{proof}
        \qed
    \end{proof}

    \begin{definition}[Partition]
        A \emph{partition} of a set $A$ is a set $P \subseteq \mathcal{P} A$ such that:
        \begin{itemize}
            \item Every member of $P$ is nonempty.
            \item Any two distinct members of $P$ are disjoint.
            \item $A = \bigcup P$
        \end{itemize}
    \end{definition}

    \begin{theorem}
        Let $R$ be an equivalence relation on the set $A$. Then the set of all equivalence classes is a
        partition of $A$.
    \end{theorem}

    \begin{proof}
        \pf
        \step{1}{Every equivalence class is nonempty.}
        \begin{proof}
            \pf\ For any $x \in A$ we have $x \in [x]_R$.
        \end{proof}
        \step{2}{Any two distinct equivalence classes are disjoint.}
        \begin{proof}
            \step{a}{\pflet{$x, y \in A$}}
            \step{b}{\assume{$z \in [x]_R \cap [y]_R$} \prove{$[x]_R = [y]_R$}}
            \step{c}{$xRy$}
            \begin{proof}
                \step{i}{$xRz$}
                \step{ii}{$yRz$}
                \step{iii}{$zRy$}
                \begin{proof}
                    \pf\ By \stepref{ii} and symmetry.
                \end{proof}
                \step{iv}{$xRy$}
                \begin{proof}
                    \pf\ By \stepref{i}, \stepref{iii} and transitivity.
                \end{proof}
            \end{proof}
            \step{d}{$[x]_R = [y]_R$}
            \begin{proof}
                \pf\ By Lemma 3N.
            \end{proof}
        \end{proof}
        \step{3}{$A$ is the union of all the equivalence classes.}
        \begin{proof}
            \pf\ For any $x \in A$ we have $x \in [x]_R$.
        \end{proof}
        \qed
    \end{proof}

    \begin{definition}[Quotient Set]
        If $R$ is an equivalence relation on the set $A$, then the \emph{quotient set} $A / R$
        is the set of all equivalence classes, and the \emph{natural map} or \emph{canonical map}
        $\phi : A \rightarrow A/R$ is defined by $\phi(x) = [x]_R$.
    \end{definition}

    \begin{theorem}
        Assume that $R$ is an equivalence relation on $A$ and that $F : A \rightarrow B$.
        Assume that $F$ is \emph{compatible} with $R$; that is, whenever $xRy$, then $F(x) = F(y)$.
        Then there exists a unique $\overline{F} : A / R \rightarrow B$ such that $F = \overline{F} \circ \phi$.
    \end{theorem}

    \begin{proof}
        \pf\ The unique such $\overline{F}$ is $\{ ([x],F(x)) : x \in A \}$. \qed
    \end{proof}

    \section{Partial Orders}

    \begin{definition}[Strict Partial Order]
        A \emph{strict partial order} is an irreflexive, transitive relation.

        If $<$ is a strict partial order, we write $x \leq y$ for $x < y \vee x = y$.
    \end{definition}

    \begin{theorem}
        Assume that $<$ is a partial order. Then for any $x$, $y$ and $z$:
        \begin{enumerate}
            \item \emph{At most} one of the three alternatives,
            \[ x < y, x = y, y < x, \]
            can hold.
            \item $x \leq y \leq x \Rightarrow x = y$.
        \end{enumerate}
    \end{theorem}

    \begin{proof}
        \pf\ Easy. \qed
    \end{proof}

    \begin{definition}[Minimal]
        Let $<$ be a partial order on $D$. An element $m \in D$ is \emph{minimal} iff there is no
        $x \in D$ such that $x < m$.
    \end{definition}

    \begin{definition}[Maximal]
        Let $<$ be a partial order on $D$. An element $m \in D$ is \emph{maximal} iff there is no
        $x \in D$ such that $m < x$.
    \end{definition}

    \begin{definition}[Least]
        Let $<$ be a partial order on $D$. An element $m \in D$ is \emph{least}, \emph{smallest} or the
        \emph{minimum} iff $\forall x \in D. m \leq x$.
    \end{definition}

    \begin{definition}[Greatest]
        Let $<$ be a partial order on $D$. An element $m \in D$ is \emph{greatest}, \emph{largest} or the
        \emph{maximum} iff $\forall x \in D. x \leq m$.
    \end{definition}

    \begin{proposition}
        If $R$ is a partial ordering on $D$ then so is $\inv{R}$.
    \end{proposition}

    \begin{proof}
        \pf\ Easy. \qed
    \end{proof}

    \begin{definition}[Upper Bound]
        Let $<$ be a partial order on $A$ and $C \subseteq A$. An \emph{upper bound} for $C$ is an element
        $b \in A$ such that $\forall x \in C. x \leq b$.
    \end{definition}

    \begin{definition}[Least Upper Bound]
        Let $<$ be a partial order on $A$ and $C \subseteq A$. The \emph{least upper bound} or \emph{supremum}
        for $C$ is the least element in the set of upper bounds for $C$.
    \end{definition}
    
    \section{Linear Orders}

    \begin{definition}[Linear Ordering]
        Let $\mathbf{A}$ be a class. A \emph{linear ordering} or \emph{total ordering} on $\mathbf{A}$
        is a relation $\mathbf{R}$ on $\mathbf{A}$ such that:
        \begin{itemize}
            \item $\mathbf{R}$ is transitive.
            \item $\mathbf{R}$ satisfies \emph{trichotomy} on $\mathbf{A}$; i.e. for any $x, y \in \mathbf{A}$,
            exactly one of
            \[ x\mathbf{R}y, x=y, y\mathbf{R}x \]
            holds.
        \end{itemize}
    \end{definition}

    \begin{theorem}
        Let $\mathbf{R}$ be a linear ordering on $\mathbf{A}$.
        \begin{enumerate}
            \item There is no $x$ such that $x \mathbf{R} x$.
            \item For distinct $x$ and $y$ in $\mathbf{A}$, either $x\mathbf{R}y$ or $y\mathbf{R}x$.
        \end{enumerate}
    \end{theorem}

    \begin{proof}
        \pf\ Immediate from trichotomy. \qed
    \end{proof}

    \begin{definition}[Strictly Monotone Functions]
        Let $A$ and $B$ be linearly ordered sets. A function $f : A \rightarrow B$ is \emph{strictly
        monotone} iff, for all $x, y \in A$, if $x < y$ then $f(x) < f(y)$.
    \end{definition}

    \begin{theorem}
        Let $A$ and $B$ be linearly ordered sets and $f : A \rightarrow B$ be strictly monotone.
        For all $x, y \in A$, if $f(x) < f(y)$ then $x < y$.
    \end{theorem}

    \begin{proof}
        \pf\ We have $f(x) \neq f(y)$ and $f(y) \not < f(x)$ by trichotomy, hence $x \neq y$ and $y \not < x$
        since $f$ is strictly monotone, hence $x < y$ by trichotomy. \qed
    \end{proof}

    \begin{theorem}
        Every strictly monotone function is injective.
    \end{theorem}

    \begin{proof}
        \pf\ If $f(x) = f(y)$, then we have $f(x) \not < f(y)$ and $f(y) \not < f(x)$ by trichotomy,
        hence $x \not < y$ and $y \not < x$ since $f$ is strictly monotone, hence $x = y$ by
        trichotomy. \qed
    \end{proof}
    \section{Natural Numbers}

    \begin{definition}[Successor]
        The \emph{successor} of a set $a$ is the set $a^+ = a \cup \{ a \}$.
    \end{definition}

    \begin{definition}[Inductive]
        A class $\mathbf{A}$ is \emph{inductive} iff $\emptyset \in \mathbf{A}$ and
        $\forall a \in \mathbf{A}. a^+ \in \mathbf{A}$.
    \end{definition}

    \begin{axiom}[Infinity]
        There exists an inductive set.
    \end{axiom}

    \begin{definition}[Natural Number]
        A \emph{natural number} is a set that belongs to every inductive set.

        We write $\omega$ for the class of all natural numbers.
    \end{definition}

    \begin{theorem}
        The class $\omega$ is a set.
    \end{theorem}

    \begin{proof}
        \pf\ Pick an inductive set $I$ (by the Axiom of Infinity), then apply a Subset Axiom to $I$. \qed
    \end{proof}

    \begin{theorem}
        The set $\omega$ is inductive, and is a subset of every inductive set.
    \end{theorem}

    \begin{proof}
        \pf\ Easy. \qed
    \end{proof}

    \begin{corollary}[Proof by Induction]
        Any inductive subclass of $\omega$ is equal to $\omega$.
    \end{corollary}

    \begin{theorem}
        Every natural number except 0 is the successor of some natural number.
    \end{theorem}

    \begin{proof}
        \pf\ Easy proof by induction. \qed
    \end{proof}

    \begin{definition}[Peano System]
        A \emph{Peano system} is a triple $\langle N, S, e \rangle$ consisting of a set $N$,
        a function $S : N \rightarrow N$ and an element $e \in N$ such that:
        \begin{enumerate}
            \item $e \notin \ran S$
            \item $S$ is one-to-one
            \item Any subset $A \subseteq N$ that contains $e$ and is closed under $S$ equals $N$.
        \end{enumerate}
    \end{definition}

    \begin{definition}[Transitive Set]
        A set $A$ is a \emph{transitive set} iff every member of a member of $A$ is a member of $A$.
    \end{definition}

    \begin{theorem}
        \label{theorem:union_transitive_successor}
        For any transitive set $a$, $\bigcup (a^+) = a$.
    \end{theorem}

    \begin{proof}
        \pf
        \begin{align*}
            \bigcup (a^+) & = \bigcup (a \cup \{ a \}) \\
            & = \bigcup a \cup \bigcup \{a\} \\
            & = \bigcup a \cup a \\
            & = a
        \end{align*}
        since $\bigcup a \subseteq a$. \qed
    \end{proof}

    \begin{theorem}
        \label{theorem:natural_number_transitive}
        Every natural number is a transitive set.
    \end{theorem}

    \begin{proof}
        \pf
        \step{1}{0 is a transitive set.}
        \begin{proof}
            \pf\ Vacuous.
        \end{proof}
        \step{2}{For any natural number $n$, if $n$ is a transitive set then $n^+$ is a transitive set.}
        \begin{proof}
            \step{a}{\pflet{$n$ be a natural number that is a transitive set.}}
            \step{b}{$\bigcup (n^+) \subseteq n^+$}
            \begin{proof}
                \pf\ Theorem \ref{theorem:union_transitive_successor}.
            \end{proof}
        \end{proof}
        \qed
    \end{proof}

    \begin{theorem}
        $\langle \omega, \sigma, 0 \rangle$ is a Peano system, where $0 = \emptyset$ and 
        $\sigma = \{ \langle n, n^+ \rangle : n \in \omega \}$.
    \end{theorem}

    \begin{proof}
        \pf
        \step{1}{$0 \notin \ran \sigma$}
        \begin{proof}
            \pf\ For any $n \in \omega$ we have $0 \neq n^+$ since $n \in n^+$
            and $n \notin 0$.
        \end{proof}
        \step{2}{$\sigma$ is one-to-one.}
        \begin{proof}
            \pf\ If $m^+ = n^+$ then $m = \bigcup (m^+) = \bigcup (n^+) = n$ using Theorems
            \ref{theorem:union_transitive_successor} and \ref{theorem:natural_number_transitive}.
        \end{proof}
        \step{3}{Any subset $A \subseteq \omega$ that contains 0 and is closed under $\sigma$ equals $\omega$.}
        \qed
    \end{proof}

    \begin{theorem}
        The set $\omega$ is a transitive set.
    \end{theorem}

    \begin{proof}
        \pf
        \step{1}{For every natural number $n$ we have $\forall m \in n$. $m$ is a natural number.}
        \begin{proof}
            \step{a}{$\forall m \in 0$. $m$ is a natural number.}
            \begin{proof}
                \pf\ Vacuous.
            \end{proof}
            \step{b}{If $n$ is a natural number and $\forall m \in n$. $m$ is a natural number, then
            $\forall m \in n^+$. $m$ is a natural number.}
            \begin{proof}
                \pf\ Since if $m \in n^+$ we have either $m \in n$ or $m = n$, and $m$ is a natural number
                in either case.
            \end{proof}
        \end{proof}
        \qed
    \end{proof}

    \begin{theorem}[Recursion Theorem on $\omega$]
        Let $A$ be a set, $a \in A$ and $F : A \rightarrow A$. Then there exists a unique function
        $h : \omega \rightarrow A$ such that
        \[ h(0) = a \enspace , \]
        and for every $n$ in $\omega$,
        \[ h(n^+) = F(h(n)) \enspace . \]
    \end{theorem}

    \begin{proof}
        \pf
        \step{1}{Let us call a function $v$ \emph{acceptable} iff $\dom v \subseteq \omega$,
        $\ran v \subseteq A$ and:
        \begin{enumerate}
            \item If $0 \in \dom v$ then $v(0) = a$
            \item For all $n \in \omega$, if $n^+ \in \dom v$ then $n \in \dom v$ and $v(n^+) = F(v(n))$.
        \end{enumerate}}
        \step{2}{\pflet{$\mathcal{K}$ be the set of acceptable functions.}}
        \step{3}{\pflet{$h = \bigcup \mathcal{K}$}}
        \step{4}{$h$ is a function.}
        \begin{proof}
            \step{a}{\pflet{$S = \{ n \in \omega : \text{for at most one } y, (n,y) \in h \}$}}
            \step{b}{$S$ is inductive.}
            \begin{proof}
                \step{i}{$0 \in S$}
                \begin{proof}
                    \step{one}{\pflet{$\langle 0, y_1 \rangle, \langle 0, y_2 \rangle \in h$}}
                    \step{two}{\pick\ acceptable $v_1$ and $v_2$ such that $v_1(0) = y_1$ and $v_2(0) = y_2$}
                    \step{three}{$y_1 = a$}
                    \step{four}{$y_2 = a$}
                    \step{five}{$y_1 = y_2$}
                \end{proof}
                \step{ii}{$\forall k \in S. k^+ \in S$}
                \begin{proof}
                    \step{one}{\pflet{$k \in S$}}
                    \step{two}{\pflet{$(k^+,y_1),(k^+,y_2) \in h$}}
                    \step{three}{\pick\ acceptable $v_1$, $v_2$ such that $v_1(k^+)=y_1$ and $v_2(k^+)=y_2$}
                    \step{four}{$y_1 = F(v_1(k))$}
                    \step{five}{$f_2 = F(v_2(k))$}
                    \step{six}{$v_1(k) = v_2(k)$}
                    \begin{proof}
                        \step{A}{$(k,v_1(k)), (k,v_2(k)) \in h$}
                        \qedstep
                        \begin{proof}
                            \pf\ By \stepref{one}
                        \end{proof}
                    \end{proof}
                    \step{seven}{$y_1 = y_2$}
                \end{proof}
            \end{proof}
            \step{c}{$S = \omega$}
        \end{proof}
        \step{5}{$h$ is acceptable.}
        \begin{proof}
            \step{a}{If $0 \in \dom h$ then $h(0)= a$}
            \begin{proof}
                \step{i}{\assume{$0 \in \dom h$}}
                \step{ii}{\pick\ $v$ acceptable with $v(0) = h(0)$}
                \step{iii}{$v(0) = a$}
                \step{iv}{$h(0) = a$}
            \end{proof}
            \step{b}{For all $n \in \omega$, if $n^+ \in \dom h$ then $n \in \dom h$ and $h(n^+) = F(h(n))$}
            \begin{proof}
                \step{i}{\pflet{$n \in \omega$ with $n^+ \in \dom h$}}
                \step{ii}{\pick\ $v$ acceptable with $v(n^+) = h(n^+)$}
                \step{iii}{$n \in \dom v$}
                \step{iv}{$v(n) = h(n)$}
                \step{v}{$h(n^+) = F(h(n))$}
                \begin{proof}
                    \pf
                    \begin{align*}
                        h(n^+) & = v(n^+) \\
                        & = F(v(n)) \\
                        & = F(h(n))
                    \end{align*}
                \end{proof}
            \end{proof}
        \end{proof}
        \step{6}{$\dom h = \omega$}
        \begin{proof}
            \step{a}{$0 \in \dom h$}
            \begin{proof}
                \pf\ Since $\{ (0,a) \}$ is an acceptable function.
            \end{proof}
            \step{b}{$\forall n \in \dom h. n^+ \in \dom h$}
            \begin{proof}
                \step{i}{\pflet{$n \in \dom h$}}
                \step{ii}{\pick\ an acceptable $v$ such that $n \in \dom v$}
                \step{iii}{\assume{w.l.o.g. $n^+ \notin \dom v$}}
                \step{iv}{$v \cup \{ (n^+, F(v(n))) \}$ is acceptable.}
            \end{proof}
        \end{proof}
        \step{7}{For any acceptable function $h' : \omega \rightarrow A$ we have $h' = h$}
        \begin{proof}
            \step{a}{\pflet{$h' : \omega \rightarrow A$ be acceptable.}}
            \step{b}{$h'(0) = h(0)$}
            \begin{proof}
                \pf\ $h'(0) = h(0) = a$
            \end{proof}
            \step{c}{$\forall n \in \omega. h'(n) = h(n) \Rightarrow h'(n^+) = h(n^+)$}
            \begin{proof}
                \pf\ We have $h'(n^+) = F(h'(n)) = F(h(n)) = h(n^+)$.
            \end{proof}
        \end{proof}
        \qed
    \end{proof}

    \begin{theorem}
        Let $(N,S,e)$ be a Peano system. Then $(\omega, \sigma, 0)$ is isomorphic to $(N,S,e)$, i.e. there is
        a function $h$ mapping $\omega$ one-to-one onto $N$ in a way that preserves the successor operation
        \[ h(\sigma(n)) = S(h(n)) \]
        and the zero element
        \[ h(0) = e \enspace . \]
    \end{theorem}

    \begin{proof}
        \pf
        \step{1}{There exists a function $h$ that satisfies those two conditions.}
        \begin{proof}
            \pf\ By the Recursion Theorem.
        \end{proof}
        \step{2}{For all $m,n \in \omega$, if $m \neq n$ then $h(m) \neq h(n)$}
        \begin{proof}
            \step{a}{For all $n \in \omega$, if $n \neq 0$ then $h(n) \neq h(0)$}
            \begin{proof}
                \step{i}{\pflet{$n \in \omega$}}
                \step{ii}{\assume{$n \neq 0$}}
                \step{iii}{\pick\ $p$ such that $n = p^+$}
                \step{iv}{$h(n) \neq h(0)$}
                \begin{proof}
                    \pf\ $h(n) = S(h(p)) \neq e = h(0)$.
                \end{proof}
            \end{proof}
            \step{b}{For all $m \in \omega$, if $\forall n(m \neq n \Rightarrow h(m) \neq h(n))$
            then $\forall n(m^+ \neq n \Rightarrow h(m^+) \neq h(n))$}
            \begin{proof}
                \step{i}{\pflet{$m \in \omega$}}
                \step{ii}{\assume{$\forall n(m \neq n \Rightarrow h(m) \neq h(n))$}}
                \step{iii}{\pflet{$n \in \omega$}}
                \step{iv}{\assume{$m^+ \neq n$} \prove{$h(m^+) \neq h(n)$}}
                \step{v}{\case{$n = 0$}}
                \begin{proof}
                    \pf\ $h(m^+) = S(h(m)) \neq e = h(n)$
                \end{proof}
                \step{vi}{\case{$n = p^+$}}
                \begin{proof}
                    \step{one}{$m \neq p$}
                    \step{two}{$h(m) \neq h(p)$}
                    \step{three}{$S(h(m)) \neq S(h(p))$}
                    \step{four}{$h(m^+) \neq h(p^+)$}
                \end{proof}
            \end{proof}
        \end{proof}
        \step{3}{For all $x \in N$, there exists $n \in \omega$ such that $h(n) = x$}
        \begin{proof}
            \pf\ An easy induction on $x$.
        \end{proof}
        \qed
    \end{proof}

    \section{Finite Sets}

    \begin{definition}[Finite]
        A set is \emph{finite} iff it is equinumerous with a natural number. Otherwise it is infinite.
    \end{definition}

    \begin{theorem}
        No natural number is equinumerous with a proper subset of itself.
    \end{theorem}

    \begin{proof}
        \pf
        \step{1}{Any injective function $f : 0 \rightarrow 0$ has range $0$.}
        \begin{proof}
            \pf\ Since the only such function is $\emptyset$.
        \end{proof}
        \step{2}{For any natural number $n$, if every injective function $f : n \rightarrow n$
        has range $n$, then every injective function $f : n^+ \rightarrow n^+$
        has range $n^+$.}
        \begin{proof}
            \step{a}{\pflet{$n \in \omega$}}
            \step{b}{\assume{Every injective function $f : n \rightarrow n$ has range $n$.}}
            \step{c}{\pflet{$f : n^+ \rightarrow n^+$ be injective.}}
            \step{d}{Define $g : n \rightarrow n$ by
            \[ g(k) = \begin{cases}
                f(k) & \text{if } f(k) \in n \\
                f(n) & \text{if } f(k) = n
            \end{cases} \]}
            \begin{proof}
                \pf\ If $k \in n$ and $f(k) = n$ then $f(n) \in n$ since $f$ is injective.
            \end{proof}
            \step{e}{$g$ is injective.}
            \begin{proof}
                \step{i}{\pflet{$i,j \in n$}}
                \step{ii}{\assume{$g(i) = g(j)$}}
                \step{iii}{\case{$f(i) \in n$, $f(j) \in n$}}
                \begin{proof}
                    \pf\ Then $f(i) = f(j)$ so $i = j$
                \end{proof}
                \step{iv}{\case{$f(i) \in n$, $f(j) \notin n$}}
                \begin{proof}
                    \pf\ Then $f(i) = f(n)$ which is impossible as $f$ is injective.
                \end{proof}
                \step{v}{\case{$f(i) \notin n$, $f(j) \in n$}}
                \begin{proof}
                    \pf\ Then $f(n) = f(j)$ which is impossible as $f$ is injective.
                \end{proof}
                \step{vi}{\case{$f(i) \notin n$, $f(j) \notin n$}}
                \begin{proof}
                    \pf\ Then $f(i) = f(j) = n$ so $i = j$.
                \end{proof}
            \end{proof}
            \step{f}{$\ran g = n$}
            \begin{proof}
                \pf\ By \stepref{b}.
            \end{proof}
            \step{g}{$\ran f = n^+$}
            \begin{proof}
                \step{i}{$\forall k \in n. k \in \ran f$}
                \begin{proof}
                    \pf\ Since $\ran g \subseteq \ran f$.
                \end{proof}
                \step{ii}{$n \in \ran f$}
                \begin{proof}
                    \step{one}{\case{$f(n) \in n$}}
                    \begin{proof}
                        \step{ONE}{\pick\ $k$ such that $g(k) = f(n)$}
                        \step{TWO}{$f(k) = n$}
                    \end{proof}
                    \step{two}{\case{$f(n) = n$}}
                    \begin{proof}
                        \pf\ Then $n \in \ran f$.
                    \end{proof}
                \end{proof}
            \end{proof}
        \end{proof}
        \qed
    \end{proof}

    \begin{corollary}
        No finite set is equinumerous with a proper subset of itself.
    \end{corollary}

    \begin{corollary}
        The set $\omega$ is infinite.
    \end{corollary}

    \begin{proof}
        \pf\ Since the function that maps $n$ to $n+1$ is a bijection between $\omega$ and the proper
        subset $\omega - \{ 0 \}$. \qed
    \end{proof}

    \begin{corollary}
        Every finite set is equinumerous with a unique natural number.
    \end{corollary}

    \begin{lemma}
        Let $n$ be a natural number and $C \subseteq n$. Then there exists $m \underline{\in} n$ such that $C \approx m$.
    \end{lemma}

    \begin{proof}
        \pf
        \step{1}{For all $C \subseteq 0$, there exists $m \underline{\in} 0$ such that $C \approx m$.}
        \begin{proof}
            \pf\ In this case $C = \emptyset$ and so $C \approx 0$.
        \end{proof}
        \step{2}{Let $n \in \omega$. Assume that, for all $C \subseteq n$, there exists $m \underline{\in} n$
        such that $C \approx m$. Let $C \subseteq n^+$. Then there exists $m \underline{\in} n^+$
        such that $C \approx m$.}
        \begin{proof}
            \step{a}{\pflet{$n \in \omega$}}
            \step{b}{\assume{For all $C \subseteq n$, there exists $m \underline{\in} n$ such that
            $C \approx m$.}}
            \step{c}{\pflet{$C \subseteq n^+$}}
            \step{d}{\case{$n \in C$}}
            \begin{proof}
                \step{i}{\pick\ $m \underline{\in} n$ such that $C - \{ n \} \approx m$}
                \step{ii}{$C \approx m^+$}
            \end{proof}
            \step{e}{\case{$n \notin C$}}
            \begin{proof}
                \pf\ Then $C \subseteq n$ so $C \approx m$ for some $m \underline{\in} n$.
            \end{proof}
        \end{proof}
        \qed
    \end{proof}

    \begin{corollary}
        Any subset of a finite set is finite.
    \end{corollary}

    \section{Cardinal Numbers}

    \begin{definition}[Cardinality]
        TODO
    \end{definition}

    \begin{theorem}
        For any sets $A$ and $B$, $|A| = |B|$ if and only if $A \approx B$.
    \end{theorem}

    \begin{proof}
        \pf\ TODO \qed
    \end{proof}

    \begin{theorem}
        For any finite set $A$, $|A|$ is the natural number such that $A \approx |A|$.
    \end{theorem}

    \begin{proof}
        \pf\ TODO \qed
    \end{proof}

    \begin{definition}
        We write $\aleph_0$ for $|\omega$.
    \end{definition}

    \section{Cardinal Arithmetic}

    \begin{definition}[Addition]
        Let $\kappa$ and $\lambda$ be any cardinal numbers. Then $\kappa + \lambda = |K \cup L|$,
        where $K$ and $L$ are any disjoint sets of cardinality $\kappa$ and $\lambda$ respectively.

        To show this is well-defined, we must prove that, if $K_1 \approx K_2$, $L_1 \approx L_2$,
        and $K_1 \cap L_1 = K_2 \cap L_2 = \emptyset$, then $K_1 \cup L_1 \approx K_2 \cup L_2$.
    \end{definition}

    \begin{proof}
        \pf\ Easy.
    \end{proof}

    \begin{lemma}
        For any cardinal number $\kappa$ we have $\kappa + 0 = \kappa$.
    \end{lemma}

    \begin{proof}
        \pf\ Since for any set $K$ we have $K \cup \emptyset = K$.
    \end{proof}

    \begin{lemma}
        For any natural number $n$ we have $n + \aleph_0 = \aleph_0$.
    \end{lemma}

    \begin{proof}
        \pf\ Easy. \qed
    \end{proof}

    \begin{lemma}
        \label{lemma:aleph0_plus_aleph0}
        \[ \aleph_0 + \aleph_0 = \aleph_0 \]
    \end{lemma}

    \begin{proof}
        \pf\ Define $f : (\omega \times \{ 0 \}) \cup (\omega \times \{1\}) \rightarrow \omega$ by
        $f(n,0) = 2n$ and $f(n,1) = 2n+1$. Then $f$ is a bijection. \qed
    \end{proof}

    \begin{theorem}
        \[ \kappa + \lambda = \lambda + \kappa \]
    \end{theorem}

    \begin{proof}
        \pf\ Easy. \qed
    \end{proof}

    \begin{theorem}
        \[ \kappa + (\lambda + \mu) = (\kappa + \lambda) + \mu \]
    \end{theorem}

    \begin{proof}
        \pf\ Easy. \qed
    \end{proof}

    \begin{definition}[Multiplication]
        Let $\kappa$ and $\lambda$ be any cardinal numbers. Then $\kappa \lambda = |K \times L|$,
        where $K$ and $L$ are any sets of cardinality $\kappa$ and $\lambda$ respectively.
    \end{definition}

    It is easy to prove this well-defined.

    \begin{lemma}
        For any cardinal number $\kappa$ we have $\kappa 0 = 0$.
    \end{lemma}

    \begin{proof}
        \pf\ For any set $K$ we have $K \times \emptyset = \emptyset$. \qed
    \end{proof}

    \begin{lemma}
        For any natural number $n$ we have $n \aleph_0 = \aleph_0$.
    \end{lemma}

    \begin{proof}
        \pf\ Induction on $n$ using Lemma \ref{lemma:aleph0_plus_aleph0}. \qed
    \end{proof}

    \begin{lemma}
        \[ \aleph_0 \aleph_0 = \aleph_0 \]
    \end{lemma}

    \begin{proof}
        \pf\ Define $f : \omega \times \omega \rightarrow \omega$ by $f(m,n) = 2^m(2n+1)-1$. Then $f$
        is a bijection. \qed
    \end{proof}

    \begin{lemma}
        \[ \kappa 1 = \kappa \]
    \end{lemma}

    \begin{proof}
        \pf\ Easy. \qed
    \end{proof}

    \begin{theorem}
        \[ \kappa \lambda = \lambda \kappa \]
    \end{theorem}

    \begin{proof}
        \pf\ Easy. \qed
    \end{proof}

    \begin{theorem}
        \[ \kappa (\lambda \mu) = (\kappa \lambda) \mu \]
    \end{theorem}

    \begin{proof}
        \pf\ Easy. \qed
    \end{proof}

    \begin{theorem}
        \[ \kappa (\lambda + \mu) = \kappa \lambda + \kappa \mu \]
    \end{theorem}

    \begin{proof}
        \pf\ Easy. \qed
    \end{proof}

    \begin{definition}[Exponentiation]
        Let $\kappa$ and $\lambda$ be any cardinal numbers. Then $\kappa^\lambda = |K^L|$,
        where $K$ and $L$ are any sets of cardinality $\kappa$ and $\lambda$ respectively.
    \end{definition}

    It is easy to prove this well-defined.

    \begin{theorem}
        For any cardinal $\kappa$, $\kappa^0 = 1$.
    \end{theorem}

    \begin{proof}
        \pf\ For any set $K$, there is only one function $\emptyset \rightarrow K$, namely $\emptyset$. \qed
    \end{proof}

    \begin{theorem}
        For any non-zero cardinal $\kappa$, we have $0^\kappa = 0$.
    \end{theorem}

    \begin{proof}
        \pf\ For any nonempty set $K$, there is no function $K \rightarrow \emptyset$. \qed
    \end{proof}

    \begin{theorem}
        For any set $A$, $|\mathcal{P} A| = 2^{|A|}$.
    \end{theorem}

    \begin{proof}
        \pf\ Define the bijection $f : \mathcal{P} A \rightarrow 2^A$ by $f(S)(a) = 1$ if $a \in S$,
        0 if $a \notin S$. \qed
    \end{proof}

    \begin{corollary}
        For any cardinal $\kappa$, we have $\kappa \neq 2^\kappa$.
    \end{corollary}

    \begin{theorem}
        \[ \kappa^{\lambda + \mu} = \kappa^\lambda \kappa^\mu \]
    \end{theorem}

    \begin{proof}
        \pf\ Easy. \qed
    \end{proof}

    \begin{theorem}
        \[ (\kappa \lambda)^\mu = \kappa^\mu \lambda^\mu \]
    \end{theorem}

    \begin{proof}
        \pf\ Easy. \qed
    \end{proof}

    \begin{theorem}
        \[ (\kappa^\lambda)^\mu = \kappa^{\lambda \mu} \]
    \end{theorem}

    \begin{proof}
        \pf\ Easy. \qed
    \end{proof}

    \section{Arithmetic}

    \begin{lemma}
        For any natural numbers $m$ and $n$, we have $m + n^+ = (m+n)^+$.
    \end{lemma}

    \begin{proof}
        \pf\ Easy. \qed
    \end{proof}

    \begin{corollary}
        The union of two finite sets is finite.
    \end{corollary}

    \begin{lemma}
        For any natural numbers $m$ and $n$ we have $m n^+ = mn + m$.
    \end{lemma}

    \begin{proof}
        \pf\ Easy. \qed
    \end{proof}

    \begin{corollary}
        The Cartesian product of two finite sets is finite.
    \end{corollary}

    \begin{lemma}
        For any natural numbers $m$ and $n$ we have $m^{n^+} = m^n m$.
    \end{lemma}

    \begin{proof}
        \pf\ Easy. \qed
    \end{proof}

    \begin{corollary}
        If $A$ and $B$ are finite sets then $A^B$ is finite.
    \end{corollary}

    \section{Ordering on the Natural Numbers}

    \begin{lemma}
        \label{lemma:natural_number_order_successor}
        For any natural numbers $m$ and $n$, $m \in n$ if and only if $m^+ \in n^+$.
    \end{lemma}

    \begin{proof}
        \pf
        \step{1}{$\forall m,n \in \omega (m \in n \Rightarrow m^+ \in n^+)$}
        \begin{proof}
            \step{a}{$\forall m \in \omega (m \in 0 \Rightarrow m^+ \in 0^+)$}
            \begin{proof}
                \pf\ Vacuous.
            \end{proof}
            \step{b}{For all $n \in \omega$, if $\forall m \in n. m^+ \in n^+$ then
            $\forall m \in n^+. m^+ \in n^{++}$}
            \begin{proof}
                \step{i}{\pflet{$n \in \omega$}}
                \step{ii}{\assume{$\forall m \in n. m^+ \in n^+$}}
                \step{iii}{\pflet{$m \in n^+$}}
                \step{iv}{\case{$m \in n$}}
                \begin{proof}
                    \step{one}{$m^+ \in n^+$}
                    \begin{proof}
                        \pf\ By \stepref{ii}
                    \end{proof}
                    \step{two}{$m^+ \in n^{++}$}
                \end{proof}
                \step{v}{\case{$m = n$}}
                \begin{proof}
                    \pf\ $m^+ = n^+ \in n^{++}$
                \end{proof}
            \end{proof}
        \end{proof}
        \step{2}{$\forall m,n \in \omega (m^+ \in n^+ \Rightarrow m \in n)$}
        \begin{proof}
            \step{a}{\pflet{$m, n \in \omega$}}
            \step{b}{\assume{$m^+ \in n^+$}}
            \step{c}{$m \in m^+$}
            \step{d}{$m^+ \in n$ or $m^+ = n$}
            \step{e}{$m \in n$}
            \begin{proof}
                \pf\ If $m^+ \in n$ this follows because $n$ is transitive (Theorem \ref{theorem:natural_number_transitive}).
            \end{proof}
        \end{proof}
        \qed
    \end{proof}

    \begin{lemma}
        \label{lemma:natural_number_irreflexive}
        For any natural number $n$ we have $n \notin n$.
    \end{lemma}

    \begin{proof}
        \pf
        \step{1}{$0 \notin 0$}
        \step{2}{For all $n \in \omega$, if $n \notin n$ then $n^+ \notin n^+$}
        \begin{proof}
            \step{a}{\pflet{$n \in \omega$}}
            \step{c}{\assume{$n^+ \in n^+$} \prove{$n \in n$}}
            \step{d}{$n^+ \in n$ or $n^+ = n$}
            \step{e}{$n \in n^+$}
            \step{f}{$n \in n$}
            \begin{proof}
                \pf\ If $n^+ \in n$ this follows because $n$ is transitive (Theorem \ref{theorem:natural_number_transitive}).
            \end{proof}
        \end{proof}
        \qed
    \end{proof}

    \begin{theorem}[Trichotomy Law for $\omega$]
        For any natural numbers $m$ and $n$, exactly one of
        \[ m \in n, m = n, n \in m \]
        holds.
    \end{theorem}

    \begin{proof}
        \pf
        \step{1}{For any $m, n \in \omega$, at most one of $m \in n$, $m = n$, $n \in m$ holds.}
        \begin{proof}
            \pf\ If $m \in n$ and $m = n$ then $m \in m$ contradicting Lemma \ref{lemma:natural_number_irreflexive}.

            If $m \in n$ and $n \in m$ then $m \in m$ by Theorem \ref{theorem:natural_number_transitive},
            contradicting Lemma \ref{lemma:natural_number_irreflexive}.
        \end{proof}
        \step{2}{For any $m, n \in \omega$, at least one of $m \in n$, $m = n$, $n \in m$ holds.}
        \begin{proof}
            \step{a}{For all $n \in \omega$, either $0 \in n$ or $0 = n$}
            \begin{proof}
                \step{i}{$0 = 0$}
                \step{ii}{For all $n \in \omega$, if $0 \in n$ or $0 = n$ then $0 \in n^+$}
            \end{proof}
            \step{b}{For all $m \in \omega$, if $\forall n \in \omega (m \in n \vee m = n \vee n \in m)$
            then $\forall n \in \omega (m^+ \in n \vee m^+ = n \vee n \in m^+)$}
            \begin{proof}
                \step{i}{\pflet{$m \in \omega$}}
                \step{ii}{\assume{$\forall n \in \omega (m\in n \vee m = n \vee n \in m)$}}
                \step{iii}{\pflet{$n \in \omega$}}
                \step{iv}{\case{$m \in n$}}
                \begin{proof}
                    \pf\ Then $m \in n^+$
                \end{proof}
                \step{v}{\case{$m = n$}}
                \begin{proof}
                    \pf\ Then $m \in n^+$
                \end{proof}
                \step{vi}{\case{$n \in m$}}
                \begin{proof}
                    \pf\ Then $n^+ \in m^+$ by Lemma \ref{lemma:natural_number_order_successor}
                    so $n^+ \in m$ or $n^+ = m$.
                \end{proof}
            \end{proof}
        \end{proof}
        \qed
    \end{proof}

    \begin{corollary}
        The relation $\in$ is a linear ordering on $\omega$.    
    \end{corollary}

    \begin{corollary}
        For any natural numbers $m$ and $n$,
        \[ m \in n \Leftrightarrow m \subset n \enspace . \]
    \end{corollary}

    \begin{proof}
        \pf
        \step{1}{\pflet{$m, n \in \omega$}}
        \step{2}{If $m \in n$ then $m \subset n$.}
        \begin{proof}
            \step{a}{\assume{$m \in n$}}
            \step{b}{$m \subseteq n$}
            \begin{proof}
                \pf\ Theorem \ref{theorem:natural_number_transitive}.
            \end{proof}
            \step{c}{$m \neq n$}
            \begin{proof}
                \pf\ Lemma \ref{lemma:natural_number_irreflexive}.
            \end{proof}
        \end{proof}
        \step{3}{If $m \subset n$ then $m \in n$.}
        \begin{proof}
            \pf\ We have $m \neq n$ and $n \notin m$ by \stepref{2}, hence $m \in n$ by trichotomy.
        \end{proof}
        \qed
    \end{proof}

    \begin{theorem}
        \label{theorem:addition_strictly_monotone}
        For any natural number $p$, the function that maps $n$ to $n + p$ is strictly monotone.
        For any natural numbers $m$, $n$ and $p$, we have $m \in n$ if and only if $m + p \in n + p$.
    \end{theorem}

    \begin{proof}
        \pf\ We prove that $m \in n \Rightarrow m + p \in n + p$. This is an easy induction on $p$ using
        Lemma \ref{lemma:natural_number_order_successor}. \qed
    \end{proof}

    \begin{theorem}
        For any non-zero natural number $p$, the function that maps $n$ to $np$ is strictly monotone.
    \end{theorem}

    \begin{proof}
        \pf\ Easy induction on $p$ using Theorem \ref{theorem:addition_strictly_monotone}. \qed
    \end{proof}

    \begin{theorem}[Strong Induction]
        Let $A$ be a subset of $\omega$ and suppose that, for all $n \in \omega$, we have
        \[ (\forall m < n. m \in A) \Rightarrow n \in A \enspace . \]
        Then $A = \omega$.
    \end{theorem}

    \begin{proof}
        \pf\ Prove $\forall n \in \omega. \forall m < n. m \in A$ by induction on $n$. \qed
    \end{proof}
    
    \begin{theorem}[Well-Ordering of $\omega$]
        Every nonempty subset of $\omega$ has a least element.
    \end{theorem}
    
    \begin{proof}
        \pf\ If $A$ is a subset of $\omega$ with no least element, we prove $\forall n \in \omega. n \notin A$
        by strong induction on $n$. \qed
    \end{proof}

    \begin{corollary}
        There is no function $f : \omega \rightarrow \omega$ such that $f(n+1) < f(n)$ for every $n$.
    \end{corollary}

    \begin{lemma}
        \label{lemma:subtraction}
        For any natural numbers $m$ and $n$, we have $m \in n$ if and only if there exists a natural number
        $p$ such that $n = m + p^+$.
    \end{lemma}
    
    \begin{proof}
        \pf
        \step{1}{For all $m$, $p$, we have $m \in m + p^+$}
        \begin{proof}
            \pf\ $m = m + 0 \in m + p^+$
        \end{proof}
        \step{2}{For all $m$, $n$, if $m \in n$ then there exists $p$ such that $n = m + p^+$}
        \begin{proof}
            \step{a}{For all $m$, if $m \in 0$ then there exists $p$ such that $0 = m + p^+$}
            \begin{proof}
                \pf\ Vacuous.
            \end{proof}
            \step{b}{For all $n \in \omega$, if $\forall m \in n. \exists p \in \omega. n = m + p^+$
            then $\forall m \in n^+. \exists p \in \omega. n^+ = m + p^+$}
            \begin{proof}
                \step{i}{\pflet{$n \in \omega$}}
                \step{ii}{\assume{$\forall m \in n. \exists p \in \omega. n = m + p^+$}}
                \step{iii}{\pflet{$m \in n^+$}}
                \step{iv}{\case{$m \in n$}}
                \begin{proof}
                    \step{one}{\pick\ $p$ such that $n = m + p^+$}
                    \step{two}{$n^+ = m + p^{++}$}
                \end{proof}
                \step{v}{\case{$m = n$}}
                \begin{proof}
                    \pf\ $n^+ = m + 0^+$
                \end{proof}
            \end{proof}
        \end{proof}
        \qed
    \end{proof}

    \begin{lemma}
        \label{lemma:pre_integer_ordering}
        For natural numbers $m$, $n$, $p$ and $q$, if $m \in n$ and $p \in q$
        then $mp + nq \in mq + np$.
    \end{lemma}

    \begin{proof}
        \step{a}{\pick\ natural numbers $a$ and $b$ such that $n = m + a^+$ and $q = p + b^+$}
        \begin{proof}
            \pf\ Lemma \ref{lemma:subtraction}.
        \end{proof}
        \step{b}{$mp + nq = mq + np + (a^+ + b)^+$}
        \step{c}{$mp + nq \in mq + np$}
        \begin{proof}
            \pf\ Lemma \ref{lemma:subtraction}.
        \end{proof}
    \end{proof}

    \section{The Integers}

    \begin{theorem}
        The relation $\sim$ is an equivalence relation on $\omega \times \omega$, where $(m,n) \sim (p,q)$ iff
        $m + q = n + p$.
    \end{theorem}

    \begin{proof}
        \pf
        \step{1}{The relation $\sim$ is reflexive on $\omega^2$}
        \begin{proof}
            \pf\ For any $m$, $n$, we have $m + n = m + n$ and so $(m,n) \sim (m,n)$.
        \end{proof}
        \step{2}{The relation $\sim$ is symmetric.}
        \begin{proof}
            \pf\ If $m + q = n + p$ then $p + n = q + m$.
        \end{proof}
        \step{3}{The relation $\sim$ is transitive.}
        \begin{proof}
            \step{a}{\assume{$(m,n) \sim (p,q) \sim (r,s)$}}
            \step{b}{$m + q = n + p$}
            \step{c}{$p + s = q + r$}
            \step{d}{$m + p + q + s = n + p + q + r$}
            \step{d}{$m + s = n + r$}
            \begin{proof}
                \pf\ By cancellation of addition in $\omega$.
            \end{proof}
        \end{proof}
        \qed
    \end{proof}

    \begin{definition}
        The set $\mathbb{Z}$ of \emph{integers} is the quotient set $(\omega \times \omega) / \sim$.
    \end{definition}

    \begin{lemma}
        If $(m,n) \sim (m',n')$ and $(p,q) \sim (p',q')$ then $(m+p,n+q) \sim (m'+p',n'+q')$.
    \end{lemma}

    \begin{proof}
        \pf\ Assume $m + n' = m' + n$ and $p + q' = p' + q$. Then $m + p + n' + q' = m' + p' + n + q$. \qed
    \end{proof}

    \begin{definition}[Addition]
        Addition $+$ on $\mathbb{Z}$ is the binary operation such that
        \[ [(m,n)] + [(p,q)] = [(m+p,n+q)] \]
    \end{definition}

    \begin{theorem}
        Addition on $\mathbb{Z}$ is commutative.
    \end{theorem}

    \begin{proof}
        \pf\ From the definition. \qed
    \end{proof}

    \begin{theorem}
        Addition on $\mathbb{Z}$ is associtative.
    \end{theorem}

    \begin{proof}
        \pf\ Easy. \qed
    \end{proof}

    \begin{definition}[Zero]
        The zero in the integers is $0 = [(0,0)]$.
    \end{definition}

    \begin{theorem}
        For any integer $a$ we have $a + 0 = 0$.
    \end{theorem}

    \begin{proof}
        \pf\ Easy. \qed
    \end{proof}

    \begin{theorem}
        For any integer $a$, there exists an integer $b$ such that $a + b = 0$.
    \end{theorem}

    \begin{proof}
        \pf\ If $a = [(m,n)]$ take $b = [(n,m)]$. \qed
    \end{proof}

    \begin{lemma}
        If $(m,n) \sim (m',n')$ and $(p,q) \sim (p',q')$ then $(mp+nq,mq+np) \sim (m'p'+n'q',
        m'q'+n' p')$.
    \end{lemma}

    \begin{proof}
        \pf
        \step{1}{\assume{$m+n'=m'+n$ and $p+q'=p'+q$}}
        \step{2}{$mp+n'p=m'p+np$}
        \step{3}{$m'q+nq=mq+n'q$}
        \step{4}{$mp+mq'=mp'+mq$}
        \step{5}{$n'p'+n'q=n'p+n'q'$}
        \step{6}{$mp+n'p+m'q+nq+mp+mq'+n'p'+n'q=m'p+np+mq+n'q+mp'+mq+n'p+n'q'$}
        \step{7}{$mp+nq+m'q'+n'p'=mq+np+m'p'+n'q'$}
        \qed
    \end{proof}

    \begin{definition}[Multiplication]
        \emph{Multiplication} $\cdot$ is the binary operation on $\mathbb{Z}$ such that
        \[ [(m,n)][(p,q)] = [(mp+nq,mq+np)] \]
    \end{definition}

    \begin{theorem}
        Multiplication is commutative.
    \end{theorem}

    \begin{proof}
        \pf\ Easy. \qed
    \end{proof}

    \begin{theorem}
        Multiplication is associative.
    \end{theorem}

    \begin{proof}
        \pf\ Easy. \qed
    \end{proof}

    \begin{theorem}
        Multiplication is distributive over addition.
    \end{theorem}

    \begin{proof}
        \pf\ Easy. \qed
    \end{proof}

    \begin{definition}
        The integer one is $1 = [(1,0)]$.
    \end{definition}

    \begin{theorem}
        For any integer $a$ we have $a1= a$.
    \end{theorem}

    \begin{proof}
        \pf\ Easy. \qed
    \end{proof}

    \begin{theorem}
        $0 \neq 1$
    \end{theorem}

    \begin{proof}
        \pf\ Easy. \qed
    \end{proof}

    \begin{lemma}
        If $(m,n) \sim (m',n')$ and $(p,q) \sim (p',q')$ then $m + q \in p + n$ iff
        $m' + q' \in p' + n'$.
    \end{lemma}

    \begin{proof}
        \pf
        \begin{align*}
            m + q \in p + n & \Leftrightarrow m + q + n' + q' \in p + n + n' + q' \\
            & \Leftrightarrow m' + n + q + q' \in p' + n + n' + q \\
            & \Leftrightarrow m' + q' \in p' + n' & \qed
        \end{align*}
    \end{proof}

    \begin{definition}[Ordering]
        The ordering $<$ on $\mathbb{Z}$ is defined by: $[(m,n)] < [(p,q)]$ iff $m + q \in n + p$.
    \end{definition}

    \begin{theorem}
        The relation $<$ is a linear ordering on $\mathbb{Z}$.
    \end{theorem}

    \begin{proof}
        \pf
        \step{1}{$<$ is transitive.}
        \begin{proof}
            \step{a}{\assume{$[(m,n)] < [(p,q)]$ and $[(p,q)] < [(r,s)]$}}
            \step{b}{$m+q \in n+p$ and $p+s \in q+r$}
            \step{c}{$m+q+s \in n+p+s$}
            \step{d}{$n+p+s \in n+q+r$}
            \step{e}{$m+q+s \in n+q+r$}
            \step{f}{$m+s \in n+r$}
        \end{proof}
        \step{2}{$<$ satisfies trichotomy.}
        \begin{proof}
            \pf\ From trichotomy on $\omega$.
        \end{proof}
        \qed
    \end{proof}

    \begin{theorem}
        For any integers $a$, $b$ and $c$, we have $a < b$ iff $a + c < b + c$.
    \end{theorem}

    \begin{proof}
        \pf\ An easy consequence of the corresponding property in $\omega$.
    \end{proof}

    \begin{corollary}
        If $a + c = b + c$ then $a = b$.
    \end{corollary}

    \begin{theorem}
        If $0 < c$, then the function that maps an integer $a$ to $ac$ is strictly monotone.
    \end{theorem}

    \begin{proof}
        \pf
        \step{1}{\pflet{$a$, $b$ and $c$ be integers.}}
        \step{2}{\assume{$0 < c$ and $a < b$}}
        \step{3}{\pflet{$a = [(m,n)]$}}
        \step{4}{\pflet{$b = [(p,q)]$}}
        \step{5}{\pflet{$c = [(r,s)]$}}
        \step{6}{$s \in r$}
        \step{7}{$m + q \in p + n$}
        \step{10}{$(m+q)r + (p+n)s \in (m+q)s + (p+n)r$}
        \begin{proof}
            \pf\ Lemma \ref{lemma:pre_integer_ordering}.
        \end{proof}
        \step{11}{$ac < bc$}
        \qed
    \end{proof}

    %TODO Extract lemmas to ring theory
    \begin{lemma}
        For integers $a$ and $b$,
        $a(-b) = -(ab)$
    \end{lemma}

    \begin{proof}
        \pf\ This follows from the fact that $ab + a(-b) = a(b + (-b)) = a0 = 0$. \qed
    \end{proof}

    \begin{theorem}
        For integers $a$, $b$ and $c$, if $a < b$ and $c < 0$ then $ac > bc$.
    \end{theorem}

    \begin{proof}
        \pf\ We have $0 < -c$ so $a(-c) < b(-c)$ hence $-(ac) < -(bc)$ so $bc < ac$. \qed
    \end{proof}

    \begin{theorem}
        \label{theorem:integers_no_zero_divisors}
        For any integers $a$ and $b$, if $ab = 0$ then $a = 0$ or $b = 0$.
    \end{theorem}

    \begin{proof}
        \pf\ We prove if $a \neq 0$ and $b \neq 0$ then $ab \neq 0$.
        
        If $a > 0$ and $b > 0$ then $ab > 0$. Similarly for the other four cases. \qed
    \end{proof}

    \begin{theorem}
        \label{theorem:integers_cancel}
        If $ac = bc$ and $c \neq 0$ then $a = b$.
    \end{theorem}

    \begin{proof}
        \pf\ We have $(a-b)c = 0$ so $a-b = 0$ hence $a = b$. \qed
    \end{proof}

    \begin{definition}[Positive]
        An integer $a$ is \emph{positive} iff $0 < a$.
    \end{definition}

    \begin{theorem}
        Define $E : \omega \rightarrow \mathbb{Z}$ by $E(n) = [(n,0)]$. Then $E$ maps $\omega$ one-to-one
        into $\mathbb{Z}$, and:
        \begin{enumerate}
            \item $E(m+n) = E(m) + E(n)$
            \item $E(mn) = E(m) E(n)$
            \item $m \in n$ if and only if $E(m) < E(n)$.
        \end{enumerate}
    \end{theorem}

    \begin{proof}
        \pf\ Routine calculations. \qed
    \end{proof}

    \section{Equinumerosity}

    \begin{definition}[Equinumerous]
        Two sets $A$ and $B$ are \emph{equinumerous}, $A \approx B$, iff there exists a bijection between
        them.
    \end{definition}

    \begin{theorem}
        Equinumerosity is an equivalence relation on the class of sets.
    \end{theorem}

    \begin{proof}
        \pf\ Easy. \qed
    \end{proof}

    \begin{theorem}[Cantor 1873]
        No set is equinumerous with its power set.
    \end{theorem}

    \begin{proof}
        \pf
        \step{1}{\pflet{$g : A \rightarrow \mathcal{P} A$} \prove{$g$ is not surjective.}}
        \step{2}{\pflet{$B = \{x \in A : x \notin g(x) \}$}}
        \step{3}{$\forall x \in A. g(x) \neq B$}
        \begin{proof}
            \pf\ Because $x \in B$ iff $x \notin g(x)$.
        \end{proof}
        \qed
    \end{proof}

    \section{Ordering Cardinal Numbers}

    \begin{definition}[Dominated]
        A set $A$ is \emph{dominated} by a set $B$, $A \preccurlyeq B$, iff there exists an injection
        $f : A \rightarrow B$.
    \end{definition}

    \begin{lemma}
        Domination is a preorder on the class of sets.
    \end{lemma}

    \begin{proof}
        \pf\ Easy. \qed
    \end{proof}

    \begin{lemma}
        If $A \subseteq B$ then $A \preccurlyeq B$.
    \end{lemma}

    \begin{proof}
        \pf\ The inclusion from $A$ to $B$ is an injection. \qed
    \end{proof}

    \begin{lemma}
        If $A \preccurlyeq B$, $A \approx A'$ and $B \approx B'$ then $A' \preccurlyeq B'$.
    \end{lemma}
    
    \begin{proof}
        \pf\ Easy. \qed
    \end{proof}

    \begin{definition}
        Given cardinal numbers $\kappa$ and $\lambda$, we write $\kappa \leq \lambda$ iff
        $K \preccurlyeq L$, where $K$ is any set of cardinality $\kappa$ and $L$ is any set of
        cardinality $\lambda$.

        We write $\kappa < \lambda$ iff $\kappa \leq \lambda$ and $\kappa \neq \lambda$.
    \end{definition}

    \begin{theorem}[Schr\"{o}der-Bernstein]
        If $A \preccurlyeq B$ and $B \preccurlyeq A$ then $A \approx B$.
    \end{theorem}

    \begin{proof}
        \pf
        \step{1}{\pflet{$f : A \rightarrow B$ and $g : B \rightarrow A$ be one-to-one.}}
        \step{2}{Define the sequence of sets $C_n \subseteq A$ by:
        \begin{align*}
            C_0 & = A - \ran g \\
            C_{n+1} & = g(f(C_n))
        \end{align*}}
        \step{3}{Define $h : A \rightarrow B$ by
        \[ h(x) = \begin{cases}
            f(x) & \text{if } \exists n \in \mathbb{N}. x \in C_n \\
            \inv{g}(x) & \text{otherwise}
        \end{cases} \]}
        \step{4}{$h$ is injective.}
        \begin{proof}
            \step{a}{\pflet{$x,y \in A$}}
            \step{b}{\assume{$h(x) = h(y)$}}
            \step{c}{\case{$x \in C_m$, $y \in C_n$}}
            \begin{proof}
                \pf\ We have $f(x) = f(y)$ so $x = y$
            \end{proof}
            \step{d}{\case{$x \in C_m$, $y \notin \bigcup_n C_n$}}
            \begin{proof}
                \pf\ This case is impossible because we would have $y = g(f(x))$ and so $y \in C_{m+1}$.
            \end{proof}
            \step{e}{\case{$x, y \notin \bigcup_n C_n$}}
            \begin{proof}
                \pf\ We have $\inv{g}(x) = \inv{g}(y)$ so $x = y$.
            \end{proof}
        \end{proof}
        \step{5}{$h$ is surjective.}
        \begin{proof}
            \step{a}{\pflet{$y \in B$}}
            \step{b}{\assume{$y \notin f(C_n)$ for all $n$}}
            \step{c}{$g(y) \notin C_n$ for all $n$}
            \step{d}{$y = h(g(y))$}
        \end{proof}
        \qed
    \end{proof}

    \begin{corollary}
        The relation $\leq$ is a partial order on the class of cardinal numbers.
    \end{corollary}

    \begin{theorem}
        Let $\kappa$, $\lambda$ and $\mu$ be cardinal numbers.
        \begin{enumerate}
            \item $\kappa \leq \lambda \Rightarrow \kappa + \mu \leq \lambda + \mu$
            \item $\kappa \leq \lambda \Rightarrow \kappa \mu \leq \lambda \mu$
            \item $\kappa \leq \lambda \Rightarrow \kappa^\mu \leq \lambda^\mu$
            \item $\kappa \leq \lambda \Rightarrow \mu^\kappa \leq \mu^\lambda$ if $\kappa$ and $\mu$
            are not both zero.
        \end{enumerate}
    \end{theorem}

    \begin{proof}
        \pf\ Parts 1--3 are easy. For part 4:
        
        Let $|K| = \kappa$, $|L| = \lambda$ and $|M| = \mu$ with $K \subseteq L$.

        If $M = \emptyset$ then $\kappa \neq 0$ so $\mu^\kappa = 0 \leq \mu^\lambda$.

        Otherwise, pick $a \in M$. Define $\Phi : M^K \rightarrow M^L$ by:
        \[ \Phi(f)(x) = \begin{cases}
            f(x) & \text{if } x \in K \\
            a & \text{if } x \notin K
        \end{cases} \]
        Then $\Phi$ is an injection. \qed
    \end{proof}

    \begin{theorem}[Zorn's Lemma]
        The Axiom of Choice is equivalent to this statement:

        Let $\mathcal{A}$ be a set such that, for every chain $\mathcal{B} \subseteq \mathcal{A}$,
        we have $\bigcup \mathcal{B} \in \mathcal{A}$. Then $\mathcal{A}$ has a maximal element.
    \end{theorem}

    \begin{proof}
        \pf
        \step{1}{If the Axiom of Choice then Zorn's Lemma.}
        \begin{proof}
            \pf\ TODO
        \end{proof}
        \step{2}{If Zorn's Lemma then the Axiom of Choice.}
        \begin{proof}
            \step{a}{\assume{Zorn's Lemma}}
            \step{b}{\pflet{$R$ be a relation.}}
            \step{c}{\pflet{$\mathcal{A}$ be the set of all functions that are subsets of $R$.}}
            \step{d}{For any chain $\mathcal{B} \subseteq \mathcal{A}$ we have $\bigcup \mathcal{B} \in
            \mathcal{A}$}
            \step{e}{\pick\ $F \in \mathcal{A}$ maximal.}
            \step{f}{$\dom F = \dom R$}
        \end{proof}
        \qed
    \end{proof}

    \begin{theorem}[Cardinal Comparability]
        The Axiom of Choice is equivalent to the statement: for any sets $C$ and $D$,
        either $C \preccurlyeq D$ or $D \preccurlyeq C$.
    \end{theorem}

    \begin{proof}
        \pf
        \step{1}{If Zorn's Lemma then Cardinal Comparability.}
        \begin{proof}
            \step{a}{\assume{Zorn's Lemma}}
            \step{b}{\pflet{$C$ and $D$ be sets.}}
            \step{c}{\pflet{$\mathcal{A}$ be the set of all injective functions $f$ with
            $\dom f \subseteq C$ and $\ran f \subseteq D$}}
            \step{d}{For every chain $\mathcal{B} \subseteq \mathcal{A}$ we have $\bigcup \mathcal{B} \in
            \mathcal{A}$}
            \step{e}{\pflet{$f \in \mathcal{A}$ be maximal}}
            \step{f}{$\dom f = C$ or $\ran f = D$}
            \step{g}{$f$ is an injective function $C \rightarrow D$ or $\inv{f}$ is an injective function
            $D \rightarrow C$}
        \end{proof}
        \step{2}{If Cardinal Comparability then the Axiom of Choice.}
        \begin{proof}
            \pf\ TODO
        \end{proof}
        \qed
    \end{proof}

    \begin{theorem}[Choice]
        For any infinite set $A$, we have $\omega \preccurlyeq A$.
    \end{theorem}

    \begin{proof}
        \pf
        \step{1}{\pflet{$A$ be an infinite set.}}
        \step{2}{\pick\ a choice function $F$ for $A$}
        \step{3}{Define $f : \omega \rightarrow A$ by recursion by:
        $f(n) = F(A - \{ f(0), f(1), \ldots, f(n-1) \})$}
        \begin{proof}
            \pf\ $A - \{ f(0), f(1), \ldots, f(n-1) \}$ is nonempty because $A$ is infinite.
        \end{proof}
        \step{4}{$f$ is injective.}
        \qed
    \end{proof}

    \begin{corollary}[Choice]
        For any infinite cardinal $\kappa$ we have $\aleph_0 \leq \kappa$.
    \end{corollary}

    \begin{corollary}[Choice]
        A set is infinite iff it is equinumerous to a proper subset of itself.
    \end{corollary}
    
    \begin{proposition}[Choice]
        If there exists a surjection $A \rightarrow B$ then $B \preccurlyeq A$.
    \end{proposition}

    \begin{proof}
        \pf\ Any surjection $A \rightarrow B$ has a right inverse which is an injection $B \rightarrow A$.
    \end{proof}
    \section{Countable Sets}

    \begin{definition}[Countable]
        A set is \emph{countable} iff it is dominated by $\omega$.
    \end{definition}

    \begin{proposition}
        Any subset of a countable set is countable.
    \end{proposition}

    \begin{proof}
        \pf\ Easy. \qed
    \end{proof}

    \begin{proof}
        The union of two countable sets is countable.
    \end{proof}

    \begin{proof}
        \pf\ Because $\aleph_0 + \aleph_0 = \aleph_0$ \qed
    \end{proof}

    \begin{proposition}
        The product of two countable sets is countable.
    \end{proposition}

    \begin{proof}
        \pf\ Because $\aleph_0 \aleph_0 = \aleph_0$. \qed
    \end{proof}

    \begin{proposition}[Choice]
        For any infinite set $A$, the set $\mathcal{P} A$ is uncountable.
    \end{proposition}

    \begin{proof}
        \pf\ If $|A| \geq \aleph_0$ then $|\mathcal{P} A| \geq 2^{\aleph_0}$. \qed
    \end{proof}

    \begin{theorem}[Choice]
        A countable union of countable sets is countable.
    \end{theorem}

    \begin{proof}
        \pf
        \step{1}{\pflet{$\mathcal{A}$ be a countable set of countable sets.}}
        \step{2}{\assume{w.l.o.g. $\mathcal{A} \neq \emptyset$ and $\emptyset \notin \mathcal{A}$}}
        \step{3}{\pick\ a surjection $G : \omega \rightarrow A$}
        \step{4}{\pick\ a function $F$ with domain $\omega$ such that, for all $m$, $F(m)$ is a surjection
        $\omega \rightarrow G(m)$}
        \begin{proof}
            \pf\ By the Axiom of Choice.
        \end{proof}
        \step{5}{Define $f : \omega \times \omega \rightarrow \bigcup A$ by $f(m,n) = F(m)(n)$}
        \step{6}{$f$ is surjective.}
        \step{7}{$A \preccurlyeq \omega \times \omega$}
        \qed
    \end{proof}
    
    \section{Arithmetic of Infinite Cardinals}

    \begin{lemma}[Choice]
        For any infinite cardinal $\kappa$ we have $\kappa \cdot \kappa = \kappa$.
    \end{lemma}

    \begin{proof}
        \pf
        \step{1}{\pflet{$\kappa$ be an infinite cardinal.}}
        \step{2}{\pflet{$B$ be a set of cardinality $\kappa$.}}
        \step{3}{\pflet{$\mathcal{H} = \{ f : f = \emptyset \text{ or for some infinite }
        A \subseteq B, f \text{ is a bijection between $A \times A$ and $A$} \}$}}
        \step{4}{For any chain $\mathcal{C} \subseteq \mathcal{H}$, we have $\bigcup \mathcal{C} \in \mathcal{H}$}
        \begin{proof}
            \step{a}{\pflet{$\mathcal{C} \subseteq \mathcal{H}$ be a chain.}}
            \step{b}{\assume{w.l.o.g. $\mathcal{C}$ has a nonempty element.}}
            \begin{proof}
                \pf\ Otherwise $\bigcup \mathcal{C} = \emptyset \in \mathcal{H}$.
            \end{proof}
            \step{c}{$\bigcup \mathcal{C}$ is an injective function.}
            \step{d}{\pflet{$A = \ran \bigcup \mathcal{C}$}}
            \step{e}{$A$ is infinite.}
            \step{f}{$\bigcup \mathcal{C}$ is a bijection between $A \times A$ and $A$.}
            \begin{proof}
                \step{i}{\pflet{$a_1, a_2 \in A$}}
                \step{ii}{\pick\ $f_1, f_2 \in \mathcal{C}$ such that $a_1 \in \ran f_1$ and $a_2 \in \ran f_2$}
                \step{iii}{\assume{w.l.o.g. $f_1 \subseteq f_2$}}
                \step{iv}{$\langle a_1, a_2 \rangle \in \dom f_2$}
                \step{v}{$\langle a_1, a_2 \rangle \in \dom \bigcup \mathcal{C}$}
            \end{proof}
        \end{proof}
        \step{5}{\pick\ a maximal $f_0 \in \mathcal{H}$}
        \begin{proof}
            \pf\ Zorn's Lemma.
        \end{proof}
        \step{6}{$f_0 \neq \emptyset$}
        \begin{proof}
            \pf\ $B$ has a countable subset $A$, say, and $A \times A \approx A$.
        \end{proof}
        \step{7}{\pick\ $A_0 \subseteq B$ infinite such that $f_0$ is a bijection between $A_0 \times A_0$
        and $A_0$.}
        \step{8}{\pflet{$\lambda = |A_0|$}}
        \step{9}{$\lambda$ is infinite}
        \step{10}{$\lambda = \lambda \cdot \lambda$}
        \step{11}{$\lambda = \kappa$}
        \begin{proof}
            \step{a}{$|B - A_0| < \lambda$}
            \begin{proof}
                \step{i}{\assume{for a contradiction $\lambda \leq |B - A_0|$}}
                \step{ii}{\pick\ $D \subseteq B - A_0$ with $|D| = \lambda$}
                \step{iii}{$(A_0 \cup D) \times (A_0 \cup D) = (A_0 \times A_0) \cup (A_0 \times D) \cup
                (D \times A_0) \cup (D \times D)$}
                \step{iv}{$f_0 : A_0 \times A_0 \approx A_0$}
                \step{v}{$|(A_0 \times D) \cup
                (D \times A_0) \cup (D \times D)| = \lambda$}
                \begin{proof}
                    \pf
                    \begin{align*}
                        |(A_0 \times D) \cup
                (D \times A_0) \cup (D \times D)| & = \lambda \cdot \lambda + \lambda \cdot \lambda + \lambda \cdot \lambda \\
                & = \lambda + \lambda + \lambda & (\text{\stepref{10}})\\
                & = 3 \cdot \lambda \\
                & \leq \lambda \cdot \lambda \\
                & = \lambda & (\text{\stepref{10}})
                    \end{align*}
                \end{proof}
                \step{vi}{\pick\ a bijection $g : (A_0 \times D) \cup
                (D \times A_0) \cup (D \times D) \approx D$}
                \step{vii}{$f_0 \cup g : (A_0 \cup D) \times (A_0 \cup D) \approx A_0 \cup D$}
                \qedstep
                \begin{proof}
                    \pf\ This contradicts the maximality of $f_0$.
                \end{proof}
            \end{proof}
            \step{b}{$\lambda = \kappa$}
            \begin{proof}
                \pf
                \begin{align*}
                    \kappa & = |B| \\
                    & = |A_0| + |B - A_0| \\
                    & \leq \lambda + \lambda \\
                    & = 2 \cdot \lambda \\
                    & \leq \lambda \cdot \lambda \\
                    & = \lambda \\
                    & \leq \kappa
                \end{align*}
            \end{proof}
        \end{proof}
        \qed
    \end{proof}

    \begin{corollary}[Absorption Law of Cardinal Arithmetic (Choice)]
        Let $\kappa$ and $\lambda$ be cardinal numbers, the larger of which is infinite and the smaller of
        which is nonzero. Then
        \[ \kappa + \lambda = \kappa \cdot \lambda = \max(\kappa, \lambda) \enspace . \]
    \end{corollary}

    \begin{proof}
        \pf
        \step{1}{\assume{w.l.o.g. $\kappa \leq \lambda$}}
        \step{2}{$\kappa + \lambda = \lambda$}
        \begin{proof}
            \pf
            \begin{align*}
                \lambda & \leq \kappa + \lambda \\
                & \leq \lambda + \lambda \\
                & = 2 \cdot \lambda \\
                & \leq \lambda \cdot \lambda \\
                & = \lambda
            \end{align*}
        \end{proof}
        \step{3}{$\kappa \cdot \lambda = \lambda$}
        \begin{proof}
            \pf
            \begin{align*}
                \lambda & = 1 \cdot \lambda \\
                & \leq \kappa \cdot \lambda \\
                & \leq \lambda \cdot \lambda \\
                & = \lambda
            \end{align*}
        \end{proof}
        \qed
    \end{proof}
\end{document}
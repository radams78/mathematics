\documentclass{article}

\title{C2 Order Theory}
\author{Robin Adams}

\usepackage{amsmath}
\usepackage{amssymb}
\usepackage{amsthm}
\let\proof\relax
\let\endproof\relax
\let\qed\relax
\usepackage{pf2}
\usepackage[all]{xy}

\newtheorem{axiom}{Axiom}
\newtheorem{axs}[axiom]{Axiom Schema}
\newtheorem{lemma}[axiom]{Lemma}
\newtheorem{proposition}[axiom]{Proposition}
\newtheorem{props}[axiom]{Proposition Schema}
\newtheorem{theorem}[axiom]{Theorem}
\newtheorem{corollary}{Corollary}[axiom]
\theoremstyle{definition}
\newtheorem{definition}[axiom]{Definition}

\begin{document}
    \maketitle

    \section{Linear Orders}

    \begin{definition}[Linear Ordering]
        Let $\mathbf{A}$ be a class. A \emph{linear ordering} or \emph{total ordering} on $\mathbf{A}$
        is a relation $\mathbf{R}$ on $\mathbf{A}$ such that:
        \begin{itemize}
            \item $\mathbf{R}$ is transitive.
            \item $\mathbf{R}$ satisfies \emph{trichotomy} on $\mathbf{A}$; i.e. for any $x, y \in \mathbf{A}$,
            exactly one of
            \[ x\mathbf{R}y, x=y, y\mathbf{R}x \]
            holds.
        \end{itemize}
    \end{definition}

    \begin{theorem}
        Let $\mathbf{R}$ be a linear ordering on $\mathbf{A}$.
        \begin{enumerate}
            \item There is no $x$ such that $x \mathbf{R} x$.
            \item For distinct $x$ and $y$ in $\mathbf{A}$, either $x\mathbf{R}y$ or $y\mathbf{R}x$.
        \end{enumerate}
    \end{theorem}

    \begin{proof}
        \pf\ Immediate from trichotomy. \qed
    \end{proof}
    
\end{document}
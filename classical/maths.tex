\documentclass{article}

\title{C1 Set Theory}
\author{Robin Adams}

\usepackage{amsmath}
\usepackage{amssymb}
\usepackage{amsthm}
\let\proof\relax
\let\endproof\relax
\let\qed\relax
\usepackage{pf2}
\usepackage[all]{xy}

\newtheorem{axiom}{Axiom}
\newtheorem{axs}[axiom]{Axiom Schema}
\newtheorem{lm}[axiom]{Lemma}
\newtheorem{proposition}[axiom]{Proposition}
\newtheorem{props}[axiom]{Proposition Schema}
\newtheorem{thm}[axiom]{Theorem}
\newtheorem{cor}{Corollary}[axiom]
\theoremstyle{definition}
\newtheorem{definition}[axiom]{Definition}

\begin{document}
    \maketitle

    \section{Primitive Notions}

    Let there be \emph{sets}.

    Let there be a binary relation called \emph{membership}, $\in$. When $x \in y$ holds, we say $x$ is a
    \emph{member} or \emph{element} of $y$. We write $x \notin y$ iff $x$ is not a member of $y$.

    \section{The Axioms}

    \begin{axiom}[Extensionality]
        If two sets have exactly the same members, then they are equal.
    \end{axiom}

    %TODO Justify these definitions

    \begin{definition}[Empty Set]
        The \emph{empty set}, $\emptyset$, is the set with no elements.
    \end{definition}

    \begin{definition}
        Given objects $x_1$, \ldots, $x_n$, we write $\{ x_1, \ldots, x_n \}$ for the set whose elements are
        exactly $x_1$, \ldots, $x_n$.
    \end{definition}

    \begin{definition}[Union]
        The \emph{union} of sets $A$ and $B$, $A \cup B$, is the set whose elements are exactly the things
        that are members of $A$ or members of $B$.
    \end{definition}

    \begin{definition}[Intersection]
        The \emph{intersection} of sets $A$ and $B$, $A \cap B$, is the set whose elements are exactly the
        things that are members of both $A$ and $B$.
    \end{definition}

    \begin{definition}[Disjoint]
        Two sets $A$ and $B$ are \emph{disjoint} iff they have no common members.
    \end{definition}

    \begin{definition}[Subset]
        A set $A$ is a \emph{subset} of a set $B$ or is \emph{included} in $B$, $A \subseteq B$, iff every
        member of $A$ is a member of $B$.
    \end{definition}

    \begin{proposition}
        For any set $A$ we have $A \subseteq A$.
    \end{proposition}

    \begin{proof}
        \pf\ Every member of $A$ is a member of $A$. \qed
    \end{proof}

    \begin{proposition}
        For any set $A$ we have $\emptyset \subseteq A$.
    \end{proposition}

    \begin{proof}
        \pf\ Vacuously, every member of $\emptyset$ is a member of $A$. \qed
    \end{proof}

    \begin{definition}[Power Set]
        The \emph{power set} of $A$, $\mathcal{P} A$, is the set whose elements are the subsets of $A$.
    \end{definition}

    \begin{definition}[Abstraction]
        Given a property $P(x)$, we write $\{ x \mid P(x) \}$ for the set whose elements are all the objects
        $x$ such that $P(x)$ is true.
    \end{definition}
\end{document}
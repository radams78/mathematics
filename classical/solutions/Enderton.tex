\documentclass{report}

\title{Solutions Manual for Enderton \emph{Elements of Set Theory}}
\author{Robin Adams}

\usepackage{amsmath}
\usepackage{amssymb}
\usepackage{pf2}

\newcommand{\dom}{\ensuremath{\operatorname{dom}}}
\newcommand{\ran}{\ensuremath{\operatorname{ran}}}
\newcommand{\fld}{\ensuremath{\operatorname{fld}}}

\begin{document}
    \maketitle
    \tableofcontents

    \chapter{Chapter 1 --- Introduction}

    \section{Baby Set Theory}

    \paragraph{Exercise 1}
    \begin{itemize}
        \item $\{ \emptyset \} \in \{ \emptyset, \{ \emptyset \} \}$ --- true
        \item $\{ \emptyset \} \subseteq \{ \emptyset, \{ \emptyset \} \}$ --- true
        \item $\{ \emptyset \} \in \{ \emptyset, \{ \{ \emptyset \} \} \}$ --- false
        \item $\{ \emptyset \} \subseteq \{ \emptyset, \{ \{ \emptyset \} \} \}$ --- true
        \item $\{ \{ \emptyset \} \} \in \{ \emptyset, \{ \emptyset \} \}$ --- false
        \item $\{ \{ \emptyset \} \} \subseteq \{ \emptyset, \{ \emptyset \} \}$ --- true
        \item $\{ \{ \emptyset \} \} \in \{ \emptyset, \{ \{ \emptyset \} \} \}$ --- true
        \item $\{ \{ \emptyset \} \} \subseteq \{ \emptyset, \{ \{ \emptyset \} \} \}$ --- false
        \item $\{ \{ \emptyset \} \} \in \{ \emptyset, \{ \emptyset, \{ \emptyset \} \} \}$ --- false
        \item $\{ \{ \emptyset \} \} \subseteq \{ \emptyset, \{ \emptyset, \{ \emptyset \} \} \}$ --- false
    \end{itemize}

    \paragraph{Exercise 2}
    We have $\emptyset \neq \{ \emptyset \}$ because $\{ \emptyset \}$ has an element (namely $\emptyset$)
    while $\emptyset$ has no elements.

    We have $\emptyset \neq \{ \{ \emptyset \} \}$ because $\{ \{ \emptyset \} \}$ has an element (namely
    $\{ \emptyset \}$) while $\emptyset$ has no elements.

    We have $\{ \emptyset \} \neq \{ \{ \emptyset \} \}$ because $\emptyset \in \{ \emptyset \}$ but
    $\emptyset \notin \{ \{ \emptyset \} \}$. This last fact is true because $\emptyset \neq \{ \emptyset \}$
    as we proved in the first paragraph.

    \paragraph{Exercise 3}
    Assume $B \subseteq C$. Let $A \in \mathcal{P} B$; we must show that $A \in \mathcal{P} C$.

    We have $A \subseteq B$ (since $A \in \mathcal{P} B$) and $B \subseteq C$. From this it follows that
    $A \subseteq C$ (every element of $A$ is an element of $B$; every element of $B$ is an element of $C$;
    therefore every element of $A$ is an element of $C$). Hence $A \in \mathcal{P} C$ as required.

    \paragraph{Exercise 4}
    Since $x \in B$, we have $\{x\} \subseteq B$ and so $\{ x \} \in \mathcal{P} B$.

    Since $x \in B$ and $y \in B$, we have $\{ x,y \} \subseteq B$ and so $\{x,y\} \in \mathcal{P} B$.

    From these two facts, it follows that $\{ \{x\},\{x,y\}\} \subseteq \mathcal{P} B$ and so
    $\{ \{x\},\{x,y\}\} \in \mathcal{P} \mathcal{P} B$.

    \section{Sets --- An Informal View}

    \paragraph{Exercise 5}
    We have
    \begin{align*}
        V_0 & = A \\
        V_1 & = V_0 \cup \mathcal{P} V_0 \\
        & = A \cup \mathcal{P} A \\
        V_2 & = V_1 \cup \mathcal{P} V_1 \\
        & = \{ \emptyset, \{ \emptyset \} \} \\
        V_3 & = \mathcal{P} V_2 \\
        & = \{ \emptyset, \{ \emptyset \}, \{ \{ \emptyset \} \},
        \{ \emptyset, \{ \emptyset \} \} \}
    \end{align*}

    We have $\emptyset \subseteq V_0$ and so $\emptyset \in V_1$. Therefore $\{ \emptyset \} \subseteq V_1$
    and so $\{ \emptyset \} \in V_2$. Hence
    $\{ \{ \emptyset \} \} \subseteq V_2$.

    We also have $\{ \{ \emptyset \} \} \nsubseteq V_0$ because $\{ \emptyset \}$ is not an atom,
    and $\{ \{ \emptyset \} \} \nsubseteq V_1$ since $\{ \emptyset \} \notin V_1$ because $\emptyset$ is not an atom.

    Thus the rank of $\{ \{ \emptyset \} \}$ is 2.

    Likewise we have $\emptyset$ and $\{ \emptyset \}$ are both subsets of $V_1$, hence
    \[ \emptyset \in V_2, \qquad \{ \emptyset \} \in V_2 \]
    Thus $\emptyset, \{ \emptyset \}, \{ \emptyset, \{ \emptyset \} \}$ are all subsets of $V_2$,
    hence elements of $V_3$. Therefore,
    \[ \{ \emptyset, \{ \emptyset \}, \{ \emptyset, \{ \emptyset \} \} \} \subseteq V_3 \]

    Now, $\{ \emptyset, \{ \emptyset \}, \{ \emptyset, \{ \emptyset \} \} \}$ is not a subset of $V_0$
    (because $\emptyset$ is not an atom.) It is not a subset of $V_1$ ($\{ \emptyset \} \notin V_1$
    because $\emptyset$ is not an atom.) It is not a subset of $V_2$ (we have
    $\{ \emptyset, \{ \emptyset \} \} \notin V_2$ since $\{ \emptyset \} \notin V_1$).

    Therefore the rank of
    $\{ \emptyset, \{ \emptyset \}, \{ \emptyset, \{ \emptyset \} \} \}$ is 3.
    
    \paragraph{Exercise 6}

    \begin{align*}
        V_1 & = V_0 \cup \mathcal{P} V_0 \\
        & = A \cup \mathcal{P} V_0 & (\text{since } V_0 = A) \\
        V_2 & = V_1 \cup \mathcal{P} V_1 \\
        & = A \cup \mathcal{P} V_0 \cup \mathcal{P} V_1 \\
        & = A \cup \mathcal{P} V_1 & (\text{since } \mathcal{P} V_0 \subseteq \mathcal{P} V_1 \text{ by Exercise 3}) \\
        V_3 & = V_2 \cup \mathcal{P} V_2 \\
        & = A \cup \mathcal{P} V_1 \cup \mathcal{P} V_2 \\
        & = A \cup \mathcal{P} V_2 & (\text{since } \mathcal{P} V_1 \subseteq \mathcal{P} V_2 \text{ by Exercise 3}) \\
        V_4 & = V_3 \cup \mathcal{P} V_3 \\
        & = A \cup \mathcal{P} V_2 \cup \mathcal{P} V_3 \\
        & = A \cup \mathcal{P} V_3 & (\text{since } \mathcal{P} V_2 \subseteq \mathcal{P} V_3 \text{ by Exercise 3}) \\        
    \end{align*}

    \paragraph{Exercise 7}
    In Exercise 5 we calculated $V_3 = \{ \emptyset, \{ \emptyset \}, \{ \{ \emptyset \} \},
    \{ \emptyset, \{ \emptyset \} \} \}$

    Hence
    \begin{align*}
    V_4 = & \mathcal{P} V_3 \\
    = \{ & \emptyset, \\
    & \{ \emptyset \}, \\
    & \{ \{ \emptyset \} \}, \\
    & \{ \{ \{ \emptyset \} \} \}, \\
    & \{ \{ \emptyset, \{ \emptyset \} \} \}, \\
    & \{ \emptyset, \{ \emptyset \} \}, \\
    & \{ \emptyset, \{ \{ \emptyset \} \} \}, \\
    & \{ \emptyset, \{ \emptyset, \{ \emptyset \} \} \}, \\
    & \{ \{ \emptyset \}, \{ \{ \emptyset \} \} \}, \\
    & \{ \{ \emptyset \}, \{ \emptyset, \{ \emptyset \} \} \}, \\
    & \{ \{ \{ \emptyset \} \}, \{ \emptyset, \{ \emptyset \} \} \}, \\
    & \{ \emptyset, \{ \emptyset \}, \{ \{ \emptyset \} \} \}, \\
    & \{ \emptyset, \{ \emptyset \}, \{ \emptyset, \{ \emptyset \} \} \}, \\
    & \{ \emptyset, \{ \{ \emptyset \} \}, \{ \emptyset, \{ \emptyset \} \} \}, \\
    & \{ \{ \emptyset \}, \{ \{ \emptyset \} \}, \{ \emptyset, \{ \emptyset \} \} \}, \\
    & \{ \emptyset, \{ \emptyset \}, \{ \{ \emptyset \} \},
    \{ \emptyset, \{ \emptyset \} \} \} \\
    \}
    \end{align*}

    \chapter{Chapter 2 --- Axioms and Operations}

    \section{Arbitrary Unions and Intersections}

    \paragraph{Exercise 1}
    $A \cap B \cap C$ is the set of all integers that are divisible by 4, 9 and 10,
    which is the same as the set of all integers that are divisible by 180.

    \paragraph{Exercise 2}
    Take $A = \emptyset$ and $B = \{ \emptyset \}$. Then $\bigcup A = \bigcup B = \emptyset$
    but $A \neq B$. (There are many other possible answers.)

    \paragraph{Exercise 3}
    Let $b \in A$. We must show that $b \subseteq \bigcup A$.

    Let $x$ be any element of $b$. We must show that $x \in \bigcup A$. We know that $x \in b$ and $b \in A$,
    and so $x \in \bigcup A$ by the definition of $\bigcup A$.

    \paragraph{Exercise 4}
    Suppose $A \subseteq B$. Let $x \in \bigcup A$. We must show that $x \in \bigcup B$.

    Pick an element $a \in A$ such that $x \in a$. Then $a \in B$ because $A \subseteq B$. Since we know
    $x \in a$ and $a \in B$, we know that $x \in \bigcup B$.

    \paragraph{Exercise 5}
    Assume that every member of $\mathcal{A}$ is a subset of $B$. Let $x \in \bigcup \mathcal{A}$.
    We must show that $x \in B$.

    Pick $A \in \mathcal{A}$ such that $x \in A$. By our assumption, we have $A \subseteq B$. Since $x \in A$
    and $A \subseteq B$, we have $x \in B$ as required.

    \paragraph{Exercise 6}
    \subparagraph{(a)}We will show that $\bigcup \mathcal{P} A \subseteq A$ and $A \subseteq \bigcup \mathcal{P} A$.

    To show $\bigcup \mathcal{P} A \subseteq A$: This follows from Exercise 5, since every member of $\mathcal{P} A$
    is a subset of $A$.

    To show $A \subseteq \bigcup \mathcal{P} A$: Let $a \in A$. Then we have $a \in \{a\}$ and 
    $\{a\} \in \mathcal{P} A$ so $a \in \bigcup \mathcal{P} A$.

    \subparagraph{(b)}
    To show $A \subseteq \mathcal{P} \bigcup A$: This holds because every element of $A$ is a subset of
    $\bigcup A$, as we proved is Exercise 3.

    Equality holds if and only if $A = \mathcal{P} X$ for some set $X$.

    Proof: If $A = \mathcal{P} \bigcup A$ then of course $A = \mathcal{P} X$ for some $X$.

    Conversely, if $A = \mathcal{P} X$, then we have
    \begin{align*}
        \mathcal{P} \bigcup A & = \mathcal{P} \bigcup \mathcal{P} X \\
        & = \mathcal{P} X & (\text{by part (a)}) \\
        & = A
    \end{align*}

    \paragraph{Exercise 7}

    \subparagraph{(a)}
    For any set $X$,
    \begin{align*}
        & X \in \mathcal{P} A \cap \mathcal{P} B \\
        \Leftrightarrow & X \subseteq A \text{ and } X \subseteq B \\
        \Leftrightarrow & \text{Every member of $X$ is a member of $A$ and a member of $B$}  \\
        \Leftrightarrow & X \subseteq A \cap B \\
        \Leftrightarrow & X \in \mathcal{P} (A \cap B)
    \end{align*}

    \subparagraph{(b)}
    Let $X \in \mathcal{P} A \cup \mathcal{P} B$. Then either $X \in \mathcal{P} A$ or $X \in \mathcal{P} B$
    (or both). If $X \in \mathcal{P} A$, then we have $X \subseteq A$ and so $X \subseteq A \cup B$ (because
    $A \subseteq A \cup B$). Similarly if $X \in \mathcal{P} B$ then we have $X \subseteq A \cup B$. So in
    either case $X \subseteq A \cup B$, hence $X \in \mathcal{P} (A \cup B)$.

    Equality holds if and only if either $A \subseteq B$ or $B \subseteq A$.

    Proof: Suppose $A \subseteq B$. Then $\mathcal{P} A \subseteq \mathcal{P} B$ (Chapter 1 Exercise 3)
    and so $\mathcal{P} A \cup \mathcal{P} B = \mathcal{P} B$. Also $A \cup B = B$ so $\mathcal{P} (A \cup B)
    = \mathcal{P} B$. Thus $\mathcal{P} A \cup \mathcal{P} B$ and $\mathcal{P} (A \cup B)$ are equal.

    Similarly if $B \subseteq A$ then $\mathcal{P} A \cup \mathcal{P} B = \mathcal{P} (A \cup B)$.

    Conversely, suppose $\mathcal{P} A \cup \mathcal{P} B = \mathcal{P} (A \cup B)$. We have $A \cup B
    \in \mathcal{P} (A \cup B)$, so $A \cup B \in \mathcal{P} A \cup \mathcal{P} B$. If $A \cup B \in \mathcal{P}
    A$, then we have $B \subseteq A \cup B \subseteq A$. And if $A \cup B \in \mathcal{P} B$, then we have
    $A \subseteq A \cup B \subseteq B$.

    \paragraph{Exercise 8}
    If $A$ is a set such that every singleton belongs to $A$, then every set belongs to $\bigcup A$,
    contradicting Theorem 2A.

    \paragraph{Exercise 9}
    Let $a = \{ \emptyset \}$ and $B = \{ \{ \emptyset \} \}$. Then $a \in B$ but $\mathcal{P} a$
    is not a subset of $B$ because $\emptyset \in \mathcal{P} a$ and $\emptyset \notin B$.

    \paragraph{Exercise 10}
    We must show that $\mathcal{P} a \subseteq \mathcal{P} \bigcup B$. So let $X \in \mathcal{P} a$.
    Then $X \subseteq a$; we must show that $X \subseteq \bigcup B$.

    Let $x \in X$; we must show  that $x \in \bigcup B$. We have $x \in a$ (because $x \in X$ and $X \subseteq a$)
    and $a \in B$, hence $x \in \bigcup B$ as required.

    \section{Algebra of Sets}

    \paragraph{Exercise 11}

    For any $x$ we have
    \begin{align*}
        x \in (A \cap B) \cup (A - B) & \Leftrightarrow (x \in A \& x \in B) \text{ or } (x \in A \&
        x \notin B) \\
        & \Leftrightarrow x \in A \& (x \in B \text{ or } x \notin B) \\
        & \Leftrightarrow x \in A
    \end{align*}
    Hence $A = (A \cap B) \cup (A - B)$.

    For any $x$ we have
    \begin{align*}
        x \in A \cup (B - A) & \Leftrightarrow x \in A \text{ or } (x \in B \& x \notin A) \\
        & \Leftrightarrow x \in A \text{ or } x \in B \\
        & \Leftrightarrow x \in A \cup B
    \end{align*}
    Hence $A \cup (B - A) = A \cup B$.

    \paragraph{Exercise 12}
    For any $x$,
    \begin{align*}
        x \in C - (A \cap B) & \Leftrightarrow x \in C \& \neg (x \in A \& x \in B) \\
        & \Leftrightarrow x \in C \& (x \notin A \text{ or } x \notin B) \\
        & \Leftrightarrow (x \in C \& x \notin A) \text{ or } (x \in C \& x \notin B) \\
        & \Leftrightarrow x \in (C - A) \cup (C - B)
    \end{align*}

    \paragraph{Exercise 13}
    Suppose $A \subseteq B$. Let $x \in C - B$; we must show $x \in C - A$. We have $x \in C$ and $x \notin B$. Therefore $x \notin A$,
    since every member of $A$ is a member of $B$. And so we have $x \in C - A$ as required.

    \paragraph{Exercise 14}
    Let $A = \{ \emptyset \}$, $B = \emptyset$ and $C = \{ \emptyset \}$. Then $A - (B - C) = A - \emptyset
    = \{ \emptyset \}$ while $(A - B) - C = \{ \emptyset \} - C = \emptyset$.

    \paragraph{Exercise 15}
    \subparagraph{(a)}
    For any $x$ we have the following eight possibilities:

    \begin{tabular}{|c|c|c|c|c|}
        \hline
        $x \in A$ & $x \in B$ & $x \in C$ & $x \notin A \cap (B + C)$ & $x \notin (A \cap B) + (A \cap C) $ \\
        $x \in A$ & $x \in B$ & $x \notin C$ & $x \in A \cap (B + C)$ & $x \in (A \cap B) + (A \cap C) $ \\
        $x \in A$ & $x \notin B$ & $x \in C$ & $x \in A \cap (B + C)$ & $x \in (A \cap B) + (A \cap C) $ \\
        $x \in A$ & $x \notin B$ & $x \notin C$ & $x \notin A \cap (B + C)$ & $x \notin (A \cap B) + (A \cap C) $ \\
        $x \notin A$ & $x \in B$ & $x \in C$ & $x \notin A \cap (B + C)$ & $x \notin (A \cap B) + (A \cap C) $ \\
        $x \notin A$ & $x \in B$ & $x \notin C$ & $x \notin A \cap (B + C)$ & $x \notin (A \cap B) + (A \cap C) $ \\
        $x \notin A$ & $x \notin B$ & $x \in C$ & $x \notin A \cap (B + C)$ & $x \notin (A \cap B) + (A \cap C) $ \\
        $x \notin A$ & $x \notin B$ & $x \notin C$ & $x \notin A \cap (B + C)$ & $x \notin (A \cap B) + (A \cap C) $ \\
        \hline
    \end{tabular}

    In every case, we have $x \in A \cap (B + C) \Leftrightarrow x \in (A \cap B) + (A \cap C)$.

    \subparagraph{(b)}
    For any $x$ we have the following eight possibilities:

    \begin{tabular}{|c|c|c|c|c|}
        \hline
        $x \in A$ & $x \in B$ & $x \in C$ & $x \in A + (B + C)$ & $x \in (A + B) + C $ \\
        $x \in A$ & $x \in B$ & $x \notin C$ & $x \notin A + (B + C)$ & $x \notin (A + B) + C $ \\
        $x \in A$ & $x \notin B$ & $x \in C$ & $x \notin A + (B + C)$ & $x \notin (A + B) + C $ \\
        $x \in A$ & $x \notin B$ & $x \notin C$ & $x \in A + (B + C)$ & $x \in (A + B) + C $ \\
        $x \notin A$ & $x \in B$ & $x \in C$ & $x \notin A + (B + C)$ & $x \notin (A + B) + C $ \\
        $x \notin A$ & $x \in B$ & $x \notin C$ & $x \in A + (B + C)$ & $x \in (A + B) + C $ \\
        $x \notin A$ & $x \notin B$ & $x \in C$ & $x \in A + (B + C)$ & $x \in (A + B) + C $ \\
        $x \notin A$ & $x \notin B$ & $x \notin C$ & $x \notin A + (B + C)$ & $x \notin (A + B) + C $ \\
        \hline
    \end{tabular}

    In every case, we have $x \in A + (B + C) \Leftrightarrow x \in (A + B) + C$.

    \paragraph{Exercise 16}

    \begin{align*}
        [(A \cup B \cup C) \cap (A \cup B)] - [(A \cup (B - C)) \cap A]
        & = (A \cup B) - A \\
        & = B - A
    \end{align*}

    \paragraph{Exercise 17}
    \subparagraph{(a) $\Leftrightarrow$ (b)}
    \begin{align*}
        A \subseteq B & \Leftrightarrow \text{Every element of $A$ is an element of $B$} \\
        & \Leftrightarrow \text{There is no element of $A$ that is not an element of $B$} \\
        & \Leftrightarrow A - B = \emptyset
    \end{align*}

    \subparagraph{(a) $\Rightarrow$ (c)}
    Suppose $A \subseteq B$. We have $B \subseteq A \cup B$ from the definition of $A \cup B$;
    we must prove that $A \cup B \subseteq B$. So let $x \in A \cup B$. Then $x \in A$ or $x \in B$.
    But in either case $x \in B$, since $x \in A \Rightarrow x \in B$. Thus we have $x \in B$ as required.

    \subparagraph{(c) $\Rightarrow$ (a)}
    We always have $A \subseteq A \cup B$. So if $A \cup B = B$ then we have $A \subseteq B$.

    \subparagraph{(a) $\Rightarrow$ (d)}
    Suppose $A \subseteq B$. We have $A \cap B \subseteq A$ from the definition of $A \cap B$;
    we must prove that $A \subseteq A \cap B$. So let $x \in A$. Then $x \in B$ since $A \subseteq B$,
    hence $x \in A \cap B$ as required.

    \subparagraph{(d) $\Rightarrow$ (a)}
    We always have $A \cap B \subseteq B$. So if $A \cap B = A$ then $A \subseteq B$.

    \paragraph{Exercise 18}
    We can make the following 16 sets:
    \begin{itemize}
        \item $\emptyset$ ($= A - A$)
        \item $A - B$
        \item $A \cap B$
        \item $B - A$
        \item $S - (A \cup B)$
        \item $A$
        \item $A + B$
        \item $S - B$
        \item $B$
        \item $S - (A + B)$
        \item $S - A$
        \item $A \cup B$
        \item $S - (B - A)$
        \item $S - (A \cap B)$
        \item $S - (A - B)$
    \end{itemize}

    \paragraph{Exercise 19}
    They are never equal, because for all $A$, $B$, we have $\emptyset \in \mathcal{P}(A - B)$ but
    $\emptyset \notin \mathcal{P} A - \mathcal{P} B$ since $\emptyset \in \mathcal{P} B$.

    \paragraph{Exercise 20}
    Assume $A \cup B = A \cup C$ and $A \cap B = A \cap C$.

    We first show $B \subseteq C$. Let $x \in B$; we show $x \in C$. We have $x \in A \cup B = A \cup C$, so either $x \in A$
    or $x \in C$. If $x \in C$, we are done. If $x \in A$, then we have $x \in A \cap B = A \cap C$,
    and so $x \in C$ in this case too.

    We can show $C \subseteq B$ similarly. Hence $B = C$.

    \paragraph{Exercise 21}
    For any $x$, we have
    \begin{align*}
        x \in \bigcup (A \cup B) & \Leftrightarrow \text{there exists $C$ such that $C \in A \cup B$
        and $x \in C$} \\
        & \Leftrightarrow \text{there exists $C \in A$ such that $x \in C$, or there exists $C \in B$
        such that $x \in C$} \\
        & \Leftrightarrow x\in \bigcup A \cup \bigcup B
    \end{align*}

    \paragraph{Exercise 22}
    For any $x$, we have
    \begin{align*}
        x \in \bigcap (A \cup B) & \Leftrightarrow \text{for all $C$, if $C \in A$ or $C \in B$
        then $x \in C$} \\
        & \Leftrightarrow \text{for all $C \in A$ we have $x \in C$, and for all $C \in B$
        we have $x \in C$} \\
        & \Leftrightarrow x\in \bigcap A \cap \bigcap B
    \end{align*}

    \paragraph{Exercise 23}
    \begin{proof}
        \pf
        \step{1}{$A \subseteq \bigcap \{ A \cup X \mid X \in \mathcal{B} \}$}
        \begin{proof}
            \step{a}{\pflet{$x \in A$}}
            \step{b}{\pflet{$X \in \mathcal{B}$}}
            \step{c}{$x \in A \cup X$}
        \end{proof}
        \step{2}{$\bigcap \mathcal{B} \subseteq \bigcap \{ A \cup X \mid X \in \mathcal{B} \}$}
        \begin{proof}
            \step{a}{\pflet{$x \in \bigcap \mathcal{B}$}}
            \step{b}{\pflet{$X \in \mathcal{B}$}}
            \step{c}{$x \in X$}
            \step{d}{$x \in A \cup X$}
        \end{proof}
        \step{3}{$\bigcap \{ A \cup X \mid X \in \mathcal{B} \} \subseteq A \cup \bigcap \mathcal{B}$}
        \begin{proof}
            \step{a}{\pflet{$x \in \bigcap \{ A \cup X \mid X \in \mathcal{B} \}$}}
            \step{b}{\assume{$x \notin A$} \prove{$x \in \bigcap \mathcal{B}$}}
            \step{c}{\pflet{$X \in \mathcal{B}$}}
            \step{d}{$x \in A \cup X$}
            \step{e}{$x \in X$}
        \end{proof}
        \qed
    \end{proof}

    \paragraph{Exercise 24}
    \subparagraph{(a)}
    \begin{align*}
        Y \in \mathcal{P} \bigcap \mathcal{A}
        & \Leftrightarrow Y \subseteq \bigcap \mathcal{A} \\
        & \Leftrightarrow \forall y \in Y. \forall X \in \mathcal{A}. y \in X \\
        & \Leftrightarrow \forall X \in \mathcal{A}. \forall y \in Y. y \in X \\
        & \Leftrightarrow \forall X \in \mathcal{A}. Y \in \mathcal{P} X \\
        & \Leftrightarrow Y \in \bigcap \{ \mathcal{P} X \mid X \in \mathcal{A} \}
    \end{align*}

    \subparagraph{(b)}
    $\bigcup \{ \mathcal{P} X \mid X \in \mathcal{A} \} \subseteq \mathcal{P} \bigcup \mathcal{A}$

    \begin{proof}
        \pf
        \step{1}{\pflet{$Y \in \bigcup \{ \mathcal{P} X \mid X \in \mathcal{A} \}$}}
        \step{2}{\pick\ $X \in \mathcal{A}$ such that $Y \in \mathcal{P} X$}
        \step{3}{$Y \subseteq X$}
        \step{4}{$Y \subseteq \bigcup \mathcal{A}$}
        \step{5}{$Y \in \mathcal{P} \bigcup \mathcal{A}$}
    \end{proof}

    Equality holds if and only if $\bigcup \mathcal{A} \in \mathcal{A}$.
    \begin{proof}
        \step{1}{If $\bigcup \{ \mathcal{P} X \mid X \in \mathcal{A} \} = \mathcal{P} \bigcup \mathcal{A}$
        then $\bigcup \mathcal{A} \in \mathcal{A}$}
        \begin{proof}
            \step{a}{\assume{$\bigcup \{ \mathcal{P} X \mid X \in \mathcal{A} \} = \mathcal{P} \bigcup \mathcal{A}$}}
            \step{b}{$\bigcup \mathcal{A} \in \bigcup \{ \mathcal{P} X \mid X \in \mathcal{A} \}$}
            \step{c}{\pick\ $X \in \mathcal{A}$ such that $\bigcup \mathcal{A} \in \mathcal{P} X$}
            \step{d}{$X = \bigcup \mathcal{A}$}
        \end{proof}
        \step{2}{If $\bigcup \mathcal{A} \in \mathcal{A}$ then
        $\bigcup \{ \mathcal{P} X \mid X \in \mathcal{A} \} = \mathcal{P} \bigcup \mathcal{A}$}
        \begin{proof}
            \pf\ If $\bigcup \mathcal{A} \in \mathcal{A}$ then $\mathcal{P} \bigcup \mathcal{A}
            \in \{ \mathcal{P} X \mid X \in \mathcal{A} \}$.
        \end{proof}
        \qed
    \end{proof}

    \paragraph{Exercise 25}
    We have $A \cup \bigcup \mathcal{B} = \bigcup \{ A \cup X \mid X \in \mathcal{B} \}$ if and only if
    $A = \emptyset$ or $\mathcal{B} \neq \emptyset$

    \begin{proof}
        \step{1}{If $A \cup \bigcup \mathcal{B}  = \bigcup \{ A \cup X \mid X \in \mathcal{B} \}$
        then $A = \emptyset$ or $\mathcal{B} \neq \emptyset$}
        \begin{proof}
            \pf\ If $A \cup \bigcup \mathcal{B}  = \bigcup \{ A \cup X \mid X \in \mathcal{B} \}$
            and $\mathcal{B} = \emptyset$ then
            \begin{align*}
                A \cup \bigcup \emptyset & = \bigcup \emptyset \\
                \therefore A & = \emptyset
            \end{align*}
        \end{proof}
        \step{2}{If $A = \emptyset$ then $A \cup \bigcup \mathcal{B}  = \bigcup \{ A \cup X \mid X \in \mathcal{B} \}$}
        \begin{proof}
            \pf\ Both sides are equal to $\bigcup \mathcal{B}$
        \end{proof}
        \step{3}{If $\mathcal{B} \neq \emptyset$ then $A \cup \bigcup \mathcal{B}  = \bigcup \{ A \cup X \mid X \in \mathcal{B} \}$}
        \begin{proof}
            \step{a}{\assume{$\mathcal{B} \neq \emptyset$}}
            \step{b}{$A \cup \bigcup \mathcal{B} \subseteq \bigcup \{ A \cup X \mid X \in \mathcal{B} \}$}
            \begin{proof}
                \step{i}{\pflet{$x \in A \cup \bigcup \mathcal{B}$}
                \prove{$x \in \bigcup \{ A \cup X \mid X \in \mathcal{B} \}$}}
                \step{ii}{\case{$x \in A$}}
                \begin{proof}
                    \step{one}{\pick\ $X \in \mathcal{B}$}
                    \begin{proof}
                        \pf\ By \stepref{a}
                    \end{proof}
                    \step{two}{$x \in A \cup X$}
                \end{proof}
                \step{iii}{\case{$x \in \bigcup \mathcal{B}$}}
                \begin{proof}
                    \step{one}{\pick\ $X \in \mathcal{B}$ such that $x \in X$}
                    \step{two}{$x \in A \cup X$}
                \end{proof}
            \end{proof}
            \step{c}{$\bigcup \{ A \cup X \mid X \in \mathcal{B} \} \subseteq A \cup \bigcup \mathcal{B}$}
            \begin{proof}
                \step{i}{\pflet{$x \in \bigcup \{ A \cup X \mid X \in \mathcal{B} \}$}}
                \step{ii}{\pick\ $X \in \mathcal{B}$ such that $x \in A \cup X$}
                \step{iii}{$X \subseteq \bigcup \mathcal{B}$}
                \step{iv}{$A \cup X \subseteq A \cup \bigcup \mathcal{B}$}
                \step{v}{$x \in A \cup \bigcup \mathcal{B}$}
            \end{proof}
        \end{proof}
    \end{proof}

    \section{Review Exercises}

    \paragraph{Exercise 26}
    Sets $A$, $B$, $D$ and $F$ are all equal to each other. Sets $C$, $E$ and $G$ are equal to each other.
    None of the first list is equal to any of the second list.

    \paragraph{Exercise 27}
    Take $A = \{ \{ 0 \}, \{ 1 \} \}$ and $B = \{ \{ 1 \} \}$.
    Then $A \cap B = \{ \{ 1 \} \}$
    and
    \begin{align*}
        \bigcap A \cap \bigcap B & = \emptyset \cap \{1\} \\
        & = \emptyset \\
        \bigcap (A \cap B) & = \bigcap \{ \{ 1 \} \} \\
        & = \{ 1 \}
    \end{align*}

    \paragraph{Exercise 28}
    \begin{align*}
        \bigcup \{ \{ 3,4 \}, \{ \{3\}, \{4\} \}, \{3, \{4\} \}, \{ \{3\}, 4\} \}
        & = \{ 3, 4, \{3\}, \{4\} \}
    \end{align*}

    \paragraph{Exercise 29}
    \subparagraph{(a)}
    $\emptyset$
    \subparagraph{(b)}
    We have 
    \begin{align*}
        \{ \emptyset \} & \subseteq \mathcal{P} \{ \emptyset \} \\
        \therefore \mathcal{P} \{\emptyset\} & \subseteq \mathcal{PP} \{ \emptyset \} \\
        \{ \emptyset \} & \subseteq \mathcal{PP} \{ \emptyset \} \\
        \therefore \mathcal{P} \{\emptyset\} & \subseteq \mathcal{PPP} \{ \emptyset \} \\
        \therefore \bigcap \{ \mathcal{PPP} \{ \emptyset \}, \mathcal{PP} \{ \emptyset \}, \mathcal{P} \{ \emptyset \} \}
        & = \mathcal{PPP} \{ \emptyset \} \cap \mathcal{PP} \{ \emptyset \} \cap \mathcal{P} \{ \emptyset \} \\
        & = \mathcal{P} \{ \emptyset \} \\
        & = \{ \emptyset, \{ \emptyset \} \}
    \end{align*}

    \paragraph{Exercise 30}
    \subparagraph{(a)}
    $\{ \emptyset, \{ \{ \emptyset \} \}, \{ \{ \{ \emptyset \} \} \}, \{ \{ \emptyset \}, \{ \{ \emptyset \} \} \} \}$
    \subparagraph{(b)}
    $\{ \emptyset, \{ \emptyset \} \}$
    \subparagraph{(c)}
    $\{ \emptyset, \{ \emptyset \}, \{ \{ \emptyset \} \}, \{ \emptyset, \{ \emptyset \} \} \}$
    \subparagraph{(d)}
    $\{ \{ \emptyset \}, \{ \{ \emptyset \} \} \}$

    \paragraph{Exercise 31}
    \subparagraph{(a)}
    $\{ 1, 2, 3, \emptyset \}$
    \subparagraph{(b)}
    $\emptyset$
    \subparagraph{(c)}
    $\emptyset$
    \subparagraph{(d)}
    $\emptyset$

    \paragraph{Exercise 32}
    \subparagraph{(a)}
    $a \cup b$
    \subparagraph{(b)}
    $a$
    \subparagraph{(c)}
    \begin{align*}
        \bigcap \bigcup S \cup (\bigcup \bigcup S - \bigcup \bigcap S)
        & = (a \cap b) \cup ((a \cup b) - a) \\
        & = (a \cap b) \cup (b - a) \\
        & = b
    \end{align*}

    \paragraph{Exercise 33}
    When $a \neq b$:
    \begin{align*}
        \bigcup (\bigcup S - \bigcap S) & = \bigcup (\{ a,b \} - \{a\}) \\
        & = \bigcup \{b\} \\
        & = b
    \end{align*}
    When $a = b$:
    \begin{align*}
        \bigcup (\bigcup S - \bigcap S) & = \bigcup (\{ a,b \} - \{a\}) \\
        & = \bigcup \emptyset \\
        & = \emptyset
    \end{align*}

    \paragraph{Exercise 34}
    For any set $S$, we have
    \begin{align*}
        \emptyset & \subseteq \mathcal{P} S \\
        \therefore \emptyset & \in \mathcal{PP} S \\
        \emptyset & \subseteq S \\
        \therefore \emptyset & \in \mathcal{P} S \\
        \therefore \{ \emptyset \} & \subseteq \mathcal{P} S \\
        \therefore \{ \emptyset \} & \in \mathcal{PP} S \\
        \therefore \{ \emptyset, \{ \emptyset \} \} & \subseteq \mathcal{PP} S \\
        \therefore \{ \emptyset, \{ \emptyset \} \} & \in \mathcal{PPP} S
    \end{align*}

    \paragraph{Exercise 35}
    Assume $\mathcal{P}A = \mathcal{P} B$. Then we have
    \begin{align*}
        A & \in \mathcal{P} A \\
        \therefore A & \in \mathcal{P} B \\
        \therefore A & \subseteq B \\
        B & \in \mathcal{P} B \\
        \therefore B & \in \mathcal{P} A \\
        \therefore B & \subseteq A \\
        \therefore A & = B
    \end{align*}

    \paragraph{Exercise 36}
    \subparagraph{(a)}
    \begin{align*}
        x \in A - (A \cap B) & \Leftrightarrow x \in A \ \& \neg (x \in A \ \& \ x \in B) \\
        & \Leftrightarrow x \in A\ \& \ x \notin B \\
        & \Leftrightarrow x \in A - B
    \end{align*}
    
    \subparagraph{(b)}
    \begin{align*}
        x \in A - (A - B) & \Leftrightarrow x \in A \ \& \neg (x \in A \ \&\ x \notin B) \\
        & \Leftrightarrow x \in A \ \& \ x \in B \\
        & \Leftrightarrow x \in A \cap B
    \end{align*}

    \paragraph{Exercise 37}
    \subparagraph{(a)}
    \begin{align*}
        x \in (A \cup B) - C & \Leftrightarrow (x \in A \text{ or } x \in B) \ \&\ x \notin C \\
        & \Leftrightarrow (x \in A \ \&\ x \notin C) \text{ or } (x \in B \ \&\ x \notin C) \\
        & \Leftrightarrow x \in (A - C) \cup (B - C)
    \end{align*}

    \subparagraph{(b)}
    \begin{align*}
        x \in A - (B - C) & \Leftrightarrow x \in A \ \& \neg (x \in B \ \&\ x \notin C) \\
        & \Leftrightarrow x \in A \ \& (x \notin B \text{ or } x \in C) \\
        & \Leftrightarrow (x \in A \ \&\ x \notin B) \text{ or } (x \in A \ \&\ x \in C) \\
        & \Leftrightarrow x \in (A - B) \cup (A \cap C)
    \end{align*}

    \subparagraph{(c)}
    \begin{align*}
        x \in (A - B) - C & \Leftrightarrow x \in A \ \&\ x \notin B \ \&\ x \notin C \\
        & \Leftrightarrow x \in A \ \& \neg (x \in B \vee x \in C) \\
        & \Leftrightarrow x \in A - (B \cup C)
    \end{align*}
    
    \paragraph{Exercise 38}
    \subparagraph{(a)}
    If every element of $A$ is an element of $C$, and every element of $B$ is an element of $C$,
    then everything that is an element of either $A$ or $B$ is an element of $C$.

    \subparagraph{(b)}
    If every element of $C$ is an element of $A$, and every element of $C$ is an element of $B$,
    then every element of $C$ is an element of both $A$ and $B$.

    \chapter{Chapter 3 --- Relations and Functions}

    \section{Ordered Pairs}

    \paragraph{Exercise 1}
    We have $\langle 0, 1, 0 \rangle^* = \langle 0, 1, 1 \rangle^* = \{ \{ 0 \}, \{ 0,1 \} \}$.

    \paragraph{Exercise 2}
    \subparagraph{(a)}
    \begin{align*}
        & z \in A \times (B \cup C) \\
        \Leftrightarrow & \exists x,y (z = (x,y) \ \&\ x \in A \ \&\ (y \in B \text{ or } y \in C)) \\
        \Leftrightarrow & \exists x,y (z = (x,y)\ \&\ x \in A\ \&\ y \in B) \text{ or }
        (z = (x,y)\ \&\ x \in A\ \&\ y \in C) \\
        \Leftrightarrow & z \in (A \times B) \cup (A \times C)
    \end{align*}
    \subparagraph{(b)}
    \begin{proof}
        \step{1}{\assume{$A \times B = A \times C$ and $A \neq \emptyset$}}
        \step{2}{\pick\ $a \in A$}
        \step{3}{For all $x$, $x \in B \Leftrightarrow x \in C$}
        \begin{proof}
            \pf\ $x \in B$ iff $(a,x) \in A \times B$ iff $(a,x) \in A \times C$ iff $x \in C$.
        \end{proof}
        \qed
    \end{proof}

    \paragraph{Exercise 3}
    \begin{align*}
        & z \in A \times \bigcup \mathcal{B} \\
        \Leftrightarrow & \exists x,y (z = (x,y)\ \&\ x \in A\ \&\ \exists X \in \mathcal{B}. y \in X) \\
        \Leftrightarrow & \exists X \in \mathcal{B}. \exists x,y (z = (x,y)\ \&\ x \in A\ \&\ y \in X) \\
        \Leftrightarrow & z \in \bigcup \{ A \times X : X \in \mathcal{B} \}
    \end{align*}

    \paragraph{Exercise 4}
    If every ordered pair belongs to $A$ then every set belongs to $\bigcup \bigcup A$ contradicting
    Theorem 2A.

    \paragraph{Exercise 5}
    \subparagraph{(a)}
    Apply a Subset Axiom to $\mathcal{P}(A \times B)$: we have 
    $C = \{ z \in \mathcal{P}(A \times B) \mid \exists x \in A. z = \{x\} \times B \}$.
    \subparagraph{(b)}
    \begin{align*}
        & z \in \bigcup C \\
        \Leftrightarrow & \exists x \in A. z \in \{x\} \times B \\
        \Leftrightarrow & \exists x \in A. \exists y \in B. z = (x,y) \\
        \Leftrightarrow & z \in A \times B
    \end{align*}

    \section{Relations}

    \paragraph{Exercise 6}
    If $A \subseteq \dom A \times \ran A$ then $A$ is a set of ordered pairs, i.e. a relation.

    Conversely, suppose $A$ is a relation. Let $z \in A$. Then $z$ is an ordered pair; let $z = (x,y)$.
    We have $x \in \dom A$ and $y \in \ran A$ and so $z \in \dom A \times \ran A$ as required.

    \paragraph{Exercise 7}
    We have $\fld R \subseteq \bigcup \bigcup R$ by Lemma 3D.

    Conversely, let $x \in \bigcup \bigcup R$. Pick $a$ and $b$ such that $x \in a$, $a \in b$ and $b \in R$.
    Then $b$ is an ordered pair; let $b = (y,z)$. We have $a = \{y\}$ or $\{y,z\}$, hence $x = y$ or
    $x = z$. In either case, $x \in \fld R$.

    \paragraph{Exercise 8}
    \subparagraph{(a)}
    \begin{align*}
        & x \in \dom \bigcup \mathcal{A} \\
        \Leftrightarrow & \exists y. \exists R \in \mathcal{A}. (x,y) \in R \\
        \Leftrightarrow & \exists R \in \mathcal{A}. \exists y. (x,y) \in R \\
        \Leftrightarrow & x \in \bigcup \{ \dom R : R \in \mathcal{A} \}
    \end{align*}

    \subparagraph{(b)}
    \begin{align*}
        & y \in \ran \bigcup \mathcal{A} \\
        \Leftrightarrow & \exists x. \exists R \in \mathcal{A}. (x,y) \in R \\
        \Leftrightarrow & \exists R \in \mathcal{A}. \exists x. (x,y) \in R \\
        \Leftrightarrow & y \in \bigcup \{ \ran R : R \in \mathcal{A} \}
    \end{align*}

    \paragraph{Exercise 9}
    Assume $\mathcal{A}$ is nonempty.
    We have $\dom \bigcap \mathcal{A} \subseteq \bigcap \{ \dom R : R \in \mathcal{A} \}$.

    \begin{proof}
        \pf
            \begin{align*}
                & x \in \dom \bigcap \mathcal{A} \\
                \Leftrightarrow & \exists y. \forall R \in \mathcal{A}. (x,y) \in R \\
                \Rightarrow & \forall R \in \mathcal{A}. \exists y. (x,y) \in R \\
                \Leftrightarrow & x \in \bigcap \{ \dom R : R \in \mathcal{A} \}
            \end{align*}                    
    \end{proof}

    Equality holds iff the middle '$\Rightarrow$' can be reversed, i.e. iff for all $x$,
    if $\forall R \in \mathcal{A}. \exists y. (x,y) \in R$ then $\exists y. \forall R \in \mathcal{A}. (x,y) \in R$.
    I haven't found a simpler condition than this. The condition does not always hold, for example
    if $\mathcal{A} = \{ \{ (1,2) \}, \{ (1,3) \} \}$ then $\dom \bigcap \mathcal{A} = \emptyset$
    while $\bigcap \{ \dom R : R \in \mathcal{A} \} = \{1\}$.

    Similarly, $\ran \bigcap \mathcal{A} \subseteq \bigcap \{ \ran R : R \in \mathcal{A} \}$, and
    equality holds iff, for any $y$, if $\forall R \in \mathcal{A}. \exists x. (x,y) \in R$ then $\exists x. \forall R \in \mathcal{A}. (x,y) \in R$.
\end{document}
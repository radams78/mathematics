\documentclass{report}

\title{Solutions Manual for Enderton \emph{Elements of Set Theory}}
\author{Robin Adams}

\usepackage{amsmath}
\usepackage{amssymb}

\begin{document}
    \maketitle
    \tableofcontents

    \chapter{Chapter 1 --- Introduction}

    \section{Baby Set Theory}

    \subsection{Exercise 1}
    \begin{itemize}
        \item $\{ \emptyset \} \in \{ \emptyset, \{ \emptyset \} \}$ --- true
        \item $\{ \emptyset \} \subseteq \{ \emptyset, \{ \emptyset \} \}$ --- true
        \item $\{ \emptyset \} \in \{ \emptyset, \{ \{ \emptyset \} \} \}$ --- false
        \item $\{ \emptyset \} \subseteq \{ \emptyset, \{ \{ \emptyset \} \} \}$ --- true
        \item $\{ \{ \emptyset \} \} \in \{ \emptyset, \{ \emptyset \} \}$ --- false
        \item $\{ \{ \emptyset \} \} \subseteq \{ \emptyset, \{ \emptyset \} \}$ --- true
        \item $\{ \{ \emptyset \} \} \in \{ \emptyset, \{ \{ \emptyset \} \} \}$ --- true
        \item $\{ \{ \emptyset \} \} \subseteq \{ \emptyset, \{ \{ \emptyset \} \} \}$ --- false
        \item $\{ \{ \emptyset \} \} \in \{ \emptyset, \{ \emptyset, \{ \emptyset \} \} \}$ --- false
        \item $\{ \{ \emptyset \} \} \subseteq \{ \emptyset, \{ \emptyset, \{ \emptyset \} \} \}$ --- false
    \end{itemize}

    \subsection{Exercise 2}
    We have $\emptyset \neq \{ \emptyset \}$ because $\{ \emptyset \}$ has an element (namely $\emptyset$)
    while $\emptyset$ has no elements.

    We have $\emptyset \neq \{ \{ \emptyset \} \}$ because $\{ \{ \emptyset \} \}$ has an element (namely
    $\{ \emptyset \}$) while $\emptyset$ has no elements.

    We have $\{ \emptyset \} \neq \{ \{ \emptyset \} \}$ because $\emptyset \in \{ \emptyset \}$ but
    $\emptyset \notin \{ \{ \emptyset \} \}$. This last fact is true because $\emptyset \neq \{ \emptyset \}$
    as we proved in the first paragraph.

    \subsection{Exercise 3}
    Assume $B \subseteq C$. Let $A \in \mathcal{P} B$; we must show that $A \in \mathcal{P} C$.

    We have $A \subseteq B$ (since $A \in \mathcal{P} B$) and $B \subseteq C$. From this it follows that
    $A \subseteq C$ (every element of $A$ is an element of $B$; every element of $B$ is an element of $C$;
    therefore every element of $A$ is an element of $C$). Hence $A \in \mathcal{P} C$ as required.

    \subsection{Exercise 4}
    Since $x \in B$, we have $\{x\} \subseteq B$ and so $\{ x \} \in \mathcal{P} B$.

    Since $x \in B$ and $y \in B$, we have $\{ x,y \} \subseteq B$ and so $\{x,y\} \in \mathcal{P} B$.

    From these two facts, it follows that $\{ \{x\},\{x,y\}\} \subseteq \mathcal{P} B$ and so
    $\{ \{x\},\{x,y\}\} \in \mathcal{P} \mathcal{P} B$.

    \section{Sets --- An Informal View}

    \subsection{Exercise 5}
    We have
    \begin{align*}
        V_0 & = A \\
        V_1 & = V_0 \cup \mathcal{P} V_0 \\
        & = A \cup \mathcal{P} A \\
        V_2 & = V_1 \cup \mathcal{P} V_1 \\
        & = \{ \emptyset, \{ \emptyset \} \} \\
        V_3 & = \mathcal{P} V_2 \\
        & = \{ \emptyset, \{ \emptyset \}, \{ \{ \emptyset \} \},
        \{ \emptyset, \{ \emptyset \} \} \}
    \end{align*}

    We have $\emptyset \subseteq V_0$ and so $\emptyset \in V_1$. Therefore $\{ \emptyset \} \subseteq V_1$
    and so $\{ \emptyset \} \in V_2$. Hence
    $\{ \{ \emptyset \} \} \subseteq V_2$.

    We also have $\{ \{ \emptyset \} \} \nsubseteq V_0$ because $\{ \emptyset \}$ is not an atom,
    and $\{ \{ \emptyset \} \} \nsubseteq V_1$ since $\{ \emptyset \} \notin V_1$ because $\emptyset$ is not an atom.

    Thus the rank of $\{ \{ \emptyset \} \}$ is 2.

    Likewise we have $\emptyset$ and $\{ \emptyset \}$ are both subsets of $V_1$, hence
    \[ \emptyset \in V_2, \qquad \{ \emptyset \} \in V_2 \]
    Thus $\emptyset, \{ \emptyset \}, \{ \emptyset, \{ \emptyset \} \}$ are all subsets of $V_2$,
    hence elements of $V_3$. Therefore,
    \[ \{ \emptyset, \{ \emptyset \}, \{ \emptyset, \{ \emptyset \} \} \} \subseteq V_3 \]

    Now, $\{ \emptyset, \{ \emptyset \}, \{ \emptyset, \{ \emptyset \} \} \}$ is not a subset of $V_0$
    (because $\emptyset$ is not an atom.) It is not a subset of $V_1$ ($\{ \emptyset \} \notin V_1$
    because $\emptyset$ is not an atom.) It is not a subset of $V_2$ (we have
    $\{ \emptyset, \{ \emptyset \} \} \notin V_2$ since $\{ \emptyset \} \notin V_1$).

    Therefore the rank of
    $\{ \emptyset, \{ \emptyset \}, \{ \emptyset, \{ \emptyset \} \} \}$ is 3.
    
    \subsection{Exercise 6}

    \begin{align*}
        V_1 & = V_0 \cup \mathcal{P} V_0 \\
        & = A \cup \mathcal{P} V_0 & (\text{since } V_0 = A) \\
        V_2 & = V_1 \cup \mathcal{P} V_1 \\
        & = A \cup \mathcal{P} V_0 \cup \mathcal{P} V_1 \\
        & = A \cup \mathcal{P} V_1 & (\text{since } \mathcal{P} V_0 \subseteq \mathcal{P} V_1 \text{ by Exercise 3}) \\
        V_3 & = V_2 \cup \mathcal{P} V_2 \\
        & = A \cup \mathcal{P} V_1 \cup \mathcal{P} V_2 \\
        & = A \cup \mathcal{P} V_2 & (\text{since } \mathcal{P} V_1 \subseteq \mathcal{P} V_2 \text{ by Exercise 3}) \\
        V_4 & = V_3 \cup \mathcal{P} V_3 \\
        & = A \cup \mathcal{P} V_2 \cup \mathcal{P} V_3 \\
        & = A \cup \mathcal{P} V_3 & (\text{since } \mathcal{P} V_2 \subseteq \mathcal{P} V_3 \text{ by Exercise 3}) \\        
    \end{align*}

    \subsection{Exercise 7}
    In Exercise 5 we calculated $V_3 = \{ \emptyset, \{ \emptyset \}, \{ \{ \emptyset \} \},
    \{ \emptyset, \{ \emptyset \} \} \}$

    Hence
    \begin{align*}
    V_4 = & \mathcal{P} V_3 \\
    = \{ & \emptyset, \\
    & \{ \emptyset \}, \\
    & \{ \{ \emptyset \} \}, \\
    & \{ \{ \{ \emptyset \} \} \}, \\
    & \{ \{ \emptyset, \{ \emptyset \} \} \}, \\
    & \{ \emptyset, \{ \emptyset \} \}, \\
    & \{ \emptyset, \{ \{ \emptyset \} \} \}, \\
    & \{ \emptyset, \{ \emptyset, \{ \emptyset \} \} \}, \\
    & \{ \{ \emptyset \}, \{ \{ \emptyset \} \} \}, \\
    & \{ \{ \emptyset \}, \{ \emptyset, \{ \emptyset \} \} \}, \\
    & \{ \{ \{ \emptyset \} \}, \{ \emptyset, \{ \emptyset \} \} \}, \\
    & \{ \emptyset, \{ \emptyset \}, \{ \{ \emptyset \} \} \}, \\
    & \{ \emptyset, \{ \emptyset \}, \{ \emptyset, \{ \emptyset \} \} \}, \\
    & \{ \emptyset, \{ \{ \emptyset \} \}, \{ \emptyset, \{ \emptyset \} \} \}, \\
    & \{ \{ \emptyset \}, \{ \{ \emptyset \} \}, \{ \emptyset, \{ \emptyset \} \} \}, \\
    & \{ \emptyset, \{ \emptyset \}, \{ \{ \emptyset \} \},
    \{ \emptyset, \{ \emptyset \} \} \} \\
    \}
    \end{align*}
\end{document}
\documentclass{report}

\title{Solutions Manual for Enderton \emph{Elements of Set Theory}}
\author{Robin Adams}

\usepackage{amsmath}
\usepackage{amssymb}

\begin{document}
    \maketitle
    \tableofcontents

    \chapter{Chapter 1 --- Introduction}

    \section{Baby Set Theory}

    \paragraph{Exercise 1}
    \begin{itemize}
        \item $\{ \emptyset \} \in \{ \emptyset, \{ \emptyset \} \}$ --- true
        \item $\{ \emptyset \} \subseteq \{ \emptyset, \{ \emptyset \} \}$ --- true
        \item $\{ \emptyset \} \in \{ \emptyset, \{ \{ \emptyset \} \} \}$ --- false
        \item $\{ \emptyset \} \subseteq \{ \emptyset, \{ \{ \emptyset \} \} \}$ --- true
        \item $\{ \{ \emptyset \} \} \in \{ \emptyset, \{ \emptyset \} \}$ --- false
        \item $\{ \{ \emptyset \} \} \subseteq \{ \emptyset, \{ \emptyset \} \}$ --- true
        \item $\{ \{ \emptyset \} \} \in \{ \emptyset, \{ \{ \emptyset \} \} \}$ --- true
        \item $\{ \{ \emptyset \} \} \subseteq \{ \emptyset, \{ \{ \emptyset \} \} \}$ --- false
        \item $\{ \{ \emptyset \} \} \in \{ \emptyset, \{ \emptyset, \{ \emptyset \} \} \}$ --- false
        \item $\{ \{ \emptyset \} \} \subseteq \{ \emptyset, \{ \emptyset, \{ \emptyset \} \} \}$ --- false
    \end{itemize}

    \paragraph{Exercise 2}
    We have $\emptyset \neq \{ \emptyset \}$ because $\{ \emptyset \}$ has an element (namely $\emptyset$)
    while $\emptyset$ has no elements.

    We have $\emptyset \neq \{ \{ \emptyset \} \}$ because $\{ \{ \emptyset \} \}$ has an element (namely
    $\{ \emptyset \}$) while $\emptyset$ has no elements.

    We have $\{ \emptyset \} \neq \{ \{ \emptyset \} \}$ because $\emptyset \in \{ \emptyset \}$ but
    $\emptyset \notin \{ \{ \emptyset \} \}$. This last fact is true because $\emptyset \neq \{ \emptyset \}$
    as we proved in the first paragraph.

    \paragraph{Exercise 3}
    Assume $B \subseteq C$. Let $A \in \mathcal{P} B$; we must show that $A \in \mathcal{P} C$.

    We have $A \subseteq B$ (since $A \in \mathcal{P} B$) and $B \subseteq C$. From this it follows that
    $A \subseteq C$ (every element of $A$ is an element of $B$; every element of $B$ is an element of $C$;
    therefore every element of $A$ is an element of $C$). Hence $A \in \mathcal{P} C$ as required.

    \paragraph{Exercise 4}
    Since $x \in B$, we have $\{x\} \subseteq B$ and so $\{ x \} \in \mathcal{P} B$.

    Since $x \in B$ and $y \in B$, we have $\{ x,y \} \subseteq B$ and so $\{x,y\} \in \mathcal{P} B$.

    From these two facts, it follows that $\{ \{x\},\{x,y\}\} \subseteq \mathcal{P} B$ and so
    $\{ \{x\},\{x,y\}\} \in \mathcal{P} \mathcal{P} B$.

    \section{Sets --- An Informal View}

    \paragraph{Exercise 5}
    We have
    \begin{align*}
        V_0 & = A \\
        V_1 & = V_0 \cup \mathcal{P} V_0 \\
        & = A \cup \mathcal{P} A \\
        V_2 & = V_1 \cup \mathcal{P} V_1 \\
        & = \{ \emptyset, \{ \emptyset \} \} \\
        V_3 & = \mathcal{P} V_2 \\
        & = \{ \emptyset, \{ \emptyset \}, \{ \{ \emptyset \} \},
        \{ \emptyset, \{ \emptyset \} \} \}
    \end{align*}

    We have $\emptyset \subseteq V_0$ and so $\emptyset \in V_1$. Therefore $\{ \emptyset \} \subseteq V_1$
    and so $\{ \emptyset \} \in V_2$. Hence
    $\{ \{ \emptyset \} \} \subseteq V_2$.

    We also have $\{ \{ \emptyset \} \} \nsubseteq V_0$ because $\{ \emptyset \}$ is not an atom,
    and $\{ \{ \emptyset \} \} \nsubseteq V_1$ since $\{ \emptyset \} \notin V_1$ because $\emptyset$ is not an atom.

    Thus the rank of $\{ \{ \emptyset \} \}$ is 2.

    Likewise we have $\emptyset$ and $\{ \emptyset \}$ are both subsets of $V_1$, hence
    \[ \emptyset \in V_2, \qquad \{ \emptyset \} \in V_2 \]
    Thus $\emptyset, \{ \emptyset \}, \{ \emptyset, \{ \emptyset \} \}$ are all subsets of $V_2$,
    hence elements of $V_3$. Therefore,
    \[ \{ \emptyset, \{ \emptyset \}, \{ \emptyset, \{ \emptyset \} \} \} \subseteq V_3 \]

    Now, $\{ \emptyset, \{ \emptyset \}, \{ \emptyset, \{ \emptyset \} \} \}$ is not a subset of $V_0$
    (because $\emptyset$ is not an atom.) It is not a subset of $V_1$ ($\{ \emptyset \} \notin V_1$
    because $\emptyset$ is not an atom.) It is not a subset of $V_2$ (we have
    $\{ \emptyset, \{ \emptyset \} \} \notin V_2$ since $\{ \emptyset \} \notin V_1$).

    Therefore the rank of
    $\{ \emptyset, \{ \emptyset \}, \{ \emptyset, \{ \emptyset \} \} \}$ is 3.
    
    \paragraph{Exercise 6}

    \begin{align*}
        V_1 & = V_0 \cup \mathcal{P} V_0 \\
        & = A \cup \mathcal{P} V_0 & (\text{since } V_0 = A) \\
        V_2 & = V_1 \cup \mathcal{P} V_1 \\
        & = A \cup \mathcal{P} V_0 \cup \mathcal{P} V_1 \\
        & = A \cup \mathcal{P} V_1 & (\text{since } \mathcal{P} V_0 \subseteq \mathcal{P} V_1 \text{ by Exercise 3}) \\
        V_3 & = V_2 \cup \mathcal{P} V_2 \\
        & = A \cup \mathcal{P} V_1 \cup \mathcal{P} V_2 \\
        & = A \cup \mathcal{P} V_2 & (\text{since } \mathcal{P} V_1 \subseteq \mathcal{P} V_2 \text{ by Exercise 3}) \\
        V_4 & = V_3 \cup \mathcal{P} V_3 \\
        & = A \cup \mathcal{P} V_2 \cup \mathcal{P} V_3 \\
        & = A \cup \mathcal{P} V_3 & (\text{since } \mathcal{P} V_2 \subseteq \mathcal{P} V_3 \text{ by Exercise 3}) \\        
    \end{align*}

    \paragraph{Exercise 7}
    In Exercise 5 we calculated $V_3 = \{ \emptyset, \{ \emptyset \}, \{ \{ \emptyset \} \},
    \{ \emptyset, \{ \emptyset \} \} \}$

    Hence
    \begin{align*}
    V_4 = & \mathcal{P} V_3 \\
    = \{ & \emptyset, \\
    & \{ \emptyset \}, \\
    & \{ \{ \emptyset \} \}, \\
    & \{ \{ \{ \emptyset \} \} \}, \\
    & \{ \{ \emptyset, \{ \emptyset \} \} \}, \\
    & \{ \emptyset, \{ \emptyset \} \}, \\
    & \{ \emptyset, \{ \{ \emptyset \} \} \}, \\
    & \{ \emptyset, \{ \emptyset, \{ \emptyset \} \} \}, \\
    & \{ \{ \emptyset \}, \{ \{ \emptyset \} \} \}, \\
    & \{ \{ \emptyset \}, \{ \emptyset, \{ \emptyset \} \} \}, \\
    & \{ \{ \{ \emptyset \} \}, \{ \emptyset, \{ \emptyset \} \} \}, \\
    & \{ \emptyset, \{ \emptyset \}, \{ \{ \emptyset \} \} \}, \\
    & \{ \emptyset, \{ \emptyset \}, \{ \emptyset, \{ \emptyset \} \} \}, \\
    & \{ \emptyset, \{ \{ \emptyset \} \}, \{ \emptyset, \{ \emptyset \} \} \}, \\
    & \{ \{ \emptyset \}, \{ \{ \emptyset \} \}, \{ \emptyset, \{ \emptyset \} \} \}, \\
    & \{ \emptyset, \{ \emptyset \}, \{ \{ \emptyset \} \},
    \{ \emptyset, \{ \emptyset \} \} \} \\
    \}
    \end{align*}

    \chapter{Chapter 2 --- Axioms and Operations}

    \section{Arbitrary Unions and Intersections}

    \paragraph{Exercise 1}
    $A \cap B \cap C$ is the set of all integers that are divisible by 4, 9 and 10,
    which is the same as the set of all integers that are divisible by 180.

    \paragraph{Exercise 2}
    Take $A = \emptyset$ and $B = \{ \emptyset \}$. Then $\bigcup A = \bigcup B = \emptyset$
    but $A \neq B$. (There are many other possible answers.)

    \paragraph{Exercise 3}
    Let $b \in A$. We must show that $b \subseteq \bigcup A$.

    Let $x$ be any element of $b$. We must show that $x \in \bigcup A$. We know that $x \in b$ and $b \in A$,
    and so $x \in \bigcup A$ by the definition of $\bigcup A$.

    \paragraph{Exercise 4}
    Suppose $A \subseteq B$. Let $x \in \bigcup A$. We must show that $x \in \bigcup B$.

    Pick an element $a \in A$ such that $x \in a$. Then $a \in B$ because $A \subseteq B$. Since we know
    $x \in a$ and $a \in B$, we know that $x \in \bigcup B$.

    \paragraph{Exercise 5}
    Assume that every member of $\mathcal{A}$ is a subset of $B$. Let $x \in \bigcup \mathcal{A}$.
    We must show that $x \in B$.

    Pick $A \in \mathcal{A}$ such that $x \in A$. By our assumption, we have $A \subseteq B$. Since $x \in A$
    and $A \subseteq B$, we have $x \in B$ as required.

    \paragraph{Exercise 6}
    \subparagraph{(a)}We will show that $\bigcup \mathcal{P} A \subseteq A$ and $A \subseteq \bigcup \mathcal{P} A$.

    To show $\bigcup \mathcal{P} A \subseteq A$: This follows from Exercise 5, since every member of $\mathcal{P} A$
    is a subset of $A$.

    To show $A \subseteq \bigcup \mathcal{P} A$: Let $a \in A$. Then we have $a \in \{a\}$ and 
    $\{a\} \in \mathcal{P} A$ so $a \in \bigcup \mathcal{P} A$.

    \subparagraph{(b)}
    To show $A \subseteq \mathcal{P} \bigcup A$: This holds because every element of $A$ is a subset of
    $\bigcup A$, as we proved is Exercise 3.

    Equality holds if and only if $A = \mathcal{P} X$ for some set $X$.

    Proof: If $A = \mathcal{P} \bigcup A$ then of course $A = \mathcal{P} X$ for some $X$.

    Conversely, if $A = \mathcal{P} X$, then we have
    \begin{align*}
        \mathcal{P} \bigcup A & = \mathcal{P} \bigcup \mathcal{P} X \\
        & = \mathcal{P} X & (\text{by part (a)}) \\
        & = A
    \end{align*}

    \paragraph{Exercise 7}

    \subparagraph{(a)}
    For any set $X$,
    \begin{align*}
        & X \in \mathcal{P} A \cap \mathcal{P} B \\
        \Leftrightarrow & X \subseteq A \text{ and } X \subseteq B \\
        \Leftrightarrow & \text{Every member of $X$ is a member of $A$ and a member of $B$}  \\
        \Leftrightarrow & X \subseteq A \cap B \\
        \Leftrightarrow & X \in \mathcal{P} (A \cap B)
    \end{align*}

    \subparagraph{(b)}
    Let $X \in \mathcal{P} A \cup \mathcal{P} B$. Then either $X \in \mathcal{P} A$ or $X \in \mathcal{P} B$
    (or both). If $X \in \mathcal{P} A$, then we have $X \subseteq A$ and so $X \subseteq A \cup B$ (because
    $A \subseteq A \cup B$). Similarly if $X \in \mathcal{P} B$ then we have $X \subseteq A \cup B$. So in
    either case $X \subseteq A \cup B$, hence $X \in \mathcal{P} (A \cup B)$.

    Equality holds if and only if either $A \subseteq B$ or $B \subseteq A$.

    Proof: Suppose $A \subseteq B$. Then $\mathcal{P} A \subseteq \mathcal{P} B$ (Chapter 1 Exercise 3)
    and so $\mathcal{P} A \cup \mathcal{P} B = \mathcal{P} B$. Also $A \cup B = B$ so $\mathcal{P} (A \cup B)
    = \mathcal{P} B$. Thus $\mathcal{P} A \cup \mathcal{P} B$ and $\mathcal{P} (A \cup B)$ are equal.

    Similarly if $B \subseteq A$ then $\mathcal{P} A \cup \mathcal{P} B = \mathcal{P} (A \cup B)$.

    Conversely, suppose $\mathcal{P} A \cup \mathcal{P} B = \mathcal{P} (A \cup B)$. We have $A \cup B
    \in \mathcal{P} (A \cup B)$, so $A \cup B \in \mathcal{P} A \cup \mathcal{P} B$. If $A \cup B \in \mathcal{P}
    A$, then we have $B \subseteq A \cup B \subseteq A$. And if $A \cup B \in \mathcal{P} B$, then we have
    $A \subseteq A \cup B \subseteq B$.

    \paragraph{Exercise 8}
    If $A$ is a set such that every singleton belongs to $A$, then every set belongs to $\bigcup A$,
    contradicting Theorem 2A.

    \paragraph{Exercise 9}
    Let $a = \{ \emptyset \}$ and $B = \{ \{ \emptyset \} \}$. Then $a \in B$ but $\mathcal{P} a$
    is not a subset of $B$ because $\emptyset \in \mathcal{P} a$ and $\emptyset \notin B$.

    \paragraph{Exercise 10}
    We must show that $\mathcal{P} a \subseteq \mathcal{P} \bigcup B$. So let $X \in \mathcal{P} a$.
    Then $X \subseteq a$; we must show that $X \subseteq \bigcup B$.

    Let $x \in X$; we must show  that $x \in \bigcup B$. We have $x \in a$ (because $x \in X$ and $X \subseteq a$)
    and $a \in B$, hence $x \in \bigcup B$ as required.
\end{document}
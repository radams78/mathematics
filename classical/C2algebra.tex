\documentclass{article}

\title{C2 Algebra}
\author{Robin Adams}

\usepackage{amsmath}
\usepackage{amssymb}
\usepackage{amsthm}
\let\proof\relax
\let\endproof\relax
\let\qed\relax
\usepackage{pf2}
\usepackage[all]{xy}

\newtheorem{axiom}{Axiom}[section]
\newtheorem{axs}[axiom]{Axiom Schema}
\newtheorem{lemma}[axiom]{Lemma}
\newtheorem{proposition}[axiom]{Proposition}
\newtheorem{props}[axiom]{Proposition Schema}
\newtheorem{theorem}[axiom]{Theorem}
\newtheorem{corollary}{Corollary}[axiom]
\theoremstyle{definition}
\newtheorem{definition}[axiom]{Definition}
\newtheorem{example}[axiom]{Example}

\newcommand{\inv}[1]{\ensuremath{{#1}^{-1}}}

\begin{document}
    \maketitle

    \section{Groups}

    \begin{definition}[Group]
        A \emph{group} is a triple $(G, \cdot, e)$ where $G$ is a set, $\cdot$ is a binary operation on $G$,
        and $e \in G$, such that:
        \begin{enumerate}
            \item $\cdot$ is associative.
            \item $\forall x \in G. xe = ex = x$
            \item $\forall x \in G. \exists y \in G. xy = yx = e$
        \end{enumerate}
    \end{definition}

    \begin{lemma}
        The integers $\mathbb{Z}$ form a group under $+$ and $0$.
    \end{lemma}

    \begin{proof}
        \pf\ Easy. \qed
    \end{proof}

    \begin{lemma}
        In any group, inverses are unique.
    \end{lemma}

    \begin{proof}
        \pf\ Suppose $y$ and $z$ are inverses to $x$. Then
        \[ y = ey = zxy = ze = z \]
        \qed
    \end{proof}

    \begin{definition}
        We write $x^{-1}$ for the inverse of $x$.
    \end{definition}

    \section{Abelian Groups}

    \begin{definition}[Abelian Group]
        A group $(G, +, 0)$ is \emph{Abelian} iff $+$ is commutative.

        When using additive notation (i.e. the symbols $+$ and $0$) for a group,
        we write $-y$ for the inverse of $y$, and $x-y$ for $x+(-y)$.
    \end{definition}

    \begin{lemma}
        The integers $\mathbb{Z}$ are Abelian.
    \end{lemma}

    \begin{proof}
        \pf\ Easy. \qed
    \end{proof}

    \begin{lemma}
        The rationals $\mathbb{Q}$ form an Abelian group under $+$.
    \end{lemma}

    \begin{proof}
        \pf\ Easy.
    \end{proof}

    \begin{lemma}
        The non-zero rationals form an Abelian group under multiplication.
    \end{lemma}

    \begin{proof}
        \pf\ Easy. \qed
    \end{proof}

    \section{Ring Theory}

    \begin{definition}[Rng]
        A \emph{rng} is a quintuple $(R, +, \cdot, 0)$ consisting of a set $R$,
        binary operations $+$ and $\cdot$ on $R$, and element $0 \in R$ such that:
        \begin{enumerate}
            \item $(R, +, 0)$ is an Abelian group.
            \item The operation $\cdot$ is associative, and distributive over $+$.
        \end{enumerate}
    \end{definition}

    \begin{proposition}
        In any rng we have $x0 = 0$.
    \end{proposition}

    \begin{proof}
        \pf\ $x0 = x(0+0) = x0+x0$ and also $x0 = x0+0$.
        The result follows by the cancellation law. \qed
    \end{proof}

    \begin{proposition}
        In any rng we have $-(xy) = (-x)y = x(-y)$.
    \end{proposition}

    \begin{proof}
        \pf\ The result $-(xy) = (-x)y$ holds because
        \[ xy + (-x)y = (x+(-x))y = 0y = 0 \enspace . \]
        We prove $-(xy) = x(-y)$ similarly. \qed
    \end{proof}

    \begin{corollary}
        In any rng, $(-x)(-y) = xy$.
    \end{corollary}

    \begin{definition}[Ring]
        A \emph{ring} consists of a rng $R$ and an element $1 \in R$, the \emph{unit element},
        such that $\forall x \in R. x1 = 1x = x$.
    \end{definition}

    \begin{proposition}
        In a ring $R$, if $0 = 1$ then $R$ has only one element.
    \end{proposition}

    \begin{definition}
        Let $n$ be an integer. In any ring, we write just $n$ for $n1$.
    \end{definition}

    \begin{definition}[Commutative Rng]
        A rng $R$ is \emph{commutative} iff $\forall x,y \in R. xy = yx$.
    \end{definition}

    \begin{definition}[Zero Divisor]
        A \emph{zero divisor} in a rng is an element $x$ such that $x \neq 0$ but there exists $y \neq 0$ such
        that $xy = 0$.
    \end{definition}

    \begin{definition}[Integral Domain]
        An \emph{integral domain} is a commutative ring with no zero divisors.
    \end{definition}

    \begin{example}
        \begin{enumerate}
            \item The trivial ring is an integral domain.
            \item The integers form an integral domain.
            \item The rationals form an integral domain.
        \end{enumerate}
    \end{example}

    \begin{proposition}
        \label{proposition:cancellation}
        Let $R$ be a commutative ring. Then $R$ is an integral domain if and only if, whenever $xy = xz$
        and $x \neq 0$, then $y = z$.
    \end{proposition}

    \begin{definition}[Boolean Ring]
        A \emph{Boolean rng} is a rng $R$ such that $\forall x \in R. x^2 = x$
    \end{definition}

    \begin{example}
        $\mathbb{Z}_2$ is a Boolean rng.
    \end{example}

    \begin{proposition}
        In any Boolean rng we have $x + x = 0$ for all $x$
    \end{proposition}

    \begin{proof}
        \pf\ We have $x = x^2 = (-x)^2 = -x$. \qed
    \end{proof}

    \begin{proposition}
        Every Boolean rng is commutative.
    \end{proposition}

    \begin{proof}
        \pf\ We have
        \begin{align*}
            (x+y)^2 & = x+y \\
            & = x^2 + y^2 \\
            \therefore x^2 + xy + yx + y^2 & = x^2 + y^2 \\
            \therefore xy + yx & = 0 \\
            \therefore xy & = -(yx) \\
            & = yx & \qed
        \end{align*}
    \end{proof}

    \begin{definition}[Characteristic]
        The \emph{characteristic} of an integral domain is the least positive integer $n$ such that $n = 0$,
        or 0 if there is no such $n$.
    \end{definition}

    \begin{example}
        \begin{enumerate}
            \item The characteristic of $\mathbb{Z}$ is 0.
            \item The characteristic of $\mathbb{Z}_n$ is $n$.
        \end{enumerate}
    \end{example}

    \begin{proposition}
        The characteristic of an integral domain is either 0, 1 or a prime.
    \end{proposition}

    \begin{proof}
        \pf
        \step{1}{\pflet{$D$ be any integral domain of characteristic $n > 1$.}}
        \step{3}{\assume{for a contradiction $n = ab$ with $a,b > 1$}}
        \step{4}{$ab = 0$ in $D$}
        \step{5}{$a = 0$ or $b = 0$ in $D$}
        \qedstep
        \begin{proof}
            \pf\ This contradicts the minimality of $n$.
        \end{proof}
        \qed
    \end{proof}

    \begin{theorem}
        An integral domain $D$ has characteristic 0 iff $\{ n1 : n \in \mathbb{N} \}$ is infinite.
    \end{theorem}

    \begin{proof}
        \pf
        \step{1}{If $D$ has characteristic $p > 0$ then $\{ n1 : n \in \mathbb{N} \}$ is finite.}
        \begin{proof}
            \step{a}{\assume{the characteristic of $D$ is $p > 0$} \prove{For all $n \in \mathbb{N}$
            there exists $k < p$ such that $n1 = k1$ in $D$}}
            \step{b}{\pflet{$n \in \mathbb{N}$}}
            \step{c}{\pflet{$q$, $r$ be the integers such that $n = qp + r$ with $0 \leq r < p$}}
            \step{d}{$n1 = r1$}
            \begin{proof}
                \pf
                \begin{align*}
                    n1 & = q(p1) + r1 \\
                    & = q0 + r1 \\
                    & = r1
                \end{align*}
            \end{proof}
        \end{proof}
        \step{2}{If $\{ n1 : n \in \mathbb{N} \}$ is finite then $D$ has non-zero characteristic.}
        \begin{proof}
            \step{a}{\assume{$\{ n1 : n \in \mathbb{N} \}$ is finite.}}
            \step{b}{\pick\ a positive integer $p$ such that $p1 = k1$ for some non-negative $k < p$}
            \step{c}{$(p-k)1 = 0$ and $p-k > 0$}
        \end{proof}
        \qed
    \end{proof}

    \begin{proposition}
        For any integral domain $D$, the set $\{ n1 : n \in \mathbb{Z} \}$ is a subdomain.
    \end{proposition}

    \begin{proposition}
        For any integral domain $D$ of characteristic 0, the mapping that sends $n$ to $n1$ is an
        embedding of $\mathbb{Z}$ in $D$.
    \end{proposition}

    \begin{corollary}
        The integers are the unique integral domain $D$ up to isomorphism with characteristic 0 such that
        $D$ has no proper subdomains.
    \end{corollary}

    \section{Polynomials}

    \begin{definition}[Polynomial]
        Let $D$ be an integral domain. The set $D[x]$ of \emph{polynomials} over $D$ is the set of sequences
        in $D$ that are eventually zero. We write the sequence $(a_n)$ as 
        $a_0 + a_1 x + \cdots +
        a_m x^m$ if $a_n = 0$ for all $n > m$. The element $a_i$ is called the $i$th \emph{coefficient}, or the \emph{coefficient} of
        $x^i$.
    \end{definition}

    \begin{definition}[Degree]
        The \emph{degree} of a non-zero polynomial $p$ is the largest integer $n$ such that the coefficient of
        $x^n$ is non-zero. This coefficient is the \emph{leading coefficient} of $p$.
    \end{definition}

    \begin{definition}[Addition]
        Addition of polynomials is defined by: $(a_n) + (b_n) = (a_n + b_n)$.
    \end{definition}

    \begin{definition}[Multiplication]
        Multiplication of polynomials is defined by: $(\sum_n a_n x^n) (\sum_n b_n x^n) = \sum_n (\sum_{m=0}^n a_m b_{n-m}) x^n$.
    \end{definition}

    \begin{theorem}
        Under these operations, $D[x]$ is an integral domain.
    \end{theorem}

    \section{Ordered Integral Domains}

    \begin{definition}[Ordered Integral Domain]
        An \emph{ordered integral domain} is an integral domain $D$ with a linear order $<$ such that:
        \begin{itemize}
            \item Whenever $x < y$ then $x + z < y + z$.
            \item Whenever $x < y$ and $0 < z$ then $xz < yz$.
        \end{itemize}
    \end{definition}

    \begin{proposition}
        In an ordered integral domain, if $x < y$ and $z < 0$ then $yz < xz$.
    \end{proposition}

    \begin{proposition}
        $x < y$ iff $-y < -x$.
    \end{proposition}

    \begin{proposition}
        Any subdomain of an ordered integral domain is an ordered integral domain under the restriction of
        $<$.
    \end{proposition}

    \begin{definition}[Positive]
        In an integral domain, we say an element $a$ is \emph{positive} iff $0 < a$ and \emph{negative}
        iff $a < 0$.
    \end{definition}

    \begin{proposition}
        $x < y$ iff $y - x$ is positive.
    \end{proposition}

    \begin{proposition}
        $x < y$ iff $x - y$ is negative.
    \end{proposition}

    \begin{proposition}
        $x$ is positive iff $-x$ is negative.
    \end{proposition}

    \begin{proposition}
        $x$ is negative iff $-x$ is positive.
    \end{proposition}

    \begin{proposition}
        The sum of two positive elements is positive.
    \end{proposition}

    \begin{proposition}
        The product of two positive elements is positive.
    \end{proposition}

    \begin{proposition}
        The product of two negative elements is positive.
    \end{proposition}

    \begin{proposition}
        The product of a positive and a negative element is negative.
    \end{proposition}

    \begin{proposition}
        If $x \neq 0$ then $x^2$ is positive.
    \end{proposition}

    \begin{proposition}
        $x^2$ is always non-negative.
    \end{proposition}

    \begin{proposition}
        $0 < 1$
    \end{proposition}

    \begin{proposition}
        $-1 < 0$
    \end{proposition}

    \begin{theorem}
        Let $R$ be an integral domain and $P \subseteq R$ be a set such that:
        \begin{itemize}
            \item $0 \notin P$
            \item For all $x \in R$ we have $x \in P$ or $x = 0$ or $-x \in P$
            \item For all $x,y \in P$ we have $x+y \in P$
            \item For all $x,y \in P$ we have $xy \in P$
        \end{itemize}
        Define $<$ on $R$ by $x < y$ iff $y - x \in P$. Then $R$ is an ordered integral domain under $<$
        with $P$ the set of positive elements.
    \end{theorem}

    \begin{definition}[Absolute Value]
        In any ordered integral domain, define
        \[ |x| = \begin{cases}
            x & \text{if $0 \leq x$} \\
            -x & \text{if $x < 0$}
        \end{cases} \]
    \end{definition}

    \begin{proposition}
        $|x|$ is always non-negative.
    \end{proposition}

    \begin{proposition}
        $|x| = 0$ iff $x = 0$
    \end{proposition}

    \begin{proposition}
        $|-x| = |x|$
    \end{proposition}

    \begin{proposition}
        $|x-y| = |y-x|$
    \end{proposition}

    \begin{proposition}
        $|xy| = |x||y|$
    \end{proposition}

    \begin{proposition}
        $-|x| \leq x \leq |x|$
    \end{proposition}

    \begin{proposition}
        $|x| < u$ iff $-u < x < u$
    \end{proposition}

    \begin{proposition}
        $|x| \leq u$ iff $-u \leq x \leq u$
    \end{proposition}

    \begin{proposition}[Triangle Inequality]
        $|x + y| \leq |x| + |y|$
    \end{proposition}

    \begin{proposition}
        $||x| - |y|| \leq |x-y|$
    \end{proposition}

    \begin{proposition}
        Any ordered integral domain has characteristic 0.
    \end{proposition}

    \begin{proof}
        \pf\ For any positive integer $n$ we have $0 < n$ and so $n \neq 0$. \qed
    \end{proof}

    \begin{theorem}
        Let $D$ be an ordered integral domain. Then the following are equivalent.
        \begin{enumerate}
            \item $D \cong \mathbb{Z}$
            \item The set of positive elements of $D$ is $\{ n1 : n \in \mathbb{Z}^+ \}$
            \item The set of positive elements of $D$ is well-ordered by $<$.
        \end{enumerate}
    \end{theorem}

    \begin{theorem}
        Let $D$ be an ordered integral domain. Then $D[x]$ is an ordered integral domain under:
        $p(x) < q(x)$ iff $q(x) - p(x)$ is positive, where a polynomial is positive iff its leading
        coefficient is positive.
    \end{theorem}
    
    \begin{definition}[Monic Polynomial]
        A polynomial is \emph{monic} iff its leading coefficient is 1.
    \end{definition}

    \begin{theorem}
        Let $D$ be an integral domain. Let $f, g \in D[x]$ with $f$ a monic polynomial of degree $\geq 1$.
        Then there exist unique polynomials $q, r \in D[x]$ such that $g = fq + r$ and either $r = 0$
        or $\deg r < \deg f$.
    \end{theorem}

    \begin{proof}
        \pf
        \step{1}{\pflet{$f \in D[x]$ be a monic polynomial of degree $k \geq 1$}}
        \step{2}{$0$ and $0$ are the unique polynomials such that $0 = f0 + 0$}
        \step{3}{For any $n \in \mathbb{N}$ and polynomial $g$ of degree $n$, there exist polynomials
        $q, r \in D[x]$ such that $g = fq + r$ and either $r = 0$ or $\deg r < \deg f$}
        \begin{proof}
            \step{a}{For any ppolynomial $g$ of degree $< k$, there exist polynomials
            $q, r \in D[x]$ such that $g = fq + r$ and either $r = 0$ or $\deg r < \deg f$}
            \begin{proof}
                \pf\ Take $q = 0$ and $r = g$.
            \end{proof}
            \step{b}{Let $n \in \mathbb{N}$ with $k \leq n$. Assume for any polynomial $g$ of degree $\leq n$,
            there exist polynomials
            $q, r \in D[x]$ such that $g = fq + r$ and either $r = 0$ or $\deg r < \deg f$.
            Then for any polynomial $g$ of degree $n+1$, there exist polynomials
            $q, r \in D[x]$ such that $g = fq + r$ and either $r = 0$ or $\deg r < \deg f$}
            \begin{proof}
                \step{i}{\pflet{$n \in \mathbb{N}$}}
                \step{ii}{\assume{For any polynomial $g$ of degree $n$, there exist polynomials
                $q, r \in D[x]$ such that $g = fq + r$ and either $r = 0$ or $\deg r < \deg f$.}}
                \step{iii}{\pflet{$g$ be a polynomial of degree $n+1$}}
                \step{iv}{\pflet{$a_{n+1}$ be the leading coefficient of $g$}}
                \step{iv}{\pflet{$h(x) = g(x) - a_{n+1} x^{n+1-k} f(x)$}}
                \step{v}{Either $h = 0$ or $\deg h \leq n$}
                \step{vi}{\pick\ polynomials $q$, $r$ with $h = fq + r$ and either $r = 0$ or $\deg r < k$}
                \step{vii}{$g(x) = f(x) (q(x) + a_{n+1} x^{n+1-k}) + r(x)$}
            \end{proof}
        \end{proof}
        \step{4}{If $fq + r = fq' + r'$; either $r = 0$ or $\deg r < \deg f$; and either $r'= 0$
        or $\deg r' < \deg f$; then $q = q'$ and $r = r'$}
        \begin{proof}
            \step{a}{$f(q - q') = r' - r$ and $r' - r$ is either 0 or has degree $< \deg f$}
            \step{b}{$q - q' = 0$}
            \step{c}{$r = r'$}
        \end{proof}
        \qed
    \end{proof}

    \begin{definition}[Polynomial Function]
        Given $f(x) \in D[x]$ and $a \in D$, define $f(a) \in D$ in the obvious way.
    \end{definition}

    \begin{definition}[Root]
        A \emph{root} of a polynomial $p(x) \in D[x]$ is an element $a \in D$ such that $p(a) = 0$.
    \end{definition}

    \begin{theorem}
        Let $p(x) \in D[x]$ and $a \in D$. Then $p(a) = 0$ iff there exists $q(x) \in D[x]$ such that
        $p(x) = q(x) (x - a)$.
    \end{theorem}

    \begin{proof}
        \pf
        \step{1}{If $p(x) = q(x)(x-a)$ then $p(a) = 0$}
        \step{2}{If $p(a) = 0$ then there exists $q$ such that $p(x) = q(x)(x-a)$}
        \begin{proof}
            \step{a}{\assume{$p(a) = 0$}}
            \step{b}{\pflet{$q$ and $r$ be the polynomials such that $p(x) = q(x)(x-a) + r(x)$ where $r = 0$
            or $\deg r < 1$}}
            \step{c}{\pflet{$r(x) = c$, a constant}}
            \step{d}{$c = 0$}
            \begin{proof}
                \pf
                \begin{align*}
                    p(a) & = 0 \\
                    \therefore q(a)(a-a) + c & = 0 \\
                    \therefore c & = 0
                \end{align*}
            \end{proof}
            \step{e}{$p(x) = q(x)(x-a)$}
        \end{proof}
        \qed
    \end{proof}

    \begin{corollary}
        A polynomial of degree $n$ has at most $n$ distinct roots.
    \end{corollary}

    \begin{corollary}
        Let $D$ be an infinite integral domain and $f, g \in D[x]$. Then $f = g$ iff $f$ and $g$ determine
        the same function $D \rightarrow D$.
    \end{corollary}

    \begin{proof}
        \pf\ If $f$ and $g$ determine the same function then $f - g$ has infinitely many roots, hence
        $f - g = 0$. \qed
    \end{proof}

    \begin{theorem}[Division Theorem]
        Let $a$ and $b$ be integers, $a > 1$. Then there exist unique integers $q$ and $r$ such that
        $b = qa + r$ and $0 \leq r < a$.
    \end{theorem}

    \begin{proof}
        \pf\ For existence, prove the case $b \geq 0$ by induction on $b$. The case $b < 0$ follows.

        For uniqueness, if $qa + r = q'a + r'$ then $a | r - r'$ and $-a < r - r' < a$, hence $r - r' = 0$.
        So $r = r'$ and $q = q'$. \qed
    \end{proof}

    \begin{definition}[Divisibility]
        We say $a$ \emph{divides} $b$, $a \mid b$, iff there exists $c$ such that $b = ac$.
    \end{definition}

    \begin{proposition}
        For every integer $a$ we have $a \mid 0$.
    \end{proposition}

    \begin{proposition}
        For every integer $a$ we have $1 \mid a$.
    \end{proposition}

    \begin{proposition}
        For every integer $a$ we have $a \mid a$.
    \end{proposition}

    \begin{proposition}
        If $a \mid b$ and $b \mid c$ then $a \mid c$.
    \end{proposition}

    \begin{proposition}
        If $a \mid c$ and $c \neq 0$ the $|a| \leq |c|$.
    \end{proposition}

    \begin{proposition}
        If $0 \mid a$ then $a = 0$.
    \end{proposition}

    \begin{proposition}
        If $a \mid b$ and $b \mid a$ then $a = b$ or $a = -b$.
    \end{proposition}

    \begin{proposition}
        $a \mid ab$
    \end{proposition}

    \begin{proposition}
        If $a \mid b$ and $a \mid c$ then $a \mid b + c$.
    \end{proposition}

    \begin{proposition}
        If $a \mid b$ and $a \mid c$ then $a \mid b - c$.
    \end{proposition}
    
    \begin{proposition}
        If $a \mid 1$ then $a = 1$ or $a = -1$.
    \end{proposition}

    \begin{definition}[Greatest Common Divisor]
        The integer $d$ is the \emph{greatest common divisor} of $a$ and $b$ iff $d$ is non-negative,
        $d \mid a$, $d \mid b$,
        and whenever $x \mid a$ and $x \mid b$ then $d \mid x$.
    \end{definition}

    \begin{proposition}
        Two integers have at most one gcd.
    \end{proposition}

    \begin{theorem}
        Let $a$ and $b$ be integers that are not both 0. Then there exist integers $x$ and $y$ such that
        $xa + yb$ is the greatest common divisor of $a$ and $b$.
    \end{theorem}

    \begin{proof}
        \pf\ Take the least positive member of $\{ xa + yb : x,y \in \mathbb{Z} \}$. \qed
    \end{proof}

    \begin{definition}[Relatively Prime]
        Two integers $a$ and $b$ are \emph{relatively prime} iff their gcd is 1.
    \end{definition}

    \begin{definition}[Prime]
        An integer $p$ is \emph{prime} iff $p > 1$ and the only divisors of $p$ are 1 and $p$.
        
        An integer $a$ is \emph{composite} iff $a > 1$ and $a$ is not prime.
    \end{definition}

    \begin{proposition}
        Every integer greater than 1 is divisible by a prime.
    \end{proposition}

    \begin{theorem}
        There are infinitely many primes.
    \end{theorem}

    \begin{proposition}
        If $p$ is prime and $p \mid ab$ then $p \mid a$ or $p \mid b$.
    \end{proposition}

    \begin{theorem}[Fundamental Theorem of Arithmetic]
        Every integer $> 1$ is the product of a unique multiset of primes.
    \end{theorem}

    \section{Integers Modulo $n$}

    \begin{definition}[Congruence]
        Two integers $a$ and $b$ are \emph{congruent} modulo $n$, $a \equiv b \mod n$, iff $n \mid a - b$.
    \end{definition}

    \begin{proposition}
        Congruence modulo $n$ is an equivalence relation.
    \end{proposition}

    \begin{proposition}
        If $a \equiv b \mod n$ and $c \equiv d \mod n$ then $a + c \equiv b + d \mod n$.
    \end{proposition}

    \begin{proposition}
        If $a \equiv b \mod n$ then $-a \equiv -b \mod n$.
    \end{proposition}

    \begin{proposition}
        If $a \equiv b \mod n$ and $c \equiv d \mod n$ then $ac \equiv bd \mod n$.
    \end{proposition}

    \begin{definition}
        The equivalence classes with respect to congruence modulo $n$ are called \emph{residue classes modulo $n$}.
    \end{definition}

    \begin{definition}
        The set of \emph{integers modulo $n$}, $\mathbb{Z}_n$, is the quotient of $\mathbb{Z}$ by 
        congruence modulo $n$.
    \end{definition}

    \begin{proposition}
        If $n > 0$ then $|\mathbb{Z}_n| = n$.
    \end{proposition}

    \begin{proposition}
        $\mathbb{Z}_n$ is a commutative ring.
    \end{proposition}

    \begin{proposition}
        $\mathbb{Z}_n$ is an integral domain if and only if $n$ is prime.
    \end{proposition}
    \section{Field Theory}

    \begin{definition}[Field]
        A \emph{field} is a non-trivial integral domain such that every non-zero element has a multiplicative
        inverse.
    \end{definition}

    \begin{definition}[Field of Fractions]
        Let $R$ be a non-trivial integral domain. The \emph{field of fractions} or \emph{quotient field} of $R$ is $(R \times (R - \{ 0 \})) / \sim$,
        where $(a,b) \sim (c,d)$ iff $ad = bc$, under the following operations:
        \begin{align*}
            [(a,b)] + [(c,d)] & = [(ad+bc,bd)] \\
            [(a,b)][(c,d)] & = [(ac,bd)] \\
            0 & = [(0,1)] \\
            1 & = [(1,1)]
        \end{align*}

        We prove that the relation $\sim$ is an equivalence relation, the operations are well-defined, and this
        structure is a field.
    \end{definition}

    \begin{proof}
        \pf
        \step{1}{$\sim$ is an equivalence relation.}
        \begin{proof}
            \step{a}{$\sim$ is reflexive on $R^2$.}
            \begin{proof}
                \step{i}{\pflet{$a,b \in R$ with $b \neq 0$}}
                \step{ii}{$ab = ab$}
                \step{iii}{$(a,b) \sim (a,b)$}
            \end{proof}
            \step{b}{$\sim$ is symmetric.}
            \begin{proof}
                \step{i}{\pflet{$a,b,c,d \in R$ with $b \neq 0$ and $d \neq 0$}}
                \step{ii}{\assume{$(a,b) \sim (c,d)$}}
                \step{iii}{$ad = bc$}
                \step{iv}{$cb = da$}
                \begin{proof}
                    \pf\ Since $R$ is commutative.
                \end{proof}
                \step{v}{$(c,d) \sim (a,b)$}
            \end{proof}
            \step{c}{$\sim$ is transitive.}
            \begin{proof}
                \step{i}{\pflet{$a,b,c,d,e,f \in R$ with $b \neq 0$, $d \neq 0$ and $f \neq 0$}}
                \step{ii}{\assume{$(a,b) \sim (c,d) \sim (e,f)$}}
                \step{iii}{$ad = bc$}
                \step{iv}{$cf = de$}
                \step{v}{$adf = bcf$}
                \step{vi}{$bcf = bde$}
                \step{vii}{$adf = bde$}
                \step{viii}{$af = be$}
                \begin{proof}
                    \pf\ Proposition \ref{proposition:cancellation}.
                \end{proof}
            \end{proof}
        \end{proof}
        \step{2}{Addition is well-defined.}
        \begin{proof}
            \step{o}{If $b \neq 0$ and $d \neq 0$ then $bd \neq 0$}
            \begin{proof}
                \pf\ Since $R$ has no zero-divisors.
            \end{proof}
            \step{a}{\assume{$(a,b) \sim (a',b')$ and $(c,d) \sim (c',d')$}}
            \step{b}{$ab' = a'b$}
            \step{c}{$cd' = c'd$}
            \step{d}{$(ad+bc)b'd' = (a'd'+b'c')bd$}
            \begin{proof}
                \pf
                \begin{align*}
                    (ad+bc)b'd' & = ab'dd' + bb'cd' \\
                    & = a'bdd' + bb'cd' \\
                    & = (a'd'+b'c')bd
                \end{align*}
            \end{proof}
            \step{e}{$(ad+bc,bd) \sim (a'd'+b'c',b'd')$}
        \end{proof}
        \step{3}{Multiplication is well-defined.}
        \begin{proof}
            \step{o}{If $b \neq 0$ and $d \neq 0$ then $bd \neq 0$}
            \step{a}{\assume{$(a,b) \sim (a',b')$ and $(c,d) \sim (c',d')$}}
            \step{b}{$ab'=a'b$}
            \step{c}{$cd'=c'd$}
            \step{d}{$ab'cd' = a'bc'd$}
            \step{e}{$(ac,bd) \sim (a'c',b'd')$}
        \end{proof}
        \step{4}{The axioms of a field are satisfied.}
        \begin{proof}
            \step{a}{Addition is commutative.}
            \begin{proof}
                \pf\ $[(a,b)] + [(c,d)] = [(c,d)] + [(a,b)] = [(ad+bc,bd)]$
            \end{proof}
            \step{b}{Addition is associative.}
            \begin{proof}
                \pf
                \begin{align*}
                    [(a,b)] + ([(c,d)] + [(e,f)])
                    & = [(a,b)] + [(cf+de,df)] \\
                    & = [(adf + bcf + bde, bdf)] \\
                    & = [(ad + bc,bd)] + [(e,f)] \\
                    & = ([(a,b)] + [(c,d)]) + [(e,f)]
                \end{align*}
            \end{proof}
            \step{c}{$x + 0 = x$}
            \begin{proof}
                \pf
                \begin{align*}
                    [(a,b)] + [(0,1)]
                    & = [(a1+b0,b1)] \\
                    & = [(a,b)]
                \end{align*}
            \end{proof}
            \step{d}{For all $x$, there exists $y$ such that $x + y = 0$}
            \begin{proof}
                \pf
                \begin{align*}
                    [(a,b)] + [(-a,b)]
                    & = [(ab-ab,b^2)] \\
                    & = [(0,b^2)] \\
                    & = [(0,1)]
                \end{align*}
                since $(0,b^2) \sim (0,1)$.
            \end{proof}
            \step{e}{Multiplication is commutative.}
            \begin{proof}
                \pf\ $[(a,b)][(c,d)] = [(c,d)][(a,b)] = [(ac,bd)]$
            \end{proof}
            \step{f}{Multiplication is associative.}
            \begin{proof}
                \pf\ $[(a,b)]([(c,d)][(e,f)]) = ([(a,b)][(c,d)])[(e,f)] = [(ace,bdf)]$
            \end{proof}
            \step{g}{$x1 = x$}
            \begin{proof}
                \pf\ $[(a,b)][(1,1)] = [(a1,b1)] = [(a,b)]$
            \end{proof}
            \step{h}{For all $x \neq 0$, there exists $y$ such that $xy = 1$}
            \begin{proof}
                \step{i}{\pflet{$a,b \in R$ with $b \neq 0$ and $(a,b) \nsim (0,1)$}}
                \step{ii}{$a \neq 0$}
                \step{iii}{$[(a,b)][(b,a)] = [(1,1)]$}
                \begin{proof}
                    \pf\ Since $(ab,ab) \sim (1,1)$
                \end{proof}
            \end{proof}
            \step{i}{Multiplication is distributive over addition.}
            \begin{proof}
                \pf
                \begin{align*}
                    [(a,b)]([(c,d)] + [(e,f)])
                    & = [(a,b)][(cf+de,df)] \\
                    & = [(acf+ade,bdf)] \\
                    & = [(abcf+abde,b^2df)] \\
                    & = [(ac,bd)] + [(ae,bf)] \\
                    & = [(a,b)][(c,d)] + [(a,b)][(e,f)]
                \end{align*}
            \end{proof}
            \step{j}{$0 \neq 1$}
            \begin{proof}
                \pf\ Since $(0,1) \nsim (1,1)$
            \end{proof}
        \end{proof}
        \qed
    \end{proof}

    \begin{definition}[Rational Numbers]
        The field of \emph{rational numbers} $\mathbb{Q}$ is the field of fractions of the integers.
    \end{definition}

    \begin{theorem}
        Every finite integral domain with at least two elements is a field.
    \end{theorem}

    \begin{proof}
        \pf\ Let $D$ be a non-trivial finite integral domain. Let $x \in D$. The map that sends $y$
        to $xy$ is an injective map $D \rightarrow D$, hence a bijection by the Pigeonhole Principle.
        Therefore there exists $y$ such that $xy = 1$. \qed
    \end{proof}

    \begin{corollary}
        For any integer $n > 1$, we have $\mathbb{Z}_n$ is a field if and only if $n$ is prime.
    \end{corollary}
    %TODO: The reals form a field.

    %TODO: The complex numbers form a field.

    \begin{theorem}
        Let $a_0$, $a_1$, \ldots, $a_{k-1}$ be integers.
        If $x$ is a rational number such that $x^k + a_{k-1} x^{k-1} + \cdots + a_0 = 0$
        then $x$ is an integer.
    \end{theorem}

    \begin{proof}
        \pf
        \step{1}{\pick\ integers $p$, $q$ such that $x = p / q$ with $\gcd(p,q) = 1$}
        \step{2}{$p^k + a_{k-1} q p^{k-1} + \cdots + a_0 q^k = 0$}
        \step{3}{$q = 1$}
        \begin{proof}
            \step{a}{\assume{for a contradiction $q$ has a prime factor $r$}}
            \step{b}{$r \mid a_{k-1} q p^{k-1} + \cdots + a_0 q^k$}
            \step{c}{$r \mid p^k$}
            \step{d}{$r \mid p$}
            \qedstep
            \begin{proof}
                \pf\ This contradicts the fact that $\gcd(p,q) = 1$.
            \end{proof}
        \end{proof}
        \qed
    \end{proof}

    \begin{corollary}
        There is no rational number $q$ such that $q^2 = 2$.
    \end{corollary}
    
    \subsection{Subfields}

    \begin{definition}[Subfield]
        Let $(E, +_E, \cdot_E)$ and $(F, +_F, \cdot_F)$ be fields. Then $E$ is a \emph{subfield} of $F$
        if and only if $E \subseteq F$, $+_E = +_F \restriction E^2$ and $\cdot_E = \cdot_F \restriction E^2$.
    \end{definition}

    \begin{proposition}
        Let $(F, +_F, \cdot_F)$ be a field and $E \subseteq F$. If $E$ contains a non-zero element and
        is closed under subtraction and division (i.e. whenever $x,y \in E$ and $y \neq 0$ then $x/y \in E$),
        then $(E, +_F \restriction E^2, \cdot_F \restriction E^2)$ is a subfield of $F$.
    \end{proposition}

    \begin{proof}
        \pf
        \step{1}{$1 \in E$}
        \begin{proof}
            \step{a}{\pick\ $a \in E$ with $a \neq 0$}
            \step{b}{$a/a \in E$}
        \end{proof}
        \step{2}{$0 \in E$}
        \begin{proof}
            \pf\ Since $0 = 1 - 1$
        \end{proof}
        \step{3}{$\forall x \in E. -x \in E$}
        \begin{proof}
            \pf\ Since $-x = 0 - x$
        \end{proof}
        \step{4}{$E$ is closed under addition.}
        \begin{proof}
            \pf\ For $x,y \in E$, we have $x+y = x - (-y) \in E$.
        \end{proof}
        \step{5}{$\forall x \in E - \{0\}. \inv{x} \in E$}
        \begin{proof}
            \pf\ Since $\inv{x} = 1 / x$.
        \end{proof}
        \step{6}{$E$ is closed under multiplication.}
        \begin{proof}
            \pf\ For $x,y \in E$, if $y = 0$ then $xy = 0 \in E$. Otherwise $xy = x / \inv{y} \in E$.
        \end{proof}
        \qed
    \end{proof}

    \begin{definition}[Prime Field]
        A field is \emph{prime} iff it contains no proper subfield.
    \end{definition}

    \begin{definition}[Integers and Rational Numbers of a Field]
        In any field $F$, the \emph{integers} of $F$ are the elements of the form $n1$ for $n \in \mathbb{Z}$.

        The \emph{rational numbers} of $F$ are the elements of the form $m/n$ where $m$ and $n$ are integers
        of $F$ with $n \neq 0$.
    \end{definition}

    \begin{proposition}
        For any field $F$, the rational numbers of $F$ form a subfield of $F$ which is minimal (i.e. a subfield
        of every other subfield of $F$).
    \end{proposition}

    \begin{proposition}
        If $F$ has characteristic 0 then the rationals of $F$ are isomorphic to $\mathbb{Q}$.
    \end{proposition}

    \begin{corollary}
        \label{cor:embed_rationals}
        In any ordered field $F$, the rationals of $F$ are isomorphic to $\mathbb{Q}$.
    \end{corollary}

    \begin{theorem}
        The prime fields are $\mathbb{Z}_p$ for $p$ prime and $\mathbb{Q}$.
    \end{theorem}

    \begin{proof}
        \pf
        \step{1}{Every $\mathbb{Z}_p$ is prime.}
        \begin{proof}
            \pf\ If $F$ is a subfield of $\mathbb{Z}_p$ then $F$ contains every integer and so is $\mathbb{Z}_p$.
        \end{proof}
        \step{2}{$\mathbb{Q}$ is a prime field.}
        \begin{proof}
            \pf\ If $F$ is a subfield of $\mathbb{Q}$ then $F$ contains every integer, hence contains $m/n$ for
            $m$ and $n$ integers with $n \neq 0$, and so is $\mathbb{Q}$.
        \end{proof}
        \step{3}{For $p$ prime, if $F$ is a prime field of characteristic $p$ then $F \cong \mathbb{Z}_p$.}
        \begin{proof}
            \step{a}{If $F$ is any field of characteristic $p$ then $\mathbb{Z}_p$ is a subfield of $F$.}
            \begin{proof}
                \step{i}{Define $\phi : \mathbb{Z}_p \rightarrow F$ by $\phi(k) = k1$}
                \step{ii}{$\phi$ is injective.}
                \begin{proof}
                    \pf\ Since $k1 \neq l1$ for $0 \leq k,l < p$.
                \end{proof}
                \step{iii}{$\phi$ preserves addition.}
                \begin{proof}
                    \pf\ If $k + l \cong m (\mod p)$ then $k1 + l1 = m1$ in $F$.
                \end{proof}
                \step{iv}{$\phi$ preserves multiplication.}
                \begin{proof}
                    \pf\ If $kl \cong m (\mod p)$ then $(k1)(l1) = m1$ in $F$.
                \end{proof}
            \end{proof}
        \end{proof}
        \step{4}{If $F$ is a prime field of characteristic 0 then $F \cong \mathbb{Q}$.}
        \begin{proof}
            \step{a}{If $F$ is any field of characteristic 0 then $\mathbb{Q}$ is a subfield of $F$.}
        \end{proof}
        \qed
    \end{proof}

    \section{Rational Numbers}

    \begin{lemma}
        If $(a,b) \sim (a',b')$ and $(c,d) \sim (c',d')$ and $b$, $b'$, $d$, $d'$
        are all positive then $ad<bc$ iff $a'd'<b'c'$.
    \end{lemma}

    \begin{proof}
        \pf\ Easy.
    \end{proof}

    \begin{definition}
        The ordering on the rationals is defined by: if $b$ and $d$ are positive then
        $[(a,b)] < [(c,d)]$ iff $ad < bc$.
    \end{definition}

    \begin{theorem}
        The relation $<$ is a linear ordering on $\mathbb{Q}$.
    \end{theorem}

    \begin{proof}
        \pf\ Easy. \qed
    \end{proof}

    \begin{definition}[Positive]
        A rational $q$ is \emph{positive} iff $0 < q$.
    \end{definition}

    \begin{definition}[Absolute Value]
        The \emph{absolute value} of a rational $q$ is the rational $|q|$ defined by
        \[ |q| = \begin{cases}
            q & \text{if } q \geq 0 \\
            -q & \text{if } q \leq 0
        \end{cases} \]
    \end{definition}

    \begin{theorem}
        For any rational $s$, the function that maps $q$ to $q + s$ is strictly monotone.
    \end{theorem}
    
    \begin{proof}
        \pf\ Easy. \qed
    \end{proof}

    \begin{theorem}
        For any positive rational $s$, the function that maps $q$ to $qs$ is strictly monotone.
    \end{theorem}

    \begin{proof}
        \pf\ Easy. \qed
    \end{proof}

    \begin{theorem}
        Define $E : \mathbb{Z} \rightarrow \mathbb{Q}$ by $E(a) = [(a,1)]$. Then $E$ is one-to-one and:
        \begin{enumerate}
            \item $E(a+b) = E(a) + E(b)$
            \item $E(ab) = E(a)E(b)$
            \item $E(0) = 0$
            \item $E(1) = 1$
            \item $a < b$ iff $E(a) < E(b)$
        \end{enumerate}
    \end{theorem}

    \begin{proof}
        \pf\ Easy. \qed
    \end{proof}

    \section{Ordered Fields}

    \begin{definition}[Ordered Field]
        An \emph{ordered field} is an ordered integral domain $(D, +, \cdot, 0, 1, <)$ such that $(D, +, \cdot, 0, 1)$
        is a field.
    \end{definition}

    \begin{theorem}
        The quotient field $F$ of an ordered integral domain $D$ is an ordered field under: $[(a,b)]$ is positive iff $ab > 0$ in $D$.
        The canonical imbedding $D \hookrightarrow F$ is strictly monotone.
    \end{theorem}

    \begin{proof}
        \pf
        \step{1}{\pflet{$D$ be an ordered integral domain and $F$ its quotient field.}}
        \step{2}{Define a fraction $[(a,b)]$ to be positive iff $ab > 0$}
        \begin{proof}
            \step{a}{\pflet{$a,b,c,d \in D$ with $b \neq 0 \neq d$}}
            \step{b}{\assume{$(a,b) \sim (c,d)$ and $ab > 0$} \prove{$cd > 0$}}
            \step{c}{$ad = bc$}
            \step{d}{\case{$d > 0$}}
            \begin{proof}
                \step{i}{$abd > 0$}
                \step{ii}{$b^2c > 0$}
                \step{iii}{$c > 0$}
                \step{iv}{$cd > 0$}
            \end{proof}
            \step{e}{\case{$d < 0$}}
            \begin{proof}
                \step{i}{$abd < 0$}
                \step{ii}{$b^2 c < 0$}
                \step{iii}{$c < 0$}
                \step{iv}{$cd > 0$}
            \end{proof}
        \end{proof}
        \step{3}{$0$ is not positive.}
        \begin{proof}
            \pf\ Since $0 \times 1 \ngtr 0$.
        \end{proof}
        \step{4}{For any $x \in F$, either $x$ is positive or $x = 0$ or $-x$ is positive.}
        \begin{proof}
            \step{a}{\pflet{$x = [(a,b)]$}}
            \step{b}{Either $ab > 0$ or $ab = 0$ or $ab < 0$}
            \step{c}{If $ab < 0$ then $-x$ is positive.}
            \begin{proof}
                \pf\ Since $-x = [(-a,b)]$ and $-ab > 0$.
            \end{proof}
        \end{proof}
        \step{5}{If $x$ and $y$ are positive then $x+y$ is positive.}
        \begin{proof}
            \step{a}{\pflet{$x = [(a,b)]$ and $y = [(c,d)]$}}
            \step{b}{\assume{$ab > 0$ and $cd > 0$}}
            \step{c}{$x + y = [(ad+bc,bd)]$}
            \step{d}{$(ad+bc)bd > 0$}
        \end{proof}
        \step{6}{If $x$ and $y$ are positive then $xy$ is positive.}
        \begin{proof}
            \step{a}{\pflet{$x = [(a,b)]$ and $y = [(c,d)]$}}
            \step{b}{\assume{$ab > 0$ and $cd > 0$}}
            \step{c}{$xy = [(ac,bd)]$}
            \step{d}{$acbd > 0$}
        \end{proof}
        \step{7}{For $a, b \in D$, if $a < b$ then $[(a,1)] < [(b,1)]$}
        \begin{proof}
            \pf\ We have $[(a-b,1)]$ is positive because $a-b > 0$.
        \end{proof}
        \qed
    \end{proof}

    \begin{corollary}
        The rationals are an ordered field under $p/q < r/s$ iff $ps < rq$ for $q, s$ positive.
    \end{corollary}

    \begin{theorem}
        The relation $p/q < r/s$ iff $ps < qr$ for $q,s$ positive is the only relation that makes $\mathbb{Q}$ into an ordered field.
    \end{theorem}

    \begin{proof}
        \pf\ If $\mathbb{Q}$ is an ordered field under $<$ then, for $q$, $s$ positive:
        \begin{align*}
            p/q < r/s & \Leftrightarrow ps < qr  \\
            & \Leftrightarrow 
        \end{align*}
    \end{proof}

    \begin{proposition}
        In any ordered field, if $x \neq 0$, then $x > 0$ iff $\inv{x} > 0$.
    \end{proposition}

    \begin{proof}
        \pf
        \step{1}{If $x > 0$ then $\inv{x} > 0$}
        \begin{proof}
            \pf\ If $\inv{x} \leq 0$ then $x \inv{x} = 1 \leq 0$.
        \end{proof}
        \step{2}{If $\inv{x} > 0$ then $x > 0$}
        \begin{proof}
            \pf\ From \stepref{1} since $\inv{(\inv{x})} = x$.
        \end{proof}
        \qed
    \end{proof}

    \begin{corollary}
        In any ordered field, if $x \neq 0$, then $x < 0$ iff $\inv{x} < 0$.
    \end{corollary}

    \begin{proposition}
        In any ordered field, if $y > 0$ and $v > 0$ then $x/y < u/v$ iff $xv = yu$.
    \end{proposition}

    \begin{proof}
        \pf\ Multiplying by $yv$ or by $\inv{y}\inv{v}$. \qed
    \end{proof}

    \begin{proposition}
        In any ordered field, if $y \neq 0$ then $|x/y| = |x|/|y|$.
    \end{proposition}

    \begin{proof}
        \pf\ Since $|x/y||y| = |x|$. \qed
    \end{proof}

    \begin{corollary}
        In any ordered field, if $y \neq 0$ then $|\inv{y}| = 1 / |y|$.
    \end{corollary}

    \begin{proposition}[Density]
        In any ordered field, if $x < y$ then $x < (x+y)/2 < y$.
    \end{proposition}

    \begin{proof}
        \pf\ If $x < y$ then $2x < x+y$ so $x < (x+y)/2$, and $x + y < 2y$ so $(x+y)/2 < y$. \qed
    \end{proof}

    \begin{proposition}[Cauchy-Schwarz Inequality]
        Let $F$ be an ordered field. Let $a_1, \ldots, a_n, b_1, \ldots, b_n \in F$. Then
        \[ (a_1 b_1 + \cdots + a_n b_n)^2 \leq (a_1^2 + \cdots + a_n^2)(b_1^2 + \cdots + b_n^2) \enspace . \]
    \end{proposition}

    \begin{proof}
        \pf
        \step{1}{$\sum_{i=1}^n \sum_{j=1}^n (a_i b_j - a_j b_i)^2 = 2 \sum_{i=1}^n a_i^2 \sum_{j=1}^n b_j^2
        - 2 \left( \sum_{i=1}^n a_i b_i \right)^2$}
        \begin{proof}
            \pf
            \begin{align*}
                \sum_{i=1}^n \sum_{j=1}^n (a_i b_j - a_j b_i)^2
                & = \sum_{i=1}^n \sum_{j=1}^n a_i^2 b_j^2 - 2 \sum_{i_1}^n \sum_{j=1}^n
                a_i b_j a_j b_i + \sum_{i=1}^n \sum_{j=1}^n a_j^2 b_i^2 \\
                & = 2 \sum_{i=1}^n a_i^2 \sum_{j=1}^n b_j^2 - 2 \left( \sum_{i=1}^n a_i b_i \right)^2
            \end{align*}
        \end{proof}
        \qedstep
        \begin{proof}
            \pf\ Since a sum of squares must be $\geq 0$.
        \end{proof}
        \qed
    \end{proof}

    \begin{definition}[Cut]
        Let $F$ be an ordered field. A \emph{cut} in $F$ is a pair $(A,B)$ of subsets of $F$ such that:
        \begin{enumerate}
            \item $A$ and $B$ are nonempty.
            \item $A \cup B = F$
            \item $\forall x \in A. \forall y \in B. x < y$
        \end{enumerate}
    \end{definition}

    \begin{definition}[Gap]
        Let $F$ be an ordered field. A \emph{gap} in $F$ is a cut $(A,B)$ in $F$ such that $A$ has no
        maximum element and $B$ has no minimum element.
    \end{definition}

    \begin{proposition}
        \label{prop:gap_supremum}
        Let $(A,B)$ be a cut in an ordered field $F$. Then $(A,B)$ is a gap if and only if $A$ has no
        supremum.
    \end{proposition}

    \begin{proof}
        \pf
        \step{1}{If $A$ has a supremum then $(A,B)$ is not a gap.}
        \begin{proof}
            \step{a}{\pflet{$s$ be the supremum of $A$}}
            \step{b}{\case{$s \in A$}}
            \begin{proof}
                \pf\ In this case $s$ is the maximum element of $A$.
            \end{proof}
            \step{c}{\case{$s \in B$}}
            \begin{proof}
                \pf\ In this case $s$ is the minimum element of $B$.
            \end{proof}
        \end{proof}
        \step{2}{If $(A,B)$ is not a gap then $A$ has a supremum.}
        \begin{proof}
            \pf\ If $A$ has a maximum element then it is a supremum of $A$, and if $B$ has a minimum element
            then it is a supremum of $A$.
        \end{proof}
        \qed
    \end{proof}

    \begin{proposition}
        \label{prop:gap_infimum}
        Let $(A,B)$ be a cut in an ordered field $F$. Then $(A,B)$ is a gap if and only if $B$ has no
        infimum.
    \end{proposition}

    \begin{proof}
        \pf\ Dual. \qed
    \end{proof}

    \begin{definition}[Cut Determined by an Element]
        Let $F$ be an ordered field and $c \in F$. The cuts \emph{determined} by $c$ are
        $(\{ x \in F : x \leq c \}, \{ x \in F : x > c \})$ and $(\{ x \in F : x \leq c \}, \{ x \in F : x > c \})$.
    \end{definition}

    \begin{definition}[Complete Ordered Field]
        A \emph{complete ordered field} is an ordered field with no gaps.
    \end{definition}

    \begin{theorem}
        Let $F$ be an ordered field. The following are equivalent.
        \begin{enumerate}
            \item $F$ is complete.
            \item Every nonempty subset of $F$ bounded above has a supremum.
            \item Every nonempty subset of $F$ bounded below has an infimum.
        \end{enumerate}
    \end{theorem}

    \begin{proof}
        \pf
        \step{1}{$1 \Rightarrow 2$}
        \begin{proof}
            \step{a}{\assume{$F$ is complete}}
            \step{b}{\pflet{$A$ be a nonempty subset of $F$ bounded above.}}
            \step{c}{\pflet{$A_1 = \{ x \in F : \exists y \in A. x \leq y \}$}}
            \step{d}{\pflet{$B = F - A_1$}}
            \step{e}{$(A_1,B)$ is a cut.}
            \begin{proof}
                \step{a}{$A_1 \neq \emptyset$}
                \begin{proof}
                    \step{i}{\pick\ $a \in A$}
                    \begin{proof}
                        \pf\ $A$ is nonempty (\stepref{b}).
                    \end{proof}
                    \step{ii}{$a - 1 \in A_1$}
                \end{proof}
                \step{b}{$B \neq \emptyset$}
                \begin{proof}
                    \step{i}{\pick\ an upper bound $u$ for $A$}
                    \begin{proof}
                        \pf\ $A$ is bounded above (\stepref{b}).
                    \end{proof}
                    \step{ii}{$u + 1 \in B$}
                \end{proof}
                \step{c}{$A_1 \cup B = F$}
                \begin{proof}
                    \pf\ By \stepref{d}.
                \end{proof}
                \step{d}{$\forall x \in A_1. \forall y \in B. x < y$}
                \begin{proof}
                    \pf\ If $x \in A_1$ and $y \leq x$ then $y \in A_1$.
                \end{proof}
            \end{proof}
            \step{f}{$(A_1,B)$ is not a gap.}
            \begin{proof}
                \pf\ By \stepref{a}.
            \end{proof}
            \step{g}{\case{$A_1$ has a maximum element.}}
            \begin{proof}
                \step{i}{\pflet{$s$ be the maximum of $A_1$.}}
                \step{ii}{$s \in A$}
                \begin{proof}
                    \step{one}{\pick\ $x \in A$ such that $s \leq x$}
                    \step{two}{$x \in A_1$}
                    \step{three}{$x \leq s$}
                    \begin{proof}
                        \pf\ By the maximality of $s$.
                    \end{proof}
                    \step{four}{$x = s$}
                \end{proof}
                \step{ii}{$s$ is an upper bound for $A$.}
                \begin{proof}
                    \pf\ Since $A \subseteq A_1$.
                \end{proof}
                \step{iii}{$s$ is the maximum element of $A$.}
                \step{iv}{$s$ is the supremum of $A$.}
            \end{proof}
            \step{h}{\case{$B$ has a minimum element.}}
            \begin{proof}
                \step{i}{\pflet{$s$ be the minimum element in $B$.}}
                \step{ii}{$s$ is an upper bound for $A$}
                \begin{proof}
                    \pf\ For all $x \in A$ we have $x \in A_1$ and so $x < s$.
                \end{proof}
                \step{iii}{For any upper bound $u$ for $A$ we have $s \leq u$}
                \begin{proof}
                    \step{one}{\pflet{$u$ be an upper bound for $A$.}}
                    \step{two}{$u \notin A_1$}
                    \begin{proof}
                        \step{O}{\assume{for a contradiction $u \in A_1$}}
                        \step{A}{$u < s$}
                        \step{B}{\pick\ $y$ such that $u < y < s$}
                        \step{C}{\case{$y \in A_1$}}
                        \begin{proof}
                            \step{ONE}{\pick\ $x \in A$ such that $y \leq x$}
                            \step{TWO}{$u < x$}
                            \qedstep
                            \begin{proof}
                                \pf\ This contradicts \stepref{one}.
                            \end{proof}
                        \end{proof}
                        \step{D}{\case{$y \in B$}}
                        \begin{proof}
                            \pf\ This contradicts \stepref{i}.
                        \end{proof}
                    \end{proof}
                    \step{three}{$u \in B$}
                    \step{four}{$s \leq u$}
                    \begin{proof}
                        \pf\ By minimality of $s$.
                    \end{proof}
                \end{proof}
            \end{proof}
        \end{proof}
        \step{2}{$2 \Rightarrow 1$}
        \begin{proof}
            \pf\ By Proposition \ref{prop:gap_supremum}
        \end{proof}
        \step{3}{$1 \Rightarrow 3$}
        \begin{proof}
            \pf\ Similar to \stepref{1}.
        \end{proof}
        \step{4}{$3 \Rightarrow 1$}
        \begin{proof}
            \pf\ By Proposition \ref{prop:gap_infimum}.
        \end{proof}
        \qed
    \end{proof}

    \begin{definition}[Archimedean]
        An ordered field $F$ is \emph{Archimedean} if and only if, for all positive $x,y \in F$, there exists
        $n \in \mathbb{Z}^+$ such that $nx > y$.
    \end{definition}

    \begin{lemma}
        \label{lemma:rationals_Archimedean}
        The rational numbers are Archimedean.
    \end{lemma}

    \begin{proof}
        \pf\ Let $p = a/b$ and $r = c/d$ where $a$, $b$ and $d$ are positive. 
        Let $n = bc+1$. Then $bc < adn$ so $r < pn$. \qed
    \end{proof}

    \begin{example}
        The quotient field of $\mathbb{Z}[x]$ is not Archimedean, since $n1 < x$ for all $n \in \mathbb{Z}^+$.
    \end{example}

    \begin{theorem}
        Let $F$ be an ordered field. Then $F$ is Archimedean if and only if the set of integers in $F$ is not
        bounded above.
    \end{theorem}

    \begin{proof}
        \pf
        \step{1}{If $F$ is Archimedean then the set of integers in $F$ is not bounded above.}
        \begin{proof}
            \step{a}{\assume{$F$ is Archimedean.}}
            \step{b}{For every integer $y$ in $F$, there exists an integer $n$ such that $n1 > y$.}
            \step{c}{The integers in $F$ have no upper bound.}
        \end{proof}
        \step{2}{If the set of integers in $F$ is not bounded above then $F$ is Archimedean.}
        \begin{proof}
            \step{a}{\assume{The set of integers in $F$ is not bounded above.}}
            \step{b}{\pflet{$x, y \in F$ be positive.}}
            \step{c}{\pick\ an integer $n$ such that $n1 > y / x$}
            \step{d}{$nx > y$}
        \end{proof}
        \qed
    \end{proof}

    \begin{corollary}
        Let $F$ be an ordered field. Then $F$ is Archimedean if and only if, for every positive $z \in F$,
        there exists a positive integer $n$ such that $1/n < z$.
    \end{corollary}

    \begin{corollary}
        \label{cor:under_one_over_n}
        Let $F$ be an Archimedean ordered field and $x \in F$. If $x < 1/n$ for all $n \in \mathbb{Z}^+$
        then $x \leq 0$.
    \end{corollary}

    \begin{theorem}
        Let $F$ be an Archimedean ordered field. Let $x \in F$. Then there exists a unique integer $n$
        such that $n \leq x < n + 1$.
    \end{theorem}

    \begin{proof}
        \pf
        \step{1}{There exists an integer $n$ such that $n \leq x < n + 1$}
        \begin{proof}
            \step{2}{\pick\ a positive integer $j$ such that $-x < j$.}
            \begin{proof}
                \pf\ By the Archimedean property applied to $-x$.
            \end{proof}
            \step{4}{\pflet{$h$ be the least positive integer such that $x + j < h$.}}
            \begin{proof}
                \pf\ By the Archimedean property applied to $x + j$.
            \end{proof}
            \step{5}{\pflet{$n = h - j - 1$}}
            \step{6}{$x < n + 1$}
            \step{7}{$x \geq n$}        
            \begin{proof}
                \pf\ Since $h - 1 \leq x + j$ by the minimality of $h$ and the fact that $0 < x + j$.
            \end{proof}
        \end{proof}
        \step{2}{If $m$ and $n$ are integers with $m \leq x < m + 1$ and $n \leq x < n + 1$ then $m = n$}
        \begin{proof}
            \step{a}{$m < n + 1$}
            \step{b}{$m \leq n$}
            \step{c}{$n < m + 1$}
            \step{d}{$n \leq m$}
        \end{proof}
        \qed
    \end{proof}

    \begin{definition}[Floor]
        In any Archimedean ordered field, the \emph{floor} of $x$, $\llcorner x \lrcorner$, is the integer
        such that $\llcorner x \lrcorner \leq x < \llcorner x \lrcorner + 1$.
    \end{definition}

    \begin{theorem}
        Every complete ordered field is Archimedean.
    \end{theorem}

    \begin{proof}
        \pf
        \step{1}{\pflet{$F$ be a complete ordered field.}}
        \step{2}{\assume{for a contradiction the integers of $F$ are bounded above.}}
        \step{3}{\pflet{$u$ be the supremum of the integers of $F$}}
        \step{4}{$u < \llcorner u \lrcorner + 1$}
        \step{5}{$\llcorner u \lrcorner + 1 \leq u$}
        \begin{proof}
            \pf\ Since $u$ is an upper bound for the integers of $F$.
        \end{proof}
        \qedstep
        \begin{proof}
            \pf\ This is a contradiction.
        \end{proof}
        \qed
    \end{proof}

    \begin{definition}[Dense]
        Let $F$ be an ordered field and $A \subseteq F$. Then $A$ is \emph{dense} in $F$ if and only if,
        for all $x, y \in F$ with $x < y$, there exists $z \in A$ with $x < z < y$.
    \end{definition}

    \begin{theorem}
        \label{theorem:rationals_dense}
        Let $F$ be an ordered field. Then $F$ is Archimedean if and only if the rational numbers are dense
        in $F$.
    \end{theorem}

    \begin{proof}
        \pf
        \step{1}{If $F$ is Archimedean then the rationals are dense in $F$.}
        \begin{proof}
            \step{1}{\pflet{$F$ be an Archimedean ordered field.}}
            \step{2}{\pflet{$x, y \in F$ with $x < y$.}}
            \step{3}{\pflet{$n$ be the least positive integer such that $1 / n < y - x$.}}
            \step{4}{\pflet{$k = \llcorner n x + 1 \lrcorner$}}
            \step{5}{$nx < k \leq nx + 1$}
            \step{6}{$x < k / n < y$}
        \end{proof}
        \step{2}{If the rationals are dense in $F$ then $F$ is Archimedean.}
        \begin{proof}
            \step{a}{\assume{The rationals are dense in $F$.}}
            \step{b}{\pflet{$x \in F$} \prove{There exists an integer $n$ such that $x < n$}}
            \step{c}{\assume{w.l.o.g. $x > 0$}}
            \step{d}{\pick\ }
        \end{proof}
        \qed
    \end{proof}

    \begin{lemma}
        \label{lm:sup_of_rationals}
        Let $F$ be an Archimedean ordered field. Let $x \in F$. Then $x$ is the supremum of
        $A = \{ q \in F : q \text{ is rational}, q < x \}$.
    \end{lemma}

    \begin{proof}
        \pf
        \step{1}{$x$ is an upper bound for $A$.}
        \step{2}{For any upper bound $u$ for $A$ we have $x \leq u$}
        \begin{proof}
            \step{a}{\pflet{$u$ be an upper bound for $A$.}}
            \step{b}{\assume{for a contradiction $u < x$.}}
            \step{c}{\pick\ a rational $q$ with $u < q < x$.}
            \begin{proof}
                \pf\ Theorem \ref{theorem:rationals_dense}.
            \end{proof}
            \step{d}{$q \in A$ and $u < q$}
            \qedstep
            \begin{proof}
                \pf\ This contradicts \stepref{a}.
            \end{proof}
        \end{proof}
        \qed
    \end{proof}

    \section{The Real Numbers}

    \begin{definition}[Dedekind Cut]
        A \emph{real number} or \emph{Dedekind cut} is a subset $x$ of $\mathbb{Q}$ such that:
        \begin{enumerate}
            \item $\emptyset \neq x \neq \mathbb{Q}$
            \item $x$ is \emph{closed downwards}, i.e. for all $q \in x$, if $r \in \mathbb{Q}$
            and $r < q$ then $r \in x$.
            \item $x$ has no largest member.
        \end{enumerate}
        Let $\mathbb{R}$ be the set of all real numbers.
    \end{definition}

    \begin{definition}
        For any rational number $u$, let $u_\mathbb{R} = \{ x \in \mathbb{Q} : x < u \}$.
    \end{definition}

    \begin{proposition}
        $\forall u \in \mathbb{Q}. u_\mathbb{R} \in \mathbb{R}$
    \end{proposition}

    \begin{proof}
        \pf
        \step{1}{\pflet{$u \in \mathbb{Q}$}}
        \step{2}{$u_\mathbb{R} \neq \emptyset$}
        \begin{proof}
            \pf\ Since $u-1 \in u_\mathbb{R}$.
        \end{proof}
        \step{3}{$u_\mathbb{R} \neq \mathbb{Q}$}
        \begin{proof}
            \pf\ Since $u \notin u_\mathbb{R}$.
        \end{proof}
        \step{4}{$u_\mathbb{R}$ is closed downwards.}
        \begin{proof}
            \pf\ If $x < y < u$ then $x < u$.
        \end{proof}
        \step{5}{$u_\mathbb{R}$ has no largest member.}
        \begin{proof}
            \pf\ If $x \in u_\mathbb{R}$ then $x < (x+u)/2 \in u_\mathbb{R}$.
        \end{proof}
        \qed
    \end{proof}
    \begin{definition}
        Given real numbers $x$ and $y$, we write $x < y$ iff $x \subset y$.
    \end{definition}

    \begin{theorem}
        The relation $<$ is a linear ordering on $\mathbb{R}$.
    \end{theorem}

    \begin{proof}
        \pf\ The only hard part is proving that, for any reals $x$ and $y$, either $x \subseteq y$ or
        $y \subseteq x$.

        Suppose $x \nsubseteq y$. Pick $q \in x$ such that $q \notin y$. Let $r \in y$. Then $q \nless r$
        (since $y$ is closed downwards) therefore $r < q$. Hence $r \in x$ (because $x$ is closed downwards).
        \qed
    \end{proof}

    \begin{theorem}
        Any nonempty set $A$ of reals bounded above has a least upper bound.
    \end{theorem}

    \begin{proof}
        \pf\ We prove that $\bigcup A$ is a Dedekind cut. It is then the least upper bound of $A$.

        The set $\bigcup A$ is nonempty because $A$ is nonempty. Pick an upper bound $r$ for $A$, and a
        rational $q \notin r$; then $q \notin \bigcup A$, so $\bigcup A \neq \mathbb{Q}$.

        $\bigcup A$ is closed downwards because every member of $A$ is closed downwards.

        $\bigcup A$ has no largest member because every member of $A$ has no largest member. \qed
    \end{proof}

    \begin{definition}[Addition]
        \emph{Addition} $+$ on $\mathbb{R}$ is defined by:
        \[ x + y = \{ q + r \mid q \in x, r \in y \} \enspace .\]
        We prove this is a Dedekind cut.
    \end{definition}

    \begin{proof}
        \pf
        \step{1}{$x + y \neq \emptyset$}
        \begin{proof}
            \pf\ Pick $q \in x$ and $r \in y$. Then $q + r \in x + y$.
        \end{proof}
        \step{2}{$x + y \neq \mathbb{Q}$}
        \begin{proof}
            \step{a}{\pick{$q \in \mathbb{Q} - x$ and $r \in \mathbb{Q} - y$}}
            \step{b}{For all $q' \in x$ we have $q' < q$}
            \step{c}{For all $r' \in y$ we have $r' < r$}
            \step{d}{For all $q' \in x$ and $r' \in y$ we have $q' + r' < q + r$}
            \step{e}{$q + r \notin x + y$}
        \end{proof}
        \step{3}{$x + y$ is closed downwards.}
        \begin{proof}
            \step{a}{\pflet{$q \in x$ and $r \in y$}}
            \step{b}{\pflet{$s < q + r$}}
            \step{c}{$s-q < r$}
            \step{d}{$s-q \in y$}
            \step{e}{$s = q + (s-q) \in x + y$}
        \end{proof}
        \step{4}{$x + y$ has no largest member.}
        \begin{proof}
            \step{a}{\pflet{$q \in x$ and $r \in y$}}
            \step{b}{\pick\ $q' \in x$ with $q < q'$}
            \step{c}{\pick\ $r' \in y$ with $r < r'$}
            \step{d}{$q' + r' \in x + y$ and $q + r < q' + r'$}
        \end{proof}
        \qed
    \end{proof}

    \begin{theorem}
        Addition is associative and commutative.
    \end{theorem}

    \begin{proof}
        \pf\ Easy. \qed
    \end{proof}

    \begin{theorem}
        For every real $x$ we have $x + 0_\mathbb{R} = x$.
    \end{theorem}

    \begin{proof}
        \pf
        \step{a}{$x + 0 \subseteq x$}
        \begin{proof}
            \pf\ Let $q \in x$ and $r \in 0$. Then $q + r < q$ so $q + r \in x$.
        \end{proof}
        \step{b}{$x \subseteq x + 0$}
        \begin{proof}
            \pf\ Let $q \in x$. Pick $r \in x$ such that $q < r$. Then $q - r \in 0$ and $q = r + (q-r) \in
            x + 0$.
        \end{proof}
        \qed
    \end{proof}

    \begin{definition}
        For any real $x$, define
        \[ - x = \{ r \in \mathbb{Q} : \exists s > r. -s \notin x \} \enspace . \]
        We prove this is a Dedekind cut.
    \end{definition}

    \begin{proof}
        \pf
        \step{1}{$-x \neq \emptyset$}
        \begin{proof}
            \pf\ Pick $s$ such that $s \notin x$. Then $-s-1 \in -x$.
        \end{proof}
        \step{2}{$-x \neq \mathbb{Q}$}
        \begin{proof}
            \step{a}{\pick\ $r \in x$ \prove{$-r \notin -x$}}
            \step{b}{\assume{for a contradiction $-r \in -x$}}
            \step{c}{\pick\ $s > -r$ such that $-s \notin x$}
            \step{d}{$-s < r$}
            \step{e}{$-s \in x$}
            \qedstep
            \begin{proof}
                \pf\ This is a contradiction.
            \end{proof}
        \end{proof}
        \step{3}{$-x$ is closed downwards.}
        \begin{proof}
            \pf\ Easy.
        \end{proof}
        \step{4}{$-x$ has no largest element.}
        \begin{proof}
            \step{a}{\pflet{$r \in -x$}}
            \step{b}{\pick\ $s > r$ such that $-s \notin x$}
            \step{c}{\pick\ $q$ such that $r < q < s$}
            \step{d}{$r < q$ and $q \in -x$}
        \end{proof}
        \qed
    \end{proof}

    \begin{lemma}
        \label{lemma:reals_pre_negation}
        Let $\epsilon$ be a positive real number. For any real $x$, there exists $q \in x$
        such that $q + \epsilon$ is an upper bound for $x$ but not the least upper bound for $x$.
    \end{lemma}

    \begin{proof}
        \pf
        \step{1}{\pick\ a rational $a_1 \in x$ such that if $x$ has a least upper bound $s$ then $a_1 > s - \epsilon$.}
        \step{3}{\pflet{$k$ be least such that $a_1 + k \epsilon$ is an upper bound for $x$}}
        \begin{proof}
            \pf\ By Lemma \ref{lemma:rationals_Archimedean}.
        \end{proof}
        \step{4}{$a_1 + k \epsilon$ is an upper bound for $x$ that is not the least upper bound for $x$}
        \step{5}{$a_1 + (k-1) \epsilon \in x$}
        \qed
    \end{proof}

    \begin{theorem}
        For any real $x$ we have $x + (-x) = 0$.
    \end{theorem}

    \begin{proof}
        \pf
        \step{1}{$x + (-x) \subseteq 0$}
        \begin{proof}
            \step{a}{\pflet{$q \in x$ and $r \in -x$}}
            \step{b}{\pick\ $s > r$ such that $-s \notin x$}
            \step{c}{$q < -s$}
            \step{d}{$q < -r$}
            \step{e}{$q + r < 0$}
        \end{proof}
        \step{2}{$0 \subseteq x + (-x)$}
        \begin{proof}
            \step{a}{\pflet{$p < 0$}}
            \step{b}{\pick\ $q \in x$ such that $q - p/2 \notin x$}
            \begin{proof}
                \pf\ By Lemma \ref{lemma:reals_pre_negation}.
            \end{proof}
            \step{c}{\pflet{$s = p/2 - q$}}
            \step{d}{$-s \notin x$}
            \step{e}{$p-q \in -x$}
            \begin{proof}
                \pf\ Since $p-q < s$ and $-s \notin x$.
            \end{proof}
            \step{e}{$p = q + (p-q) \in x + (-x)$}
        \end{proof}
        \qed
    \end{proof}

    \begin{theorem}
        The reals form an Abelian group under addition.
    \end{theorem}

    \begin{proof}
        \pf\ Easy. \qed
    \end{proof}

    \begin{theorem}
        For any real $z$, the function that maps $x$ to $x + z$ is strictly monotone.
    \end{theorem}

    \begin{proof}
        \pf
        \step{1}{\assume{$x < y$}}
        \step{2}{$x + z \subseteq y + z$}
        \begin{proof}
            \pf\ From the definition.
        \end{proof}
        \step{3}{$x + z \neq y + z$}
        \begin{proof}
            \pf\ By cancellation.
        \end{proof}
        \qed
    \end{proof}

    \begin{definition}[Absolute Value]
        The \emph{absolute value} of a real number $x$ is
        \[ |x| = \begin{cases}
            x & \text{if } x \geq 0 \\
            -x & \text{if } x \leq 0
        \end{cases} \]
    \end{definition}

    \begin{definition}[Multiplication]
        Given real numbers $x$, $y$, define the real $xy$ by:
        \begin{itemize}
            \item If $x \geq 0$ and $y \geq 0$ then
            \[ xy = 0 \cup \{ rs : 0 \leq r \in x, 0 \leq s \in y \} \]
            \item If $x \geq 0$ and $y < 0$ then $xy = -(x(-y))$
            \item If $x < 0$ and $y \geq 0$ then $xy = -((-x)y)$
            \item If $x < 0$ and $y < 0$ then $xy = (-x)(-y)$
        \end{itemize}

        We prove this is a Dedekind cut.
    \end{definition}

    \begin{proof}
        \pf
        \step{1}{\pflet{$x \geq 0$ and $y \geq 0$}}
        \step{2}{$xy \neq \emptyset$}
        \begin{proof}
            \pf\ Since $-1 \in xy$
        \end{proof}
        \step{3}{$xy \neq \mathbb{Q}$}
        \begin{proof}
            \step{a}{\pick\ $r \in \mathbb{Q} - x$ and $s \in \mathbb{Q} - y$}
            \step{b}{For all $r'$ with $0 \leq r' \in x$ and $s'$ with $0 \leq s' \in y$
            we have $r' < r$ and $s' < s$ so $r's' < rs$}
            \step{c}{$rs \notin xy$}
        \end{proof}
        \step{4}{$xy$ is closed downwards.}
        \begin{proof}
            \step{a}{\pflet{$q \in xy$ and $r < q$}}
            \step{b}{\assume{$0 \leq r$}}
            \step{c}{\pick\ rationals $a$, $b$ with $0 \leq a \in x$ and $0 \leq b \in y$
            such that $q = ab$}
            \step{d}{$a \neq 0$ or $b \neq 0$}
            \begin{proof}
                \pf\ Since $q \neq 0$ because $0 \leq r < q$.
            \end{proof}
            \step{e}{\assume{w.l.o.g. $a \neq 0$}}
            \step{f}{$r/a < b$}
            \step{g}{$r/a \in y$}
            \step{h}{$r = a(r/a) \in xy$}
        \end{proof}
        \step{5}{$xy$ has no greatest element.}
        \begin{proof}
            \step{a}{\pflet{$q \in xy$} \prove{There exists $r \in xy$ such that $q < r$}}
            \step{b}{\assume{w.l.o.g. $0 \leq q$}}
            \step{c}{\pick\ rationals $a$ and $b$ with $0 \leq a \in x$ and $0 \leq b \in y$
            such that $q = ab$}
            \step{d}{\pick\ rationals $a'$ and $b'$ with $a < a' \in x$ and $b < b' \in y$}
            \step{e}{$q < a'b' \in xy$}
        \end{proof}
        \qed
    \end{proof}

    \begin{theorem}
        Multiplication is commutative and associative.
    \end{theorem}

    \begin{proof}
        \pf\ Easy. \qed
    \end{proof}

    \begin{theorem}
        \[ \forall x,y,z \in \mathbb{R}. x(y+z) = xy + xz \]
    \end{theorem}

    \begin{proof}
        \pf
        \step{1}{\pflet{$x,y,z \in \mathbb{R}$}}
        \step{2}{\case{$x,y,z > 0$}}
        \begin{proof}
            \step{a}{$xy > 0$}
            \step{b}{$xz > 0$}
            \step{c}{$y + z > 0$}
            \step{d}{$x(y+z) > 0$}
            \step{e}{$xy + xz > 0$}
            \step{f}{$x(y+z) \subseteq xy + xz$}
            \begin{proof}
                \step{i}{\pflet{$q \in x(y+z)$}}
                \step{ii}{\assume{w.l.o.g. $0 < q$}}
                \begin{proof}
                    \pf\ Otherwise $q \in xy$ and $0 \in xz$ so $q \in xy + xz$
                \end{proof}
                \step{iii}{\pick\ $a \in x$, $b \in y$ and $c \in z$ such that $0 < a$, $0 < b+c$ and $q = a(b+c)$}
                \step{iv}{$ab \in xy$}
                \begin{proof}
                    \step{one}{\case{$b \leq 0$}}
                    \begin{proof}
                        \pf\ Then $ab \leq 0$ so $ab \in xy$
                    \end{proof}
                    \step{two}{\case{$b > 0$}}
                    \begin{proof}
                        \pf\ Then $ab \in xy$ by definition.
                    \end{proof}
                \end{proof}
                \step{v}{$ac \in xz$}
                \begin{proof}
                    \pf\ Similar.
                \end{proof}
                \step{vi}{$q \in xy + xz$}
            \end{proof}
            \step{g}{$xy + xz \subseteq x(y+z)$}
            \begin{proof}
                \step{zero}{\pflet{$q \in xy + xz$}}
                \step{one}{\case{$\exists a,a_1 \in x. \exists b \in y. \exists c \in z. (a,b,c,a_1 > 0
                \wedge q = ab + a_1 c)$}}
                \begin{proof}
                    \step{A}{\pflet{$a_2 = \max(a,a_1)$}}
                    \step{B}{$q \leq a_2(b+c)$}
                    \step{C}{$q \in x(y+z)$}
                \end{proof}
                \step{two}{\case{$\exists a \in x. \exists b \in y. \exists u \leq 0. q = ab + u$}}
                \begin{proof}
                    \step{A}{$ab + u \leq ab$}
                    \step{B}{$ab + u \in xy$}
                    \step{C}{\case{$ab + u \leq 0$}}
                    \begin{proof}
                        \pf\ $ab + u \in x(y+z)$
                    \end{proof}
                    \step{D}{\case{$ab + u > 0$}}
                    \begin{proof}
                        \step{ONE}{\pick\ $a' \in x$, $b' \in y$ such that $0 < a'$, $0 < b'$ and
                        $ab + u = a'b'$}
                        \step{TWO}{$b' \in y + z$}
                        \step{THREE}{$a'b' \in x(y+z)$}
                    \end{proof}
                \end{proof}
                \step{three}{\case{$\exists u \leq 0. \exists a \in x. \exists c \in z. q = u + ac$}}
                \begin{proof}
                    \pf\ Similar.
                \end{proof}
                \step{four}{\case{$\exists u,u' \leq 0. q = u + u'$}}
                \begin{proof}
                    \step{A}{$u + u' \leq 0$}
                    \step{B}{$u + u' \in x(y+z)$}
                \end{proof}
            \end{proof}
        \end{proof}
        \step{3}{\case{$x = 0$ or $y = 0$ or $z = 0$}}
        \begin{proof}
            \pf\ Then $x(y+z) = xy + xz = 0$
        \end{proof}
        \step{4}{\case{$x < 0$ and $y > 0$ and $z > 0$}}
        \begin{proof}
            \pf
            \begin{align*}
                x(y+z) & = -((-x)(y+z)) \\
                & = -((-x)y + (-x)z) & (\text{\stepref{2}})\\
                & = -(-(xy) + -(xz)) \\
                & = xy + xz
            \end{align*}
        \end{proof}
        \step{5}{\case{$x > 0$ and $y < 0$ and $z > 0$}}
        \begin{proof}
            \step{a}{$z = -y$}
            \begin{proof}
                \step{i}{$x(y+z)=0$}
                \step{ii}{$xy+xz=0$}
            \end{proof}
            \step{b}{$z > -y$}
            \begin{proof}
                \pf
                \begin{align*}
                    xy + xz & = xy + (x(-y + y + z)) \\
                    & = -(x(-y)) + x(-y) + x(y+z) & (\text{\stepref{2}})\\
                    & = x(y+z)
                \end{align*}
            \end{proof}
            \step{c}{$z < -y$}
            \begin{proof}
                \pf
                \begin{align*}
                    xy + xz & = -(x(-y)) + xz \\
                    & = -(x(z-y-z)) + xz \\
                    & = -(xz + x(-y-z)) + xz & (\text{\stepref{2}})\\
                    & = -xz - x(-y-z) + xz \\
                    & = -x(-y-z) \\
                    & = x(y+z)
                \end{align*}
            \end{proof}
        \end{proof}
        \step{6}{\case{$x > 0$ and $y < 0$ and $z < 0$}}
        \begin{proof}
            \pf
            \begin{align*}
                x(y+z) & = -(x(-y-z)) \\
                & = -(x(-y))-(x(-z)) & (\text{\stepref{2}}) \\
                & = xy + xz
            \end{align*}
        \end{proof}
        \step{7}{\case{$x < 0$ and $y < 0$ and $z > 0$}}
        \begin{proof}
            \step{a}{\case{$y = -z$}}
            \begin{proof}
                \pf\ Then $x(y+z) = xy+xz = 0$.
            \end{proof}
            \step{b}{\case{$y > -z$}}
            \begin{proof}
                \pf
                \begin{align*}
                    x(y+z) & = -((-x)(y+z)) \\
                    & = -((-x)y)- ((-x)z) & (\text{\stepref{5}}) \\
                    & = --((-x)(-y)) + xz \\
                    & = xy + xz
                \end{align*}
            \end{proof}
            \step{c}{\case{$y < -z$}}
            \begin{proof}
                \pf
                \begin{align*}
                    x(y+z) & = (-x)(-y-z) \\
                    & = (-x)(-y)+(-x)(-z) & (\text{\stepref{5}}) \\
                    & = xy + xz
                \end{align*}
            \end{proof}
        \end{proof}
        \step{8}{\case{$x < 0$ and $y < 0$ and $z < 0$}}
        \begin{proof}
            \pf
            \begin{align*}
                x(y+z) & = (-x)(-y-z) \\
                & = (-x)(-y) + (-x)(-z) & (\text{\stepref{2}}) \\
                & = xy + xz
            \end{align*}
        \end{proof}
        \qed
    \end{proof}

    \begin{definition}
        The real number \emph{one} is $1 = \{ q \in \mathbb{Q} : q < 1 \}$.

        It is easy to check this is a Dedekind cut.
    \end{definition}

    \begin{theorem}
        $0 \neq 1$
    \end{theorem}

    \begin{proof}
        \pf\ $0 \in 1$ and $0 \notin 0$. \qed
    \end{proof}

    \begin{theorem}
        For any real $x$, $x1 = x$.
    \end{theorem}

    \begin{proof}
        \pf
        \step{1}{\pflet{$x \in \mathbb{R}$} \prove{$x1=x$}}
        \step{2}{\case{$0 \leq x$}}
        \begin{proof}
            \step{a}{$x1 \subseteq x$}
            \begin{proof}
                \step{i}{\pflet{$q \in x1$} \prove{$q \in x$}}
                \step{ii}{\case{$q < 0$}}
                \begin{proof}
                    \pf\ Then $q \in x$ because $0 \leq x$.
                \end{proof}
                \step{iii}{\case{There exist nonnegative rationals $r \in x$, $s \in 1$ such that $q = rs$}}
                \begin{proof}
                    \pf\ Then $q < r \in x$ so $q \in x$.
                \end{proof}
            \end{proof}
            \step{b}{$x \subseteq x1$}
            \begin{proof}
                \step{i}{\pflet{$q \in x$}}
                \step{o}{\assume{w.l.o.g. $0 \leq q$}}
                \step{ii}{\pick\ $r \in x$ with $q < r$}
                \step{iii}{$0 \leq q/r < 1$}
                \step{iv}{$q = r (q/r) \in x1$}
            \end{proof}
        \end{proof}
        \step{3}{\case{$x < 0$}}
        \begin{proof}
            \pf\ Then $x1 = -((-x)1) = -(-x) = x$.
        \end{proof}
        \qed
    \end{proof}

    \begin{theorem}
        For any nonzero real $x$, there is a nonzero real $y$ such that $xy = 1$.
    \end{theorem}

    \begin{proof}
        \pf
        \step{1}{\case{$x > 0$}}
        \begin{proof}
            \step{a}{\pflet{$y = \{ q \in \mathbb{Q} : q \leq 0 \} \cup \{ 1/q : q \text{ is an upper bound
            of $x$ but not the least upper bound of $x$} \}$}}
            \step{b}{$y \in \mathbb{R}$}
            \begin{proof}
                \step{i}{$y \neq \emptyset$}
                \begin{proof}
                    \pf\ Since $-1 \in y$.
                \end{proof}
                \step{ii}{$y \neq \mathbb{Q}$}
                \begin{proof}
                    \pf\ Pick a positive integer $q \in x$. Then $1/q \notin y$.
                \end{proof}
                \step{iii}{$y$ is closed downwards.}
                \begin{proof}
                    \pf\ Easy.
                \end{proof}
                \step{iv}{$y$ has no largest member.}
                \begin{proof}
                    \step{one}{\pflet{$q \in y$} \prove{There exists $r \in y$ such that $q < r$}}
                    \step{two}{\case{$q \leq 0$}}
                    \begin{proof}
                        \step{A}{\pick\ a rational $r$ that is an upper bound of $x$ but not the least upper bound of $x$}
                        \step{B}{$q < 1/r \in y$}
                    \end{proof}
                    \step{three}{\case{$q > 0$}}
                    \begin{proof}
                        \step{A}{$1/q$ is an upper bound of $x$ but not the least upper bound of $x$}
                        \step{B}{\pick\ $r < 1/q$ such that $r$ is an upper bound of $x$ but not the least upper bound of $x$}
                        \step{C}{$q < 1/r \in y$}
                    \end{proof}
                \end{proof}
            \end{proof}
            \step{c}{$0 < y$}
            \begin{proof}
                \pf\ Easy
            \end{proof}
            \step{d}{$xy = 1$}
            \begin{proof}
                \step{i}{$xy \subseteq 1$}
                \begin{proof}
                    \step{one}{\pflet{$q \in xy$}}
                    \step{two}{\assume{w.l.o.g. $q > 0$}}
                    \step{three}{\pick\ $r \in x$ and $s \in y$ such that $r > 0$, $s > 0$ and $q = rs$}
                    \step{four}{$1/s$ is an upper bound of $x$}
                    \step{five}{$r < 1/s$}
                    \step{six}{$rs < 1$}
                \end{proof}
                \step{ii}{$1 \subseteq xy$}
                \begin{proof}
                    \step{one}{\pflet{$q$ be a rational with $0 < q < 1$}}
                    \step{two}{\pick\ $r \in x$ with $0 < r$}
                    \step{three}{$(1-q)r > 0$}
                    \step{four}{\pick\ $a \in x$ such that $a > 0$ and $a + (1-q)r$ is an upper bound for $x$ but not
                    the least upper bound for $x$}
                    \step{five}{\pflet{$w = a + (1-q)r$}}
                    \step{six}{$w-a = (1-q)r < (1-q)w$}
                    \step{seven}{$qw < a$}
                    \step{eight}{$w < a / q$}
                    \step{nine}{$a/q$ is an upper bound of $x$ and not the least upper bound of $x$.}
                    \step{ten}{$q/a \in y$}
                    \step{ten}{$q = a(q/a) \in xy$}
                \end{proof}
            \end{proof}
        \end{proof}
        \step{2}{\case{$x < 0$}}
        \begin{proof}
            \step{a}{\pick\ $y$ such that $(-x)y = 1$}
            \begin{proof}
                \pf\ By \stepref{1}.
            \end{proof}
            \step{b}{$x(-y) = 1$}
        \end{proof}
        \qed
    \end{proof}

    \begin{theorem}
        For any positive real $z$, the function that maps $x$ to $xz$ is strictly monotone.
    \end{theorem}

    \begin{proof}
        \pf
        \step{1}{\pflet{$0 < z$ and $x < y$}}
        \step{2}{$y-x > 0$}
        \step{3}{$z(y-x) > 0$}
        \begin{proof}
            \pf\ Definition of multiplication.
        \end{proof}
        \step{4}{$zx < zy$}
        \qed
    \end{proof}

    \section{Complete Ordered Fields}

    \begin{definition}[Complete Ordered Field]
        An ordered field is \emph{complete} iff it has the least upper bound property.
    \end{definition}

    \begin{theorem}
        The reals form a complete ordered field.
    \end{theorem}

    \begin{proof}
        \pf\ From the results above. \qed
    \end{proof}

    \begin{theorem}
        Any two complete ordered fields are isomorphic.
    \end{theorem}

    \begin{proof}
        \pf
        \step{1}{\pflet{$K$ and $F$ be complete ordered fields.}}
        \step{2}{\pflet{$\Phi : \mathbb{Q} \rightarrow K$ and $\Theta : \mathbb{Q} \rightarrow F$
        be the embeddings of $\mathbb{Q}$ in $K$ and $F$ respectively.}}
        \begin{proof}
            \pf\ Corollary \ref{cor:embed_rationals}.
        \end{proof}
        \step{2a}{For $z \in K$, \pflet{$Q(z) = \{ q \in \mathbb{Q} : \Phi(q) z < \}$}}
        \step{2}{\pflet{$W : K \rightarrow F$ be the function: $W(z) = \sup \{ \Theta(q) : q \in
        Q(z) \}$}}
        \begin{proof}
            \step{a}{For all $z \in K$, the set $\{ \Theta(q) : q \in
            Q(z) \}$ is bounded above.}
            \begin{proof}
                \step{i}{\pflet{$z \in K$}}
                \step{ii}{\pick\ a rational $r$ with $z \leq \Phi(r)$.}
                \step{iii}{$\Theta(r)$ is an upper bound for $\{ \Theta(q) : q \in Q(z) \}$.}
            \end{proof}
        \end{proof}
        \step{3}{$W$ is strictly monotone.}
        \begin{proof}
            \step{a}{\pflet{$x, y \in K$ with $x < y$}}
            \step{b}{\pick\ rationals $q$, $r$ such that $x < \Phi(q) < \Phi(r) < y$}
            \step{c}{$W(x) \leq \Theta(q)$}
            \begin{proof}
                \step{i}{$\Theta(q)$ is an upper bound for $\{ \Theta(s) : s \in Q(x) \}$}
                \begin{proof}
                    \step{one}{\pflet{$s \in Q(x)$} \prove{$\Theta(s) \leq \Theta(q)$}}
                    \step{two}{$\Phi(s) < \Phi(q)$}
                    \step{three}{$s < q$}
                    \step{four}{$\Theta(s) < \Theta(q)$}
                \end{proof}
            \end{proof}
            \step{d}{$\Theta(q) < \Theta(r)$}
            \begin{proof}
                \pf\ Since $\Phi(q) < \Phi(r)$ and $\Phi$ and $\Theta$ are embeddings.
            \end{proof}
            \step{e}{$\Theta(r) \leq W(y)$}
            \begin{proof}
                \pf\ By definition of $W(y)$ since $\Phi(r) < y$.
            \end{proof}
        \end{proof}
        \step{4}{$W \circ \Phi = \Theta$}
        \begin{proof}
            \pf\ Since $\Theta(q)$ is the supremum of $\{ \Theta(r) : r < q \}
            = \{ \Theta(r) : \Phi(r) < \Phi(q) \}$ by Lemma \ref{lm:sup_of_rationals}.
        \end{proof}
        \step{5}{$W$ is surjective.}
        \begin{proof}
            \step{a}{\pflet{$y \in F$}}
            \step{aa}{\pflet{$Q'(y) = \{ q \in \mathbb{Q} : \Theta(q) < y \}$}}
            \step{b}{\pflet{$x = \sup \{ \Phi(q) : q \in Q'(y) \}$} \prove{$W(x) = y$}}
            \begin{proof}
                \pf\ The set is bounded above similarly to \stepref{2}.
            \end{proof}
            \step{c}{$y$ is an upper bound for $\{ \Theta(q) : q \in Q(x) \}$}
            \begin{proof}
                \step{i}{\pflet{$q \in Q(x)$}}
                \step{ii}{$\Phi(q) < x$}
                \step{iii}{\pick\ $r \in Q'(y)$ such that $\Phi(q) < \Phi(r)$}
                \step{iv}{$\Theta(q) < \Theta(r) < y$}
            \end{proof}
            \step{d}{For any upper bound $u$ for $\{ \Theta(q) : q \in Q(x) \}$
            we have $y \leq u$}
            \begin{proof}
                \step{i}{\pflet{$u$ be an upper bound for $\{ \Theta(q) : q \in Q(x) \}$}}
                \step{ii}{\assume{for a contradiction $u < y$}}
                \step{iii}{\pick\ rationals $q$, $r$  with $u < \Theta(q) < \Theta(r) < y$}
                \step{iv}{$r \in Q'(y)$}
                \step{v}{$\Phi(r) \leq x$}
                \step{vi}{$\Phi(q) < x$}
                \step{vii}{$q \in Q(x)$}
                \step{viii}{$\Theta(q) \leq u$}
                \qedstep
                \begin{proof}
                    \pf\ This contradicts \stepref{iii}.
                \end{proof}
            \end{proof}
        \end{proof}
        \step{6}{For all $x, y \in K$ we have $W(x + y) = W(x) + W(y)$}
        \begin{proof}
            \step{a}{\pflet{$x, y \in K$}}
            \step{b}{For all $u \in Q(x+y)$, there exist $r_1 \in Q(x)$ and $r_2 \in Q(y)$ such that
            $u = r_1 + r_2$}
            \begin{proof}
                \step{i}{\pflet{$u \in Q(x+y)$}}
                \step{ii}{$\Theta(u) - y < x$}
                \step{iii}{\pick\ a rational $r_1$ such that $\Theta(u) - y < \Theta(r_1) < x$}
                \step{iv}{\pflet{$r_2 = u - r_1$}}
                \step{v}{$\Theta(r_2) < y$}
                \step{vi}{$u = r_1 + r_2$}
            \end{proof}
            \step{c}{$W(x+y) \leq W(x) + W(y)$}
            \begin{proof}
                \step{i}{$W(x) + W(y)$ is an upper bound for $Q(x+y)$}
                \begin{proof}
                    \step{one}{\pflet{$q \in Q(x+y)$}}
                    \step{two}{\pick\ $r_1 \in Q(x)$ and $r_2 \in Q(y)$ such that $q = r_1 + r_2$}
                    \step{three}{$r_1 \leq W(x)$}
                    \step{four}{$r_2 \leq W(y)$}
                    \step{five}{$q \leq W(x) + W(y)$}
                \end{proof}
            \end{proof}
            \step{d}{$W(x) + W(y) \leq W(x+y)$}
            \begin{proof}
                \step{i}{$\forall n \in \mathbb{Z}^+. W(x) + W(y) - W(x+y) \leq 1/n$}
                \begin{proof}
                    \step{one}{\pflet{$n \in \mathbb{Z}^+$}}
                    \step{two}{\pick\ $s_1 \in Q(x)$ such that $W(x) - 1/2n < s_1$}
                    \step{three}{\pick\ $s_2 \in Q(y)$ such that $W(y) - 1/2n < s_2$}
                    \step{four}{$W(x) + W(y) < W(x+y) + 1/n$}
                    \begin{proof}
                        \pf
                        \begin{align*}
                            W(x) + W(y) & < (s_1 + 1/2n) + (s_2 + 1/2n) \\
                            & = s_1 + s_2 + 1/n \\
                            & \leq W(x+y) + 1/n
                        \end{align*}
                    \end{proof}
                \end{proof}
                \step{ii}{$W(x) + W(y) - W(x+y) \leq 0$}
                \begin{proof}
                    \pf\ Corollary \ref{cor:under_one_over_n}.
                \end{proof}
            \end{proof}
        \end{proof}
        \step{6a}{For all $x \in K$ we have $W(-x) = -W(x)$}
        \begin{proof}
            \pf\ Since $W(x) + W(-x) = W(0) = 0$.
        \end{proof}
        \step{7}{For all $x, y \in K$ we have $W(xy) = W(x)W(y)$}
        \begin{proof}
            \step{a}{\pflet{$x, y \in K$}}
            \step{b}{\case{$x = 0$ or $y = 0$}}
            \begin{proof}
                \pf\ Then $W(xy) = W(x) W(y) = 0$
            \end{proof}
            \step{c}{\case{$x > 0$ and $y > 0$}}
            \begin{proof}
                \step{i}{For all $u \in \mathbb{Q}$, if $0 < u < xy$ then there exist rationals
                $r_1$, $r_2$ such that $0 < r_1 < x$, $0 < r_2 < y$ and $u = r_1 r_2$}
                \begin{proof}
                    \step{one}{\pflet{$u \in \mathbb{Q}$ with $0 < u < xy$}}
                    \step{two}{$u / y < x$}
                    \step{three}{\pick\ a rational $r_1$ with $u/y < r_1 < x$}
                    \step{four}{\pflet{$r_2 = u / r_1$}}
                    \step{five}{$r_2 < y$}
                \end{proof}
                \step{ii}{$W(xy) \leq W(x) W(y)$}
                \begin{proof}
                    \step{one}{\pflet{$q$ be a rational with $q < xy$} \prove{$q < W(x) W(y)$}}
                    \step{two}{\pick\ rationals $r_1$, $r_2$ with $0 < r_1 < x$, $0 < r_2 < y$
                    and $q = r_1 r_2$}
                    \step{three}{$r_1 \leq W(x)$}
                    \step{four}{$r_2 \leq W(y)$}
                    \step{five}{$q \leq W(x) W(y)$}
                \end{proof}
                \step{iii}{$W(x)W(y) \leq W(xy)$}
                \begin{proof}
                    \step{one}{\pick\ $k \in \mathbb{Z}^+$ such that $W(x) + W(y) < k$}
                    \step{two}{For all $n \in \mathbb{Z}^+$ we have $W(x)W(y) < W(xy) + 1/n$}
                    \begin{proof}
                        \step{A}{\pflet{$n \in \mathbb{Z}^+$}}
                        \step{B}{\pick\ $m \in \mathbb{Z}^+$ with $m > 2kn$ and $m > 1/W(x)$ and $m > 1/W(y)$}
                        \step{C}{\pick\ a rational $s_1$ with $s_1 < x$ and $W(x) - 1/m < s_1$}
                        \step{D}{\pick\ a rational $s_2$ with $s_2 < y$ and $W(y) - 1/m < s_2$}
                        \step{E}{$W(x)W(y) < W(xy) + 1/n$}
                        \begin{proof}
                            \pf
                            \begin{align*}
                                W(x) W(y) & < (s_1 + 1/m) (s_2 + 1/m) \\
                                & = s_1 s_2 + \frac{s_1 + s_2}{m} + 1/m^2 \\
                                & \leq W(xy) + \frac{W(x) + W(y)}{m} + 1/m^2 \\
                                & \leq W(xy) + k/m + 1/m^2 \\
                                & < W(xy) + 1/2n + 1/2n
                            \end{align*}
                        \end{proof}
                    \end{proof}
                    \step{three}{$W(x)W(y) \leq W(xy)$}
                    \begin{proof}
                        \pf\ Corollary \ref{cor:under_one_over_n}
                    \end{proof}
                \end{proof}
            \end{proof}
            \step{d}{\case{$x > 0$ and $y < 0$}}
            \begin{proof}
                \pf
                \begin{align*}
                    W(xy) & = W(-(x(-y))) \\
                    & = -W(x(-y)) & (\text{\stepref{5a}}) \\
                    & = -W(x)W(-y) & (\text{\stepref{c}}) \\
                    & = W(x)W(y) & (\text{\stepref{5a}})
                \end{align*}
            \end{proof}
            \step{e}{\case{$x < 0$ and $y > 0$}}
            \begin{proof}
                \pf\ Similar.
            \end{proof}
            \step{f}{\case{$x < 0$ and $y < 0$}}
            \begin{proof}
                \pf
                \begin{align*}
                    W(xy) & = W((-x)(-y)) \\
                    & = W(-x) W(-y) & (\text{\stepref{c}}) \\
                    & = (-W(x)) (-W(y)) & (\text{\stepref{5a}})\\
                    & = W(x) W(y)
                \end{align*}
            \end{proof}
        \end{proof}
        \qed
    \end{proof}

    \begin{theorem}
        Define $E : \mathbb{Q} \rightarrow \mathbb{R}$ by $E(q) = \{ p \in \mathbb{Q} : p < q \}$.
        Then $E$ is one-to-one and
        \begin{enumerate}
            \item $E(q+r) = E(q) + E(r)$
            \item $E(qr) = E(q) E(r)$
            \item $E(0) = 0$
            \item $E(1) = 1$
            \item $q < r$ iff $E(q) < E(r)$
        \end{enumerate}
    \end{theorem}

    \begin{proof}
        \pf
        \step{1}{For all $q \in \mathbb{Q}$, $E(q)$ is a Dedekind cut.}
        \begin{proof}
            \pf\ Easy.
        \end{proof}
        \step{2}{$\forall q,r \in \mathbb{Q}. E(q+r) = E(q) + E(r)$}
        \begin{proof}
            \step{a}{\pflet{$q, r \in \mathbb{Q}$}}
            \step{b}{$E(q+r) \subseteq E(q) + E(r)$}
            \begin{proof}
                \step{i}{\pflet{$t \in E(q+r)$}}
                \step{ii}{\pflet{$\epsilon = (r + s - t) / 2$}}
                \step{iii}{$\epsilon > 0$}
                \step{iv}{\pflet{$p = r - \epsilon$}}
                \step{v}{\pflet{$q = s - \epsilon$}}
                \step{vi}{$p < r$}
                \step{vii}{$q < s$}
                \step{viii}{$p + q = t$}
                \step{ix}{$t \in E(r) + E(s)$}
            \end{proof}
            \step{c}{$E(q) + E(r) \subseteq E(q+r)$}
            \begin{proof}
                \pf\ If $p < q$ and $s < r$ then $p+s < q+r$.
            \end{proof}
        \end{proof}
        \step{3}{$\forall q,r \in \mathbb{Q}. E(qr) = E(q)E(r)$}
        \begin{proof}
            \pf\ TODO
        \end{proof}
        \step{4}{$E(0) = 0$}
        \begin{proof}
            \pf\ By definition.
        \end{proof}
        \step{5}{$E(1) = 1$}
        \begin{proof}
            \pf\ By definition.
        \end{proof}
        \step{6}{$E$ is strictly monotone.}
        \begin{proof}
            \pf\ If $q < r$ then $E(q) \subseteq E(r)$ by transitivity of $<$ on $\mathbb{Q}$,
            and $E(q) \neq E(r)$ because $q \in E(r)$ and $q \notin E(q)$.
        \end{proof}
        \qed
    \end{proof}

    \begin{theorem}[Cantor 1873]
        The set $\omega$ is not equinumerous with $\mathbb{R}$.
    \end{theorem}

    \begin{proof}
        \pf
        \step{1}{\pflet{$f : \omega \rightarrow \mathbb{R}$} \prove{$f$ is not surjective.}}
        \step{2}{\pflet{$z$ be the real number between 0 and 1 whose $n+1$st decimal place is 7
        unless the $n+1$st decimal place of $f(n)$ is 7, in which case it is 6}}
        \step{3}{$\forall n \in \omega. f(n) \neq z$}
        \qed
    \end{proof}
\end{document}
\documentclass{article}

\title{C2 Algebra}
\author{Robin Adams}

\usepackage{amsmath}
\usepackage{amssymb}
\usepackage{amsthm}
\let\proof\relax
\let\endproof\relax
\let\qed\relax
\usepackage{pf2}
\usepackage[all]{xy}

\newtheorem{axiom}{Axiom}
\newtheorem{axs}[axiom]{Axiom Schema}
\newtheorem{lemma}[axiom]{Lemma}
\newtheorem{proposition}[axiom]{Proposition}
\newtheorem{props}[axiom]{Proposition Schema}
\newtheorem{theorem}[axiom]{Theorem}
\newtheorem{corollary}{Corollary}[axiom]
\theoremstyle{definition}
\newtheorem{definition}[axiom]{Definition}

\begin{document}
    \maketitle

    \section{Groups}

    \begin{definition}[Group]
        A \emph{group} is a triple $(G, \cdot, e)$ where $G$ is a set, $\cdot$ is a binary operation on $G$,
        and $e \in G$, such that:
        \begin{enumerate}
            \item $\cdot$ is associative.
            \item $\forall x \in G. xe = ex = x$
            \item $\forall x \in G. \exists y \in G. xy = yx = e$
        \end{enumerate}
    \end{definition}

    \begin{lemma}
        The integers $\mathbb{Z}$ form a group under $+$ and $0$.
    \end{lemma}

    \begin{proof}
        \pf\ Easy. \qed
    \end{proof}

    \begin{lemma}
        In any group, inverses are unique.
    \end{lemma}

    \begin{proof}
        \pf\ Suppose $y$ and $z$ are inverses to $x$. Then
        \[ y = ey = zxy = ze = z \]
        \qed
    \end{proof}

    \begin{definition}
        We write $x^{-1}$ for the inverse of $x$.
    \end{definition}

    \section{Abelian Groups}

    \begin{definition}[Abelian Group]
        A group $(G, +, 0)$ is \emph{Abelian} iff $+$ is commutative.

        When using additive notation (i.e. the symbols $+$ and $0$) for a group,
        we write $-y$ for the inverse of $y$, and $x-y$ for $x+(-y)$.
    \end{definition}

    \begin{lemma}
        The integers $\mathbb{Z}$ are Abelian.
    \end{lemma}

    \begin{proof}
        \pf\ Easy. \qed
    \end{proof}

    \begin{lemma}
        The rationals $\mathbb{Q}$ form an Abelian group under $+$.
    \end{lemma}

    \begin{proof}
        \pf\ Easy.
    \end{proof}

    \begin{lemma}
        The non-zero rationals form an Abelian group under multiplication.
    \end{lemma}

    \begin{proof}
        \pf\ Easy. \qed
    \end{proof}

    \section{Ring Theory}

    \begin{definition}[Commutative Ring]
        A \emph{commutative ring} is a quintuple $(R, +, \cdot, 0, 1)$ consisting of a set $R$,
        binary operations $+$ and $\cdot$ on $R$, and elements $0,1 \in R$ such that:
        \begin{enumerate}
            \item $(R, +, 0)$ is an Abelian group.
            \item The operation $\cdot$ is commutative, associative, and distributive over $+$.
            \item $\forall x \in R. x1 = x$
            \item $0 \neq 1$
        \end{enumerate}
    \end{definition}

    \begin{definition}[Integral Domain]
        An \emph{integral domain} is a ring such that, whenever $xy = 0$, then $x = 0$ or $y = 0$.
    \end{definition}

    \begin{lemma}
        The integers form an integral domain.
    \end{lemma}

    \begin{proof}
        \pf\ Easy. \qed
    \end{proof}

    \section{Field Theory}

    \begin{definition}[Field]
        A \emph{field} is an integral domain such that every non-zero element has a multiplicative inverse.
    \end{definition}

    \begin{definition}[Field of Fractions]
        Let $R$ be an integral domain. The \emph{field of fractions} of $R$ is $(R \times (R - \{ 0 \})) / \sim$,
        where $(a,b) \sim (c,d)$ iff $ad = bc$, under the following operations:
        \begin{align*}
            [(a,b)] + [(c,d)] & = [(ad+bc,bd)] \\
            [(a,b)][(c,d)] & = [(ac,bd)] \\
            0 & = [(0,1)] \\
            1 & = [(1,1)]
        \end{align*}

        It is routine to check that $\sim$ is an equivalence relation and the operations are well-defined
        and form a field. The additive inverse of $[(a,b)]$ is $[(-a,b)]$, and the multiplicative inverse
        of $[(a,b)]$ is $[(b,a)]$.
    \end{definition}

    \begin{definition}[Rational Numbers]
        The field of \emph{rational numbers} $\mathbb{Q}$ is the field of fractions of the integers.
    \end{definition}

    %TODO: The reals form a field.

    %TODO: The complex numbers form a field.

    \section{Rational Numbers}

    \begin{lemma}
        If $(a,b) \sim (a',b')$ and $(c,d) \sim (c',d')$ and $b$, $b'$, $d$, $d'$
        are all positive then $ad<bc$ iff $a'd'<b'c'$.
    \end{lemma}

    \begin{proof}
        \pf\ Easy.
    \end{proof}

    \begin{definition}
        The ordering on the rationals is defined by: if $b$ and $d$ are positive then
        $[(a,b)] < [(c,d)]$ iff $ad < bc$.
    \end{definition}

    \begin{theorem}
        The relation $<$ is a linear ordering on $\mathbb{Q}$.
    \end{theorem}

    \begin{proof}
        \pf\ Easy. \qed
    \end{proof}

    \begin{definition}[Positive]
        A rational $q$ is \emph{positive} iff $0 < q$.
    \end{definition}

    \begin{definition}[Absolute Value]
        The \emph{absolute value} of a rational $q$ is the rational $|q|$ defined by
        \[ |q| = \begin{cases}
            q & \text{if } q \geq 0 \\
            -q & \text{if } q \leq 0
        \end{cases} \]
    \end{definition}

    \begin{theorem}
        For any rational $s$, the function that maps $q$ to $q + s$ is strictly monotone.
    \end{theorem}
    
    \begin{proof}
        \pf\ Easy. \qed
    \end{proof}

    \begin{theorem}
        For any positive rational $s$, the function that maps $q$ to $qs$ is strictly monotone.
    \end{theorem}

    \begin{proof}
        \pf\ Easy. \qed
    \end{proof}

    \begin{theorem}
        Define $E : \mathbb{Z} \rightarrow \mathbb{Q}$ by $E(a) = [(a,1)]$. Then $E$ is one-to-one and:
        \begin{enumerate}
            \item $E(a+b) = E(a) + E(b)$
            \item $E(ab) = E(a)E(b)$
            \item $E(0) = 0$
            \item $E(1) = 1$
            \item $a < b$ iff $E(a) < E(b)$
        \end{enumerate}
    \end{theorem}

    \begin{proof}
        \pf\ Easy. \qed
    \end{proof}

    \section{Ordered Fields}

    \begin{definition}[Ordered Field]
        An \emph{ordered field} is a sextuple $(D, +, \cdot, \cdot, 0, 1, <)$ such that $(D, +, \cdot, 0, 1)$
        is a field, $<$ is a linear ordering on $D$, and:
        \begin{align*}
            & \forall x,y,z. x < y \Leftrightarrow x + z < y + z \\
            & \forall x,y,z. 0 < z \Rightarrow (x < y \Leftrightarrow xz < yz)
        \end{align*}
    \end{definition}

    \section{The Real Numbers}

    \begin{definition}[Dedekind Cut]
        A \emph{real number} or \emph{Dedekind cut} is a subset $x$ of $\mathbb{Q}$ such that:
        \begin{enumerate}
            \item $\emptyset \neq x \neq \mathbb{Q}$
            \item $x$ is \emph{closed downwards}, i.e. for all $q \in x$, if $r \in \mathbb{Q}$
            and $r < q$ then $r \in x$.
            \item $x$ has no largest member.
        \end{enumerate}
        Let $\mathbb{R}$ be the set of all real numbers.
    \end{definition}

    \begin{definition}
        Given real numbers $x$ and $y$, we write $x < y$ iff $x \subset y$.
    \end{definition}

    \begin{theorem}
        The relation $<$ is a linear ordering on $\mathbb{R}$.
    \end{theorem}

    \begin{proof}
        \pf\ The only hard part is proving that, for any reals $x$ and $y$, either $x \subseteq y$ or
        $y \subseteq x$.

        Suppose $x \nsubseteq y$. Pick $q \in x$ such that $q \notin y$. Let $r \in y$. Then $q \nless r$
        (since $y$ is closed downwards) therefore $r < q$. Hence $r \in x$ (because $x$ is closed downwards).
        \qed
    \end{proof}

    \begin{theorem}
        Any nonempty set $A$ of reals bounded above has a least upper bound.
    \end{theorem}

    \begin{proof}
        \pf\ We prove that $\bigcup A$ is a Dedekind cut. It is then the least upper bound of $A$.

        The set $\bigcup A$ is nonempty because $A$ is nonempty. Pick an upper bound $r$ for $A$, and a
        rational $q \notin r$; then $q \notin \bigcup A$, so $\bigcup A \neq \mathbb{Q}$.

        $\bigcup A$ is closed downwards because every member of $A$ is closed downwards.

        $\bigcup A$ has no largest member because every member of $A$ has no largest member. \qed
    \end{proof}

    \begin{definition}[Addition]
        \emph{Addition} $+$ on $\mathbb{R}$ is defined by:
        \[ x + y = \{ q + r \mid q \in x, r \in y \} \enspace .\]
        We prove this is a Dedekind cut.
    \end{definition}

    \begin{proof}
        \pf
        \step{1}{$x + y \neq \emptyset$}
        \begin{proof}
            \pf\ Pick $q \in x$ and $r \in y$. Then $q + r \in x + y$.
        \end{proof}
        \step{2}{$x + y \neq \mathbb{Q}$}
        \begin{proof}
            \step{a}{\pick{$q \in \mathbb{Q} - x$ and $r \in \mathbb{Q} - y$}}
            \step{b}{For all $q' \in x$ we have $q' < q$}
            \step{c}{For all $r' \in y$ we have $r' < r$}
            \step{d}{For all $q' \in x$ and $r' \in y$ we have $q' + r' < q + r$}
            \step{e}{$q + r \notin x + y$}
        \end{proof}
        \step{3}{$x + y$ is closed downwards.}
        \begin{proof}
            \step{a}{\pflet{$q \in x$ and $r \in y$}}
            \step{b}{\pflet{$s < q + r$}}
            \step{c}{$s-q < r$}
            \step{d}{$s-q \in y$}
            \step{e}{$s = q + (s-q) \in x + y$}
        \end{proof}
        \step{4}{$x + y$ has no largest member.}
        \begin{proof}
            \step{a}{\pflet{$q \in x$ and $r \in y$}}
            \step{b}{\pick\ $q' \in x$ with $q < q'$}
            \step{c}{\pick\ $r' \in y$ with $r < r'$}
            \step{d}{$q' + r' \in x + y$ and $q + r < q' + r'$}
        \end{proof}
        \qed
    \end{proof}

    \begin{theorem}
        Addition is associative and commutative.
    \end{theorem}

    \begin{proof}
        \pf\ Easy. \qed
    \end{proof}
\end{document}
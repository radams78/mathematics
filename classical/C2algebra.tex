\documentclass{article}

\title{C2 Algebra}
\author{Robin Adams}

\usepackage{amsmath}
\usepackage{amssymb}
\usepackage{amsthm}
\let\proof\relax
\let\endproof\relax
\let\qed\relax
\usepackage{pf2}
\usepackage[all]{xy}

\newtheorem{axiom}{Axiom}
\newtheorem{axs}[axiom]{Axiom Schema}
\newtheorem{lemma}[axiom]{Lemma}
\newtheorem{proposition}[axiom]{Proposition}
\newtheorem{props}[axiom]{Proposition Schema}
\newtheorem{theorem}[axiom]{Theorem}
\newtheorem{corollary}{Corollary}[axiom]
\theoremstyle{definition}
\newtheorem{definition}[axiom]{Definition}

\begin{document}
    \maketitle

    \section{Groups}

    \begin{definition}[Group]
        A \emph{group} is a triple $(G, \cdot, e)$ where $G$ is a set, $\cdot$ is a binary operation on $G$,
        and $e \in G$, such that:
        \begin{enumerate}
            \item $\cdot$ is associative.
            \item $\forall x \in G. xe = ex = x$
            \item $\forall x \in G. \exists y \in G. xy = yx = e$
        \end{enumerate}
    \end{definition}

    \begin{lemma}
        The integers $\mathbb{Z}$ form a group under $+$ and $0$.
    \end{lemma}

    \begin{proof}
        \pf\ Easy. \qed
    \end{proof}

    \begin{lemma}
        In any group, inverses are unique.
    \end{lemma}

    \begin{proof}
        \pf\ Suppose $y$ and $z$ are inverses to $x$. Then
        \[ y = ey = zxy = ze = z \]
        \qed
    \end{proof}

    \begin{definition}
        We write $x^{-1}$ for the inverse of $x$.
    \end{definition}

    \section{Abelian Groups}

    \begin{definition}[Abelian Group]
        A group $(G, +, 0)$ is \emph{Abelian} iff $+$ is commutative.

        When using additive notation (i.e. the symbols $+$ and $0$) for a group,
        we write $-y$ for the inverse of $y$, and $x-y$ for $x+(-y)$.
    \end{definition}

    \begin{lemma}
        The integers $\mathbb{Z}$ are Abelian.
    \end{lemma}

    \begin{proof}
        \pf\ Easy. \qed
    \end{proof}

    \begin{lemma}
        The rationals $\mathbb{Q}$ form an Abelian group under $+$.
    \end{lemma}

    \begin{proof}
        \pf\ Easy.
    \end{proof}
    
    \section{Ring Theory}

    \begin{definition}[Commutative Ring]
        A \emph{commutative ring} is a quintuple $(R, +, \cdot, 0, 1)$ consisting of a set $R$,
        binary operations $+$ and $\cdot$ on $R$, and elements $0,1 \in R$ such that:
        \begin{enumerate}
            \item $(R, +, 0)$ is an Abelian group.
            \item The operation $\cdot$ is commutative, associative, and distributive over $+$.
            \item $\forall x \in R. x1 = x$
            \item $0 \neq 1$
        \end{enumerate}
    \end{definition}

    \begin{definition}[Integral Domain]
        An \emph{integral domain} is a ring such that, whenever $xy = 0$, then $x = 0$ or $y = 0$.
    \end{definition}

    \begin{lemma}
        The integers form an integral domain.
    \end{lemma}

    \begin{proof}
        \pf\ Easy. \qed
    \end{proof}

\end{document}
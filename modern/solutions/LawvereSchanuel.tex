\documentclass{report}

\title{Solutions Manual for Lawvere and Schanuel \emph{Conceptual Mathematics}}
\author{Robin Adams}

\usepackage{amsmath}
\usepackage{amssymb}
\usepackage{pf2}

\newcommand{\inv}[1]{\ensuremath{{#1}^{-1}}}

\begin{document}
    \maketitle
    \tableofcontents

    \part{Preview}
    \chapter{Session 1 --- Galileo and Multiplication of Objects}
    \paragraph{Exercise 1}
    Many examples --- every instance of a product in a category gives an example. I will not list them.

    \paragraph{Exercise 2}
    I am not entirely sure what solution the authors had in mind. Here are some that come to my mind:

    Place a spirit level between the two points and see if it reads as level.

    Place a smooth plank between the two points and see if a ball placed at one point rolls to the other,
    or \emph{vice versa}.

    Hang a plumbline at each point and see if they form a right angle with the line joining the two points.

    Of these, the third is my favourite.

    \part{Part I --- The category of sets}

    \chapter{Article I --- Sets, maps, composition}

    \paragraph{Exercise 1}
    Easy.

    \paragraph{Exercise 2}
    There are 8 maps from $A$ to $B$.

    \paragraph{Exercise 3}
    There are 27 maps from $A$ to $A$.

    \paragraph{Exercise 4}
    There are 9 maps from $B$ to $A$.

    \paragraph{Exercise 5}
    There are 4 maps from $B$ to $B$.

    \paragraph{Exercise 6}
    There are 10 such maps from $A$ to $A$.

    \paragraph{Exercise 7}
    There are 3 such maps from $B$ to $B$.

    \paragraph{Exercise 8}
    There is no such pair of maps.

    \paragraph{Exercise 9}
    There are 12 such pairs of maps.

    \chapter{Session 3 --- Composing maps and counting maps}

    \paragraph{Exercise 1}
    (a) and (c) make sense.

    \paragraph{Exercise 2}
    (a) and (c) still make sense.

    \part{Part II --- The algebra of composition}

    \chapter{Article II --- Isomorphisms}

    \section{1. Isomorphisms}

    \paragraph{Exercise 1}
    \subparagraph{(R)}
    We have $1_A \circ 1_A = 1_A$ by the Identity Laws, so $1_A$ is an isomorphism with inverse $1_A$.

    \subparagraph{(S)}
    We have $g \circ f = 1_A$ and $f \circ g = 1_B$ (this is what it means for $g$ to be an inverse for $f$).
    This says exactly that $f$ is an inverse for $g$.

    \subparagraph{(T)}
    Let $\inv{f} : B \rightarrow A$ be an inverse for $f$ and $\inv{k} : C \rightarrow B$ be an inverse for $k$.
    We prove $\inv{f} \circ \inv{k}$ is an inverse for $k \circ f$. We have
    \begin{align*}
        \inv{f} \circ \inv{k} \circ k \circ f & = \inv{f} \circ 1_B \circ f & (\text{definition of inverse}) \\
        & = \inv{f} \circ f & (\text{Identity Law})\\
        & = 1_A & (\text{definition of inverse})
    \end{align*}
    and $k \circ f \circ \inv{f} \circ \inv{k} = 1_C$ similarly.

    \paragraph{Exercise 2}
    We have
    \begin{align*}
        g & = g \circ 1_B & (\text{Identity Law}) \\
        & = g \circ f \circ k & (\text{$k$ is an inverse of $f$}) \\
        & = 1_A \circ k & (\text{$g$ is an inverse of $f$}) \\
        & = k & (\text{Identity Law})
    \end{align*}

    \paragraph{Exercise 3}
    \subparagraph{(a)}
    Let $f : A \rightarrow B$. Let $h, k : C \rightarrow A$.

    Suppose $f \circ h = f \circ k$. Then
    \begin{align*}
        \inv{f} \circ f \circ h & = \inv{f} \circ f \circ k \\
        \therefore 1_A \circ h & = 1_A \circ k & (\text{Definition of inverse}) \\
        \therefore h & = k & (\text{Identity Law}) \qed
    \end{align*}

    \subparagraph{(b)}
    Let $f : A \rightarrow B$. Let $h, k : B \rightarrow C$.

    Suppose $h \circ f = k \circ f$. Then
    \begin{align*}
        h \circ f \circ \inv{f} & = k \circ f \circ \inv{f} \\
        \therefore h \circ 1_B & = k \circ 1_B & (\text{Definition of inverse}) \\
        \therefore h & = k & (\text{Identity Law}) \qed
    \end{align*}

    \subparagraph{(c)}
    Let $A = \{ 0,1 \}$. Define $f : A \rightarrow A$ by $f(0) = 1$ and $f(1) = 0$. Define $h : A \rightarrow A$
    by $h(x) = 0$ for all $x$. Define $k : A \rightarrow A$ by $k(x) = 1$ for all $x$.

    $f$ is invertible, and is its own inverse.

    We have $h \circ f = f \circ k = h$.

    We do not have $h = k$.

    \paragraph{Exercise 4}
    \subparagraph{(1)}
    This function is invertible with inverse $\inv{f}(x) = (x-7)/3$.

    \subparagraph{(2)}
    This function is invertible with inverse $\inv{g}(x) = \sqrt{x}$.

    \subparagraph{(3)}
    This function is not invertible because $h(1) = h(-1) = 1$.

    \subparagraph{(4)}
    This function is not invertible because $k(1) = k(-1) = 1$.

    \subparagraph{(5)}
    This function is not invertible because there is no $x$ such that $l(x) = 2$.

    \section{2 --- General division problems: Determination and choice}

    \paragraph{Exercise 5}
    There are 6 maps $f$ such that $g \circ f = 1_{\{0,1\}}$; we can map 0 to any of $b$, $p$ or $q$,
    and 1 to either of $r$ or $s$.

    Given any one of these maps $f$, there are 8 maps $g$ such that $g \circ f = 1_{\{0,1\}}$.
    We must map $f(0)$ to $0$, $f(1)$ to $1$, and the other three elements to any of 0 or 1.

    \paragraph{Exercise 6}
    If $r : B \rightarrow A$ is a section for $f$, then we take $t = g \circ r$. We have
    $t \circ f = g \circ r \circ f = g \circ 1_A = g$.

    \paragraph{Exercise 7}
    Let $s : B \rightarrow A$ be a section for $f$. Let $T$ be any set and $t_1, t_2 : T \rightarrow B$.
    Suppose $t_1 \circ f = t_2 \circ f$. Then
    \begin{align*}
        t_1 \circ f \circ s & = t_2 \circ f \circ s \\
        \therefore t_1 \circ 1_B & = t_2 \circ 1_B \\
        \therefore t_1 & = t_2
    \end{align*}

    \paragraph{Exercise 8}
    If $s_1 : B \rightarrow A$ is a section for $r_1 : A \rightarrow B$ and $s_2 : C \rightarrow B$
    is a section for $r_2 : B \rightarrow C$, then $s_1 \circ s_2$ is a section for $r_2 \circ r_1$ since
    \begin{align*}
        r_2 \circ r_1 \circ s_1 \circ s_2 & = r_2 \circ 1_B \circ s_2 \\
        & = r_2 \circ s_2 \\
        & = 1_C
    \end{align*}

    \paragraph{Exercise 9}
    We have
    \begin{align*}
        e \circ e & = f \circ r \circ f \circ r \\
        & = f \circ 1 \circ r & (\text{$r$ is a retraction of $f$}) \\
        & = f \circ r \\
        & = e
    \end{align*}

    \paragraph{Exercise 10}
    From the proof of Proposition 3, $\inv{f} \circ \inv{g}$ is both a section and a retraction for
    $g \circ f$.

    \paragraph{Exercise 11}
    Set $f(Fatima) = coffee$, $f(Omer) = tea$ and $f(Alysia) = cocoa$. Then $f$ is an isomorphism.

    There is no isomorphism $g : A \rightarrow C$. For if $g(Fatima) = true$ then $g(Omer)$ must be $false$,
    and then it is impossible to choose a value for $g(Alysia)$ without having $g(Alysia) = g(Fatima)$
    or $g(Alysia) = g(Omer)$. Similarly if $g(Fatima) = false$ then $g(Omer)$ must be $true$, and then
    again we cannot choose a value for $g(Alysia)$.

    \chapter{Session 4 --- Division of Maps: Isomorphisms}

    \section{4. A small zoo of isomorphisms in other categories}

    \paragraph{Exercise 1}
    We have $h(d(x)) = h(2x) = x$ and $d(h(x)) = d(x/2) = x$ for any $x$.

    \paragraph{Exercise 2}
    $f(odd) = negative$ and $f(even) = positive$

    \paragraph{Exercise 3}
    \subparagraph{(a)}
    This is not an isomorphism because $p(0 + 0) = 1$ but $p(0) + p(0) = 2$

    \subparagraph{(b)}
    This is not an isomorphism because it is not surjective; there is no $x$ such that $sq(x) = -1$.

    \subparagraph{(c)}
    This is not an isomorphism because it is not injective. We have $sq(1) = sq(-1) = 1$.

    \subparagraph{(d)}
    This is an isomorphism; it is bijective and $-(x+y) = (-x) + (-y)$.

    \subparagraph{(e)}
    This is not an isomorphism because $m(1 \times 1) = -1$ but $m(1) \times m(1) = 1$.

    \subparagraph{(f)}
    This is not a well-defined map because $c(-1) = -1 \notin \mathbb{R}_{>0}$.

    \chapter{Session 5 --- Division of Maps: Sections and Retractions}

    \section{1. Determination Problems}

    \paragraph{Exercise 1}
    \subparagraph{(a)}
    Suppose $h = g \circ f$ and $f a_1 = f a_2$. Then $h a_1 = g(f a_1) = g(f a_2) = h a_2$.

    \subparagraph{(b)}
    No. Take $A = C = \emptyset$ and $B = \{ * \}$. Let $f : A \rightarrow B$ and $h : A \rightarrow C$
    be the unique such maps. Vacuously, if $f a_1 = f a_2$ then $h a_1 = h a_2$. But there is no map
    $g : B \rightarrow C$.

    \section{3. Choice Problems}

    \paragraph{Exercise 2}
    \subparagraph{(a)}
    Suppose $g \circ f = h$. Let $a \in A$. Let $b = f(a)$. Then $h(a) = g(f(a)) = g(b)$.

    \subparagraph{(b)}
    This is equivalent to the Axiom of Choice.

    \section{5. Stacking or Sorting}

    \paragraph{Exercise 3}
    I'm not going to draw all of them, but there are 8 of them.

    \chapter{Session 9 --- Retracts and Idempotents}

    \section{1. Retracts and Comparisons}

    \paragraph{Exercise 1}
    If $A$ is empty, then the nowhere-defined function is a map $A \rightarrow B$.

    If $B$ has a point, say $b$, then the constant map with value $b$ is a map $A \rightarrow B$.

    \section{2. Idempotents as records of retracts}

    \paragraph{Exercise 3}
    Suppose $s : A \rightarrow B$, $r : B \rightarrow A$ and $s' : A' \rightarrow B$, $r' : B \rightarrow A'$
    are splittings of $e : B \rightarrow B$. Let
    \begin{align*}
        f & = r' \circ s & : A \rightarrow A' \\
        \inv{f} & = r \circ s' & : A' \rightarrow A
    \end{align*}
    Then we have
    \begin{align*}
        f \circ \inv{f} & = r' \circ s \circ r \circ s' \\
        & = r' \circ e \circ s' \\
        & = r' \circ s' \circ r' \circ s' \\
        & = 1 \\
        \inv{f} \circ f & = r \circ s' \circ r' \circ s \\
        & = r \circ e \circ s \\
        & = r \circ s \circ r \circ s \\
        & = 1
    \end{align*}

    \chapter{Quiz}

    \paragraph{Question 1}
    Let $A = \{ * \}$ and $B = \{ 0,1 \}$. Define $f : A \rightarrow B$ by $f(*) = 0$. Then the unique
    function $r : B \rightarrow A$ is a retraction for $f$ (since $r(f(*)) = *$) but not a section for $f$
    (since $f(s(1)) = 0$). Therefore there is no section for $f$, since there is only one map $B \rightarrow A$.

    \paragraph{Question 2}
    \subparagraph{(a)}
    Yes: if $pqp = p$ then $pqpq = pq$
    \subparagraph{(b)}
    Yes: if $pqp = p$ then $qpqp = qp$

    \paragraph{Question 2*}
    Let $q' = qpq$ Then we have
    \begin{align*}
        pq'p & = pqpqp \\
        & = pqp \\
        & = p \\
        q'pq' & = qpqpqpq \\
        & = qpqpq \\
        & = qpq \\
        & = q'
    \end{align*}

    \paragraph{Question 1*}
    Take $A = B = \mathbb{N}$ and define $f : A \rightarrow B$ by $f(x) = 2x$. Then $f$ has a retraction
    $r$ given by
    \[ r(y) = \begin{cases}
        y/2 & \text{if $y$ is even} \\
        0 & \text{if $y$ is odd}
    \end{cases} \]
    It has no section since it is not surjective (Article II Proposition 1).

    \chapter{Summary / quiz on pairs of 'opposed' maps}

    \paragraph{Question 1}
    Given two maps $f$, $g$ with domains and codomains as above, we can always form the composites
    $g \circ f$ and $f \circ g$. All we can say about $g \circ f$ and $f \circ g$ as maps in themselves is
    that they are endomaps.

    \paragraph{Question 2}
    If we know that $g$ is a retraction for $f$, that means $g \circ f$ is actually the identity map $1_A$;
    then we can prove that $f \circ g$ is not only an endomap, but actually an idempotent. The latter means
    that the equation $f \circ g \circ f \circ g = f \circ g$ is true.

    \paragraph{Question 3}
    If we even know that $f$ is an isomorphism \emph{and} that $g \circ f = 1_A$, then $f \circ g$ is not
    only an idempotent, but is the identity map $1_B$. If, moreover, $s$ is a map for which $f \circ s =
    1_B$, we can conclude that $s = g$.

    \paragraph{Question 4}
    Going back to 0, i.e. assuming no equations, but only the domain and codomain statements about $f$ and $g$,
    the composite $f \circ g \circ f$ could be different from $f$. Likewise $f \circ g \circ f \circ g$
    could be different from $f \circ g$.

    \chapter{Test 1}

    \paragraph{Question 1}
    \subparagraph{(a)}
    Let $f(Mara) = Aurelio$, $f(Aurelio) = Mara$ and $f(Andrea) = Andrea$.

    \subparagraph{(b)}
    Let $e(Mara) = Aurelio$, $e(Aurelio) = Aurelio$ and $e(Andrea) = Andrea$.

    \subparagraph{(c)}
    Let $B = \{ 0, 1 \}$. Define $s : B \rightarrow A$ by $s(0) = Aurelio$ and $s(1) = Andrea$.
    Define $r : A \rightarrow B$ by $r(Mara) = 0$, $r(Aurelio) = 0$ and $r(Andrea) = 1$.

    \paragraph{Question 2}
    Define $g : \mathbb{R} \rightarrow \mathbb{R}$ by $g(y) = (y+7)/4$.
    \subparagraph{(a)}
    $g(f(x)) = g(4x-7) = (4x-7+7)/4 = 4x/4 = x$
    \subparagraph{(b)}
    $f(g(x)) = f((x+7)/4) = 4((x+7)/4) - 7 = x+7-7 = x$

    \chapter{Session 10 --- Brouwer's Theorems}

    \section{4. Relation between fixed point and retraction theorems}

    \paragraph{Exercise 1}
    Suppose for a contradiction there is no point $x$ such that $f(x) = g(x)$. Define $r : D \rightarrow C$
    as follows: for $x \in D$, $r(x)$ is the point on $C$ that is pointed at by the arrow with tail at
    $f(x)$ and head at $g(x)$. For $x \in C$, we have $g(j(x)) = j(x)$, so the point that is pointed at by
    any arrow with head at $g(j(x))$ is $x$. Hence
    \begin{align*}
        r(j(x)) & = x
    \end{align*}
    and so $r$ is a retraction for $j$, contradicting the retraction theorem.

    \paragraph{Exercise 2}
    Let $f : A \rightarrow A$ be any endomap. Then $s \circ f \circ r : X \rightarrow X$ is an endomap on $X$.
    Hence there exists $x : T \rightarrow X$ such that $sfrx = x$. But then we have
    \begin{align*}
        rsfrx & = rx \\
        \therefore frx & = rx
    \end{align*}
    and so $r \circ x : T \rightarrow A$ is a fixed point of $f$.

    \paragraph{Exercise 3}
    Let $A$ be either $E$, $C$ or $S$, and $X$ be $I$, $D$ or $B$ respectively. Assume that every endomap $X \rightarrow X$ has a fixed point. 

    Assume for a contradiction that $X$ is a retract of $A$. By Exercise 2, every endomap
    on $A$ has a fixed point. This is a contradiction, as the antipodal map on $A$ has no fixed point.

    \section{7. Using maps to formulate guesses}

    \paragraph{Exercise 1}
    \subparagraph{(a)}
    We can express 'I start in Buffalo and end in Rochester' as $m \circ j = i \circ j$.

    We can express 'You start and finish anywhere between Buffalo and Rochester' as: there exists $f : I
    \rightarrow E$ such that $y \circ j = i \circ f$.

    \subparagraph{(b)}
    There exists $t : 1 \rightarrow I$ such that $mt = yt$.

    \subparagraph{(c)}
    Let $C$ be the circle, $D$ the disk and $P$ the plane. Let $j : C \rightarrow D$ and $i : D \rightarrow P$
    be the inclusions.

    For any maps $m, y : D \rightarrow P$ such that:
    \begin{itemize}
        \item $mj = ij$
        \item there exists $f : C \rightarrow D$ such that $yj = if$
    \end{itemize}
    then there exists $t : 1 \rightarrow D$ such that $mt = yt$.

    \subparagraph{(d)}
    I have not been able to find any smooth maps for which it is not true.

    \part{Part III --- Categories of Structured Sets}

    \chapter{Article III --- Examples of Categories}

    \section{1. The category $\mathcal{S}^\circ$ of endomaps of sets}

    \paragraph{Exercise 1}
    Let $f : (X, \alpha) \rightarrow (Y, \beta)$ and $g : (Y, \beta) \rightarrow (Z, \gamma)$. Then
    \[ g \circ f \circ \alpha = g \circ \beta \circ f = \gamma \circ g \circ f \]
    and so $g \circ f : (X, \alpha) \rightarrow (Z, \gamma)$.

    \section{4. Categories of endomaps}

    \paragraph{Exercise 2}
    Suppose $e : A \rightarrow A$ is idempotent and has a retraction $r : A \rightarrow A$. Then
    \[ 1_A = r \circ e = r \circ e \circ e = 1_A \circ e = e \]
    so $e = 1_A$. Thus, the identities are the only idempotents that have retractions.

    \paragraph{Exercise 3}
    Suppose $A$ has an even number of elements, say $\{ a_1, a_2, \ldots, a_{2n} \}$. Define $\theta : A
    \rightarrow A$ by $\theta(a_{2k+1}) = a_{2k+2}$ and $\theta(a_{2k+2}) = a_{2k+1}$ ($0 \leq k < n$). Then
    $\theta$ is an involution with no fixed point.

    Conversely, suppose $\theta : A \rightarrow A$ is an involution with no fixed point. Enumerate the elements
    of $A$ as follows: Pick any element $a_1 \in A$. Let $a_2 = \theta(a_1)$; then $a_1 = \theta(a_2)$.

    Assuming we have picked $a_1$, \ldots, $a_{2m}$ such that $\{ a_1, \ldots, a_{2m} \}$ is closed under
    $\theta$ and $A \neq \{ a_1, \ldots, a_{2m} \}$, pick $a_{2m+1} \in A - \{ a_1, \ldots, a_{2m} \}$.
    Then $\theta(a_{2m+1}) \notin \{ a_1, \ldots, a_{2m} \}$ (since $\theta(\theta(a_{2m+1})) = a_{2m+1}$)
    and $\theta(a_{2m+1}) \neq a_{2m+1}$ (since $\theta$ has no fixed point). So let
    $a_{2m+2} = \theta(a_{2m+1})$.

    This process must end because $A$ is finite. So $A = \{a_1, \ldots, a_{2n} \}$ for some $n$.

    Suppose now $A$ has an odd number of elements, say $A = \{ a_1, a_2, \ldots, a_{2n+1} \}$. Define $\theta :
    A \rightarrow A$ by
    \begin{align*}
        \theta(a_{2k+1}) & = a_{2k+2} & (0 \leq k < n) \\
        \theta(a_{2k+2}) & = a_{2k+1} & (0 \leq k < n) \\
        \theta(a_{2n+1}) & = a_{2n+1}
    \end{align*}
    Then $\theta$ is an involution whose only fixed point is $a_{2n+1}$.

    Conversely, suppose $\theta : A \rightarrow A$ is an involution with one fixed point $f$. Then
    $\theta \restriction (A - \{f\})$ is an involution on $A - \{ f \}$ with no fixed point. So $A - \{f\}$
    has an even number of elements, and so $A$ has an odd number of elements.

    \paragraph{Exercise 4}
    The map $\alpha$ is an involution because $-(-x) = x$. It is not idempotent because $-(-1) \neq -1$. Its
    only fixed point is 0.

    \paragraph{Exercise 5}
    The map $\alpha$ is not an involution because $||-1|| = 1 \neq -1$. It is idempotent because $||x|| = |x|$.
    Its fixed points are the non-negative integers.

    \paragraph{Exercise 6}
    The map $\alpha$ is an automorphism with inverse $\inv{\alpha}(x) = x - 3$.

    \paragraph{Exercise 7}
    The map $\alpha$ is not an automorphism because there is no integer $x$ with $\alpha(x) = 1$.

    \paragraph{Exercise 8}
    If $\alpha$ is idempotent then $\alpha \circ \alpha \circ \alpha = \alpha \circ \alpha = \alpha$.

    If $\alpha$ is an involution then $\alpha \circ \alpha \circ \alpha = 1 \circ \alpha = \alpha$.

    \paragraph{Exercise 9}
    Label the elements in the diagram 0, 1, 2 from top to bottom. Then
    \begin{align*}
        \alpha^3(0) & = \alpha^2(1) = \alpha(2) = 1 \\
        & = \alpha(0) \\
        \alpha^3(1) & = \alpha^2(2) = \alpha(1) = 2 \\
        & = \alpha(1) \\
        \alpha^3(2) & = \alpha^2(1) = \alpha(2) = 1 \\
        & = \alpha(2)
    \end{align*}

    Thus, $\alpha^3 = \alpha$.

    However, $\alpha$ is not idempotent because $\alpha^2(0) = 2 \neq \alpha(0)$. And $\alpha$ is not an involution
    because $\alpha^2(0) = 2 \neq 0$.

    \section{5. Irreflexive graphs}

    \paragraph{Exercise 10}
    \[ s(a) = k, s(b) = m, s(c) = k, s(d) = p, s(e) = m \]
    \[ t(a) = m, t(b) = m, t(c) = m, t(d) = q, t(e) = r \]

    The arrow $b$ has $s(b) = t(b)$. There is no arrow $x$ with $t(x) = k$.

    \paragraph{Exercise 11}
    We have
    \[ s'' \circ g \circ f = g \circ s' \circ f = g \circ f \circ s \]
    \[ t'' \circ g \circ f = g \circ t' \circ f = g \circ f \circ t \]
    and so $g \circ f : (X,P,s,t) \rightarrow (Z,R,s'',t'')$.

    \section{6. Endomaps as special graphs}
    
    \paragraph{Exericse 12}
    \[ I(g \circ f) = (g \circ f, g \circ f) = (g,g) \circ (f,f) = I(g) \circ I(f) \]

    \paragraph{Exercise 13}
    For any $x \in X$ we have $f_A(x) = 1_Y(f_A(x)) = f_D(1_X(x)) = f_D(x)$, and so $f_A = f_D$.
    Thus $(f_A, f_D) = I(f_A)$.

    \section{7. The simpler category $\mathcal{S}^\downarrow$: Objects are just maps of sets}

    \paragraph{Exercise 14}
    Let $X = \{ * \}$ and $Y = \{ 0, 1 \}$. Let $\alpha$ be the only map $X \rightarrow X$, and
    $\beta : Y \rightarrow Y$ be the map with $\beta(0) = 1$ and $\beta(1) = 0$. Let $f_A(*) = 0$
    and $f_D(*) = 1$. Then $f_D \circ \alpha = \beta \circ f_A$ but $f_A \neq f_D$.

    \section{8. Reflexive graphs}

    \paragraph{Exercise 15}
    Let $x_1 = s$ and $x_2 = t$, so $e_i = i x_i$ for each $i$. Then
    \begin{align*}
        e_k e_j & = i x_k i x_j \\
        & = i 1_P x_j \\
        & = i x_j \\
        & = e_j
    \end{align*}
    In particular, $e_j e_j = e_j$, so each $e_j$ is idempotent.

    \paragraph{Exercise 16}
    Let $(f_A, f_D) : (X, P, s, t, i) \rightarrow (Y, Q, s', t', j)$. Then
    \begin{align*}
        f_D s & = s' f_A \\
        \therefore f_D & = f_D s i \\
        & = s' f_A i
    \end{align*}

    \paragraph{Exercise 17}
    A map between $(M, F, \phi, \phi', \mu, \mu')$ and $(N, G, \psi, \psi', \nu, \nu')$
    is a pair of functions $f : M \rightarrow N$ and $g : F \rightarrow G$ such that
    \begin{align*}
        \psi f & = f \phi \\
        \psi' g & = f \phi' \\
        \nu g & = g \mu \\
        \nu' f & = g \nu
    \end{align*}
\end{document}
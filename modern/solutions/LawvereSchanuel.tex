\documentclass{report}

\title{Solutions Manual for Lawvere and Schanuel \emph{Conceptual Mathematics}}
\author{Robin Adams}

\usepackage{amsmath}
\usepackage{amssymb}
\usepackage{pf2}

\newcommand{\inv}[1]{\ensuremath{{#1}^{-1}}}

\begin{document}
    \maketitle
    \tableofcontents

    \part{Preview}
    \chapter{Session 1 --- Galileo and Multiplication of Objects}
    \paragraph{Exercise 1}
    Many examples --- every instance of a product in a category gives an example. I will not list them.

    \paragraph{Exercise 2}
    I am not entirely sure what solution the authors had in mind. Here are some that come to my mind:

    Place a spirit level between the two points and see if it reads as level.

    Place a smooth plank between the two points and see if a ball placed at one point rolls to the other,
    or \emph{vice versa}.

    Hang a plumbline at each point and see if they form a right angle with the line joining the two points.

    Of these, the third is my favourite.

    \part{Part I --- The category of sets}

    \chapter{Article I --- Sets, maps, composition}

    \paragraph{Exercise 1}
    Easy.

    \paragraph{Exercise 2}
    There are 8 maps from $A$ to $B$.

    \paragraph{Exercise 3}
    There are 27 maps from $A$ to $A$.

    \paragraph{Exercise 4}
    There are 9 maps from $B$ to $A$.

    \paragraph{Exercise 5}
    There are 4 maps from $B$ to $B$.

    \paragraph{Exercise 6}
    There are 10 such maps from $A$ to $A$.

    \paragraph{Exercise 7}
    There are 3 such maps from $B$ to $B$.

    \paragraph{Exercise 8}
    There is no such pair of maps.

    \paragraph{Exercise 9}
    There are 12 such pairs of maps.

    \chapter{Session 3 --- Composing maps and counting maps}

    \paragraph{Exercise 1}
    (a) and (c) make sense.

    \paragraph{Exercise 2}
    (a) and (c) still make sense.

    \part{Part II --- The algebra of composition}

    \chapter{Article II --- Isomorphisms}

    \section{1. Isomorphisms}

    \paragraph{Exercise 1}
    \subparagraph{(R)}
    We have $1_A \circ 1_A = 1_A$ by the Identity Laws, so $1_A$ is an isomorphism with inverse $1_A$.

    \subparagraph{(S)}
    We have $g \circ f = 1_A$ and $f \circ g = 1_B$ (this is what it means for $g$ to be an inverse for $f$).
    This says exactly that $f$ is an inverse for $g$.

    \subparagraph{(T)}
    Let $\inv{f} : B \rightarrow A$ be an inverse for $f$ and $\inv{k} : C \rightarrow B$ be an inverse for $k$.
    We prove $\inv{f} \circ \inv{k}$ is an inverse for $k \circ f$. We have
    \begin{align*}
        \inv{f} \circ \inv{k} \circ k \circ f & = \inv{f} \circ 1_B \circ f & (\text{definition of inverse}) \\
        & = \inv{f} \circ f & (\text{Identity Law})\\
        & = 1_A & (\text{definition of inverse})
    \end{align*}
    and $k \circ f \circ \inv{f} \circ \inv{k} = 1_C$ similarly.

    \paragraph{Exercise 2}
    We have
    \begin{align*}
        g & = g \circ 1_B & (\text{Identity Law}) \\
        & = g \circ f \circ k & (\text{$k$ is an inverse of $f$}) \\
        & = 1_A \circ k & (\text{$g$ is an inverse of $f$}) \\
        & = k & (\text{Identity Law})
    \end{align*}

    \paragraph{Exercise 3}
    \subparagraph{(a)}
    Let $f : A \rightarrow B$. Let $h, k : C \rightarrow A$.

    Suppose $f \circ h = f \circ k$. Then
    \begin{align*}
        \inv{f} \circ f \circ h & = \inv{f} \circ f \circ k \\
        \therefore 1_A \circ h & = 1_A \circ k & (\text{Definition of inverse}) \\
        \therefore h & = k & (\text{Identity Law}) \qed
    \end{align*}

    \subparagraph{(b)}
    Let $f : A \rightarrow B$. Let $h, k : B \rightarrow C$.

    Suppose $h \circ f = k \circ f$. Then
    \begin{align*}
        h \circ f \circ \inv{f} & = k \circ f \circ \inv{f} \\
        \therefore h \circ 1_B & = k \circ 1_B & (\text{Definition of inverse}) \\
        \therefore h & = k & (\text{Identity Law}) \qed
    \end{align*}

    \subparagraph{(c)}
    Let $A = \{ 0,1 \}$. Define $f : A \rightarrow A$ by $f(0) = 1$ and $f(1) = 0$. Define $h : A \rightarrow A$
    by $h(x) = 0$ for all $x$. Define $k : A \rightarrow A$ by $k(x) = 1$ for all $x$.

    $f$ is invertible, and is its own inverse.

    We have $h \circ f = f \circ k = h$.

    We do not have $h = k$.

    \paragraph{Exercise 4}
    \subparagraph{(1)}
    This function is invertible with inverse $\inv{f}(x) = (x-7)/3$.

    \subparagraph{(2)}
    This function is invertible with inverse $\inv{g}(x) = \sqrt{x}$.

    \subparagraph{(3)}
    This function is not invertible because $h(1) = h(-1) = 1$.

    \subparagraph{(4)}
    This function is not invertible because $k(1) = k(-1) = 1$.

    \subparagraph{(5)}
    This function is not invertible because there is no $x$ such that $l(x) = 2$.

    \section{2 --- General division problems: Determination and choice}

    \paragraph{Exercise 5}
    There are 6 maps $f$ such that $g \circ f = 1_{\{0,1\}}$; we can map 0 to any of $b$, $p$ or $q$,
    and 1 to either of $r$ or $s$.

    Given any one of these maps $f$, there are 8 maps $g$ such that $g \circ f = 1_{\{0,1\}}$.
    We must map $f(0)$ to $0$, $f(1)$ to $1$, and the other three elements to any of 0 or 1.

    \paragraph{Exercise 6}
    If $r : B \rightarrow A$ is a section for $f$, then we take $t = g \circ r$. We have
    $t \circ f = g \circ r \circ f = g \circ 1_A = g$.

    \paragraph{Exercise 7}
    Let $s : B \rightarrow A$ be a section for $f$. Let $T$ be any set and $t_1, t_2 : T \rightarrow B$.
    Suppose $t_1 \circ f = t_2 \circ f$. Then
    \begin{align*}
        t_1 \circ f \circ s & = t_2 \circ f \circ s \\
        \therefore t_1 \circ 1_B & = t_2 \circ 1_B \\
        \therefore t_1 & = t_2
    \end{align*}

    \paragraph{Exercise 8}
    If $s_1 : B \rightarrow A$ is a section for $r_1 : A \rightarrow B$ and $s_2 : C \rightarrow B$
    is a section for $r_2 : B \rightarrow C$, then $s_1 \circ s_2$ is a section for $r_2 \circ r_1$ since
    \begin{align*}
        r_2 \circ r_1 \circ s_1 \circ s_2 & = r_2 \circ 1_B \circ s_2 \\
        & = r_2 \circ s_2 \\
        & = 1_C
    \end{align*}

    \paragraph{Exercise 9}
    We have
    \begin{align*}
        e \circ e & = f \circ r \circ f \circ r \\
        & = f \circ 1 \circ r & (\text{$r$ is a retraction of $f$}) \\
        & = f \circ r \\
        & = e
    \end{align*}

    \paragraph{Exercise 10}
    From the proof of Proposition 3, $\inv{f} \circ \inv{g}$ is both a section and a retraction for
    $g \circ f$.

    \paragraph{Exercise 11}
    Set $f(Fatima) = coffee$, $f(Omer) = tea$ and $f(Alysia) = cocoa$. Then $f$ is an isomorphism.

    There is no isomorphism $g : A \rightarrow C$. For if $g(Fatima) = true$ then $g(Omer)$ must be $false$,
    and then it is impossible to choose a value for $g(Alysia)$ without having $g(Alysia) = g(Fatima)$
    or $g(Alysia) = g(Omer)$. Similarly if $g(Fatima) = false$ then $g(Omer)$ must be $true$, and then
    again we cannot choose a value for $g(Alysia)$.
\end{document}
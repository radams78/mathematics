\documentclass{report}

\title{Solutions Manual for Lawvere and Schanuel \emph{Conceptual Mathematics}}
\author{Robin Adams}

\usepackage{amsmath}
\usepackage{amssymb}
\usepackage{pf2}

\newcommand{\inv}[1]{\ensuremath{{#1}^{-1}}}

\begin{document}
    \maketitle
    \tableofcontents

    \part{Preview}
    \chapter{Session 1 --- Galileo and Multiplication of Objects}
    \paragraph{Exercise 1}
    Many examples --- every instance of a product in a category gives an example. I will not list them.

    \paragraph{Exercise 2}
    I am not entirely sure what solution the authors had in mind. Here are some that come to my mind:

    Place a spirit level between the two points and see if it reads as level.

    Place a smooth plank between the two points and see if a ball placed at one point rolls to the other,
    or \emph{vice versa}.

    Hang a plumbline at each point and see if they form a right angle with the line joining the two points.

    Of these, the third is my favourite.

    \part{Part I --- The category of sets}

    \chapter{Article I --- Sets, maps, composition}

    \paragraph{Exercise 1}
    Easy.

    \paragraph{Exercise 2}
    There are 8 maps from $A$ to $B$.

    \paragraph{Exercise 3}
    There are 27 maps from $A$ to $A$.

    \paragraph{Exercise 4}
    There are 9 maps from $B$ to $A$.

    \paragraph{Exercise 5}
    There are 4 maps from $B$ to $B$.

    \paragraph{Exercise 6}
    There are 10 such maps from $A$ to $A$.

    \paragraph{Exercise 7}
    There are 3 such maps from $B$ to $B$.

    \paragraph{Exercise 8}
    There is no such pair of maps.

    \paragraph{Exercise 9}
    There are 12 such pairs of maps.

    \chapter{Session 3 --- Composing maps and counting maps}

    \paragraph{Exercise 1}
    (a) and (c) make sense.

    \paragraph{Exercise 2}
    (a) and (c) still make sense.

    \part{Part II --- The algebra of composition}

    \chapter{Article II --- Isomorphisms}

    \section{1. Isomorphisms}

    \paragraph{Exercise 1}
    \subparagraph{(R)}
    We have $1_A \circ 1_A = 1_A$ by the Identity Laws, so $1_A$ is an isomorphism with inverse $1_A$.

    \subparagraph{(S)}
    We have $g \circ f = 1_A$ and $f \circ g = 1_B$ (this is what it means for $g$ to be an inverse for $f$).
    This says exactly that $f$ is an inverse for $g$.

    \subparagraph{(T)}
    Let $\inv{f} : B \rightarrow A$ be an inverse for $f$ and $\inv{k} : C \rightarrow B$ be an inverse for $k$.
    We prove $\inv{f} \circ \inv{k}$ is an inverse for $k \circ f$. We have
    \begin{align*}
        \inv{f} \circ \inv{k} \circ k \circ f & = \inv{f} \circ 1_B \circ f & (\text{definition of inverse}) \\
        & = \inv{f} \circ f & (\text{Identity Law})\\
        & = 1_A & (\text{definition of inverse})
    \end{align*}
    and $k \circ f \circ \inv{f} \circ \inv{k} = 1_C$ similarly.

    \paragraph{Exercise 2}
    We have
    \begin{align*}
        g & = g \circ 1_B & (\text{Identity Law}) \\
        & = g \circ f \circ k & (\text{$k$ is an inverse of $f$}) \\
        & = 1_A \circ k & (\text{$g$ is an inverse of $f$}) \\
        & = k & (\text{Identity Law})
    \end{align*}

    \paragraph{Exercise 3}
    \subparagraph{(a)}
    Let $f : A \rightarrow B$. Let $h, k : C \rightarrow A$.

    Suppose $f \circ h = f \circ k$. Then
    \begin{align*}
        \inv{f} \circ f \circ h & = \inv{f} \circ f \circ k \\
        \therefore 1_A \circ h & = 1_A \circ k & (\text{Definition of inverse}) \\
        \therefore h & = k & (\text{Identity Law}) \qed
    \end{align*}

    \subparagraph{(b)}
    Let $f : A \rightarrow B$. Let $h, k : B \rightarrow C$.

    Suppose $h \circ f = k \circ f$. Then
    \begin{align*}
        h \circ f \circ \inv{f} & = k \circ f \circ \inv{f} \\
        \therefore h \circ 1_B & = k \circ 1_B & (\text{Definition of inverse}) \\
        \therefore h & = k & (\text{Identity Law}) \qed
    \end{align*}

    \subparagraph{(c)}
    Let $A = \{ 0,1 \}$. Define $f : A \rightarrow A$ by $f(0) = 1$ and $f(1) = 0$. Define $h : A \rightarrow A$
    by $h(x) = 0$ for all $x$. Define $k : A \rightarrow A$ by $k(x) = 1$ for all $x$.

    $f$ is invertible, and is its own inverse.

    We have $h \circ f = f \circ k = h$.

    We do not have $h = k$.

    \paragraph{Exercise 4}
    \subparagraph{(1)}
    This function is invertible with inverse $\inv{f}(x) = (x-7)/3$.

    \subparagraph{(2)}
    This function is invertible with inverse $\inv{g}(x) = \sqrt{x}$.

    \subparagraph{(3)}
    This function is not invertible because $h(1) = h(-1) = 1$.

    \subparagraph{(4)}
    This function is not invertible because $k(1) = k(-1) = 1$.

    \subparagraph{(5)}
    This function is not invertible because there is no $x$ such that $l(x) = 2$.

    \section{2 --- General division problems: Determination and choice}

    \paragraph{Exercise 5}
    There are 6 maps $f$ such that $g \circ f = 1_{\{0,1\}}$; we can map 0 to any of $b$, $p$ or $q$,
    and 1 to either of $r$ or $s$.

    Given any one of these maps $f$, there are 8 maps $g$ such that $g \circ f = 1_{\{0,1\}}$.
    We must map $f(0)$ to $0$, $f(1)$ to $1$, and the other three elements to any of 0 or 1.

    \paragraph{Exercise 6}
    If $r : B \rightarrow A$ is a section for $f$, then we take $t = g \circ r$. We have
    $t \circ f = g \circ r \circ f = g \circ 1_A = g$.

    \paragraph{Exercise 7}
    Let $s : B \rightarrow A$ be a section for $f$. Let $T$ be any set and $t_1, t_2 : T \rightarrow B$.
    Suppose $t_1 \circ f = t_2 \circ f$. Then
    \begin{align*}
        t_1 \circ f \circ s & = t_2 \circ f \circ s \\
        \therefore t_1 \circ 1_B & = t_2 \circ 1_B \\
        \therefore t_1 & = t_2
    \end{align*}

    \paragraph{Exercise 8}
    If $s_1 : B \rightarrow A$ is a section for $r_1 : A \rightarrow B$ and $s_2 : C \rightarrow B$
    is a section for $r_2 : B \rightarrow C$, then $s_1 \circ s_2$ is a section for $r_2 \circ r_1$ since
    \begin{align*}
        r_2 \circ r_1 \circ s_1 \circ s_2 & = r_2 \circ 1_B \circ s_2 \\
        & = r_2 \circ s_2 \\
        & = 1_C
    \end{align*}

    \paragraph{Exercise 9}
    We have
    \begin{align*}
        e \circ e & = f \circ r \circ f \circ r \\
        & = f \circ 1 \circ r & (\text{$r$ is a retraction of $f$}) \\
        & = f \circ r \\
        & = e
    \end{align*}

    \paragraph{Exercise 10}
    From the proof of Proposition 3, $\inv{f} \circ \inv{g}$ is both a section and a retraction for
    $g \circ f$.

    \paragraph{Exercise 11}
    Set $f(Fatima) = coffee$, $f(Omer) = tea$ and $f(Alysia) = cocoa$. Then $f$ is an isomorphism.

    There is no isomorphism $g : A \rightarrow C$. For if $g(Fatima) = true$ then $g(Omer)$ must be $false$,
    and then it is impossible to choose a value for $g(Alysia)$ without having $g(Alysia) = g(Fatima)$
    or $g(Alysia) = g(Omer)$. Similarly if $g(Fatima) = false$ then $g(Omer)$ must be $true$, and then
    again we cannot choose a value for $g(Alysia)$.

    \chapter{Session 4 --- Division of Maps: Isomorphisms}

    \section{4. A small zoo of isomorphisms in other categories}

    \paragraph{Exercise 1}
    We have $h(d(x)) = h(2x) = x$ and $d(h(x)) = d(x/2) = x$ for any $x$.

    \paragraph{Exercise 2}
    $f(odd) = negative$ and $f(even) = positive$

    \paragraph{Exercise 3}
    \subparagraph{(a)}
    This is not an isomorphism because $p(0 + 0) = 1$ but $p(0) + p(0) = 2$

    \subparagraph{(b)}
    This is not an isomorphism because it is not surjective; there is no $x$ such that $sq(x) = -1$.

    \subparagraph{(c)}
    This is not an isomorphism because it is not injective. We have $sq(1) = sq(-1) = 1$.

    \subparagraph{(d)}
    This is an isomorphism; it is bijective and $-(x+y) = (-x) + (-y)$.

    \subparagraph{(e)}
    This is not an isomorphism because $m(1 \times 1) = -1$ but $m(1) \times m(1) = 1$.

    \subparagraph{(f)}
    This is not a well-defined map because $c(-1) = -1 \notin \mathbb{R}_{>0}$.

    \chapter{Session 5 --- Division of Maps: Sections and Retractions}

    \section{1. Determination Problems}

    \paragraph{Exercise 1}
    \subparagraph{(a)}
    Suppose $h = g \circ f$ and $f a_1 = f a_2$. Then $h a_1 = g(f a_1) = g(f a_2) = h a_2$.

    \subparagraph{(b)}
    No. Take $A = C = \emptyset$ and $B = \{ * \}$. Let $f : A \rightarrow B$ and $h : A \rightarrow C$
    be the unique such maps. Vacuously, if $f a_1 = f a_2$ then $h a_1 = h a_2$. But there is no map
    $g : B \rightarrow C$.

    \section{3. Choice Problems}

    \paragraph{Exercise 2}
    \subparagraph{(a)}
    Suppose $g \circ f = h$. Let $a \in A$. Let $b = f(a)$. Then $h(a) = g(f(a)) = g(b)$.

    \subparagraph{(b)}
    This is equivalent to the Axiom of Choice.

    \section{5. Stacking or Sorting}

    \paragraph{Exercise 3}
    I'm not going to draw all of them, but there are 8 of them.

    \chapter{Session 9 --- Retracts and Idempotents}

    \section{1. Retracts and Comparisons}

    \paragraph{Exercise 1}
    If $A$ is empty, then the nowhere-defined function is a map $A \rightarrow B$.

    If $B$ has a point, say $b$, then the constant map with value $b$ is a map $A \rightarrow B$.

    \section{2. Idempotents as records of retracts}

    \paragraph{Exercise 3}
    Suppose $s : A \rightarrow B$, $r : B \rightarrow A$ and $s' : A' \rightarrow B$, $r' : B \rightarrow A'$
    are splittings of $e : B \rightarrow B$. Let
    \begin{align*}
        f & = r' \circ s & : A \rightarrow A' \\
        \inv{f} & = r \circ s' & : A' \rightarrow A
    \end{align*}
    Then we have
    \begin{align*}
        f \circ \inv{f} & = r' \circ s \circ r \circ s' \\
        & = r' \circ e \circ s' \\
        & = r' \circ s' \circ r' \circ s' \\
        & = 1 \\
        \inv{f} \circ f & = r \circ s' \circ r' \circ s \\
        & = r \circ e \circ s \\
        & = r \circ s \circ r \circ s \\
        & = 1
    \end{align*}

    \chapter{Quiz}

    \paragraph{Question 1}
    Let $A = \{ * \}$ and $B = \{ 0,1 \}$. Define $f : A \rightarrow B$ by $f(*) = 0$. Then the unique
    function $r : B \rightarrow A$ is a retraction for $f$ (since $r(f(*)) = *$) but not a section for $f$
    (since $f(s(1)) = 0$). Therefore there is no section for $f$, since there is only one map $B \rightarrow A$.

    \paragraph{Question 2}
    \subparagraph{(a)}
    Yes: if $pqp = p$ then $pqpq = pq$
    \subparagraph{(b)}
    Yes: if $pqp = p$ then $qpqp = qp$

    \paragraph{Question 2*}
    Let $q' = qpq$ Then we have
    \begin{align*}
        pq'p & = pqpqp \\
        & = pqp \\
        & = p \\
        q'pq' & = qpqpqpq \\
        & = qpqpq \\
        & = qpq \\
        & = q'
    \end{align*}

    \paragraph{Question 1*}
    Take $A = B = \mathbb{N}$ and define $f : A \rightarrow B$ by $f(x) = 2x$. Then $f$ has a retraction
    $r$ given by
    \[ r(y) = \begin{cases}
        y/2 & \text{if $y$ is even} \\
        0 & \text{if $y$ is odd}
    \end{cases} \]
    It has no section since it is not surjective (Article II Proposition 1).

    \chapter{Summary / quiz on pairs of 'opposed' maps}

    \paragraph{Question 1}
    Given two maps $f$, $g$ with domains and codomains as above, we can always form the composites
    $g \circ f$ and $f \circ g$. All we can say about $g \circ f$ and $f \circ g$ as maps in themselves is
    that they are endomaps.

    \paragraph{Question 2}
    If we know that $g$ is a retraction for $f$, that means $g \circ f$ is actually the identity map $1_A$;
    then we can prove that $f \circ g$ is not only an endomap, but actually an idempotent. The latter means
    that the equation $f \circ g \circ f \circ g = f \circ g$ is true.

    \paragraph{Question 3}
    If we even know that $f$ is an isomorphism \emph{and} that $g \circ f = 1_A$, then $f \circ g$ is not
    only an idempotent, but is the identity map $1_B$. If, moreover, $s$ is a map for which $f \circ s =
    1_B$, we can conclude that $s = g$.

    \paragraph{Question 4}
    Going back to 0, i.e. assuming no equations, but only the domain and codomain statements about $f$ and $g$,
    the composite $f \circ g \circ f$ could be different from $f$. Likewise $f \circ g \circ f \circ g$
    could be different from $f \circ g$.
\end{document}
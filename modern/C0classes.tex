\documentclass{article}

\title{C0 Classes}
\author{Robin Adams}

\usepackage{amsmath}
\usepackage{amssymb}
\usepackage{amsthm}
\let\proof\relax
\let\endproof\relax
\let\qed\relax
\usepackage{pf2}
\usepackage[all]{xy}

\newtheorem{axiom}{Axiom}
\newtheorem{axs}[axiom]{Axiom Schema}
\newtheorem{lm}[axiom]{Lemma}
\newtheorem{proposition}[axiom]{Proposition}
\newtheorem{props}[axiom]{Proposition Schema}
\newtheorem{thm}[axiom]{Theorem}
\newtheorem{cor}{Corollary}[axiom]
\theoremstyle{definition}
\newtheorem{definition}[axiom]{Definition}

\begin{document}
    \maketitle

    We speak informally of \emph{classes}. A class is determined by a unary predicate. We write 
    $\{ x : P(x) \}$ or $\{ x \mid P(x) \}$ for the class determined by the predicate $P(x)$.

    We define what it means for an object $a$ to be an element of the class $\mathbf{A}$, $a \in \mathbf{A}$,
    by: $a \in \{ x : P(x) \}$ means $P(a)$.

    We write $\{ x \in \mathbf{A} : P(x) \}$ for $\{ x : x \in \mathbf{A} \wedge P(x) \}$,
    and $\{ t[x_1, \ldots, x_n] : P[x_1, \ldots, x_n] \}$ for $\{ y : \exists x_1 \cdots \exists x_n
    (y = t[x_1, \ldots, x_n] \wedge P[x_1, \ldots, x_n]) \}$.

    \begin{definition}[Equality of Classes]
        Two classes $\mathbf{A}$ and $\mathbf{B}$ are \emph{equal}, $\mathbf{A} = \mathbf{B}$,
        iff they have exactly the same members.
    \end{definition}

    \begin{proposition}
        For any class $\mathbf{A}$ we have $\mathbf{A} = \mathbf{A}$.
    \end{proposition}

    \begin{proof}
        Since $\mathbf{A}$ and $\mathbf{A}$ have exactly the same members.
    \end{proof}

    \begin{proposition}
        For any classes $\mathbf{A}$ and $\mathbf{B}$, if $\mathbf{A} = \mathbf{B}$ then $\mathbf{B} =
        \mathbf{A}$.
    \end{proposition}

    \begin{proof}
        \pf\ If $\mathbf{A}$ and $\mathbf{B}$ have exactly the same members, then $\mathbf{B}$
        and $\mathbf{A}$ have exactly the same members.
    \end{proof}

    \begin{proposition}
        For any classes $\mathbf{A}$, $\mathbf{B}$ and $\mathbf{C}$, if $\mathbf{A} = \mathbf{B}$
        and $\mathbf{B} = \mathbf{C}$ then $\mathbf{A} = \mathbf{C}$.
    \end{proposition}

    \begin{proof}
        \pf\ If $\mathbf{A}$ and $\mathbf{B}$ have exactly the same members,
        and $\mathbf{B}$ and $\mathbf{C}$ have exactly the same members,
        then $\mathbf{A}$ and $\mathbf{C}$ have exactly the same members. \qed
    \end{proof}

    \begin{definition}[Subclass]
        A class $\mathbf{A}$ is a \emph{subclass} of a class $\mathbf{B}$,
        $\mathbf{A} \subseteq \mathbf{B}$, iff every member of $\mathbf{A}$
        is a member of $\mathbf{B}$.
    \end{definition}

    \begin{proposition}
        For any class $\mathbf{A}$ we have $\mathbf{A} \subseteq \mathbf{A}$.
    \end{proposition}

    \begin{proof}
        \pf\ Every member of $\mathbf{A}$ is a member of $\mathbf{A}$. \qed
    \end{proof}

    \begin{proposition}
        For any classes $\mathbf{A}$, $\mathbf{B}$ and $\mathbf{C}$, if $\mathbf{A} \subseteq \mathbf{B}$
        and $\mathbf{B} \subseteq \mathbf{C}$ then $\mathbf{A} \subseteq \mathbf{C}$.
    \end{proposition}

    \begin{proof}
        \pf\ If every member of $\mathbf{A}$ is a member of $\mathbf{B}$,
        and every member of $\mathbf{B}$ is a member of $\mathbf{C}$,
        then every member of $\mathbf{A}$ is a member of $\mathbf{C}$. \qed
    \end{proof}

    \begin{proposition}
        For any classes $\mathbf{A}$ and $\mathbf{B}$, if $\mathbf{A} \subseteq \mathbf{B}$
        and $\mathbf{B} \subseteq \mathbf{A}$ then $\mathbf{A} = \mathbf{B}$.
    \end{proposition}

    \begin{proof}
        \pf\ If every member of $\mathbf{A}$ is a member of $\mathbf{B}$,
        and every member of $\mathbf{B}$ is a member of $\mathbf{A}$,
        then $\mathbf{A}$ and $\mathbf{B}$ have exactly the same members. \qed
    \end{proof}

    \begin{definition}[Empty Class]
        The \emph{empty class}, $\emptyset$, is $\{ x : \bot \}$.
    \end{definition}

    \begin{proposition}
        For any class $\mathbf{A}$, we have $\emptyset \subseteq \mathbf{A}$.
    \end{proposition}

    \begin{proof}
        \pf\ Vacuously, every member of $\emptyset$ is a member of $\mathbf{A}$.
    \end{proof}

    \begin{definition}[Universal Class]
        The \emph{universal class} $\mathbf{V}$ is the class $\{ x : \top \}$.
    \end{definition}

    \begin{proposition}
        For any class $\mathbf{A}$, we have $\mathbf{A} \subseteq \mathbf{V}$.
    \end{proposition}

    \begin{proof}
        \pf\ Every member of $\mathbf{A}$ is a member of $\mathbf{V}$. \qed
    \end{proof}

    \begin{definition}
        For any objects $a_1$, \ldots, $a_n$, we write $\{ a_1, \ldots, a_n \}$ for the class
        $\{ x : x = a_1 \vee \cdots \vee x = a_n \}$.

        A class of the form $\{a\}$ is called a \emph{singleton}.

        A class of the form $\{a,b\}$ is called a \emph{pair class}.
    \end{definition}

    \begin{definition}[Union]
        The \emph{union} of classes $\mathbf{A}$ and $\mathbf{B}$, 
        $\mathbf{A} \cup \mathbf{B}$, is the set whose elements are exactly the things
        that are members of $\mathbf{A}$ or members of $\mathbf{B}$.
    \end{definition}

    \begin{proposition}
        For any classes $\mathbf{A}$, $\mathbf{B}$ and $\mathbf{C}$, we have:
        \begin{enumerate}
            \item $\mathbf{A} \subseteq \mathbf{A} \cup \mathbf{B}$
            \item $\mathbf{B} \subseteq \mathbf{A} \cup \mathbf{B}$
            \item If $\mathbf{A} \subseteq \mathbf{C}$ and $\mathbf{B} \subseteq \mathbf{C}$
            then $\mathbf{A} \cup \mathbf{B} \subseteq \mathbf{C}$
        \end{enumerate}
    \end{proposition}

    \begin{proof}
        \pf\ Immediate from definitions. \qed
    \end{proof}

    \begin{proposition}
        For any classes $\mathbf{A}$ and $\mathbf{B}$ we have $\mathbf{A} \cup \mathbf{B}
        = \mathbf{B} \cup \mathbf{A}$.
    \end{proposition}

    \begin{proof}
        \pf\ They are each the class of objects that belong to either $\mathbf{A}$ or $\mathbf{B}$. \qed
    \end{proof}

    \begin{proposition}
        For any classes $\mathbf{A}$, $\mathbf{B}$ and $\mathbf{C}$ we have
        $\mathbf{A} \cup (\mathbf{B} \cup \mathbf{C}) = (\mathbf{A} \cup \mathbf{B}) \cup \mathbf{C}$.
    \end{proposition}

    \begin{proof}
        \pf\ They are each the class of objects that belong to at least one of $\mathbf{A}$,
        $\mathbf{B}$ or $\mathbf{C}$. \qed
    \end{proof}

    \begin{proposition}
        For any class $\mathbf{A}$ we have $\mathbf{A} \cup \emptyset = \mathbf{A}$.
    \end{proposition}

    \begin{proof}
        \pf\ Immediate from definitions. \qed
    \end{proof}

    \begin{proposition}
        If $\mathbf{A} \subseteq \mathbf{B}$ then $\mathbf{A} \cup \mathbf{C} \subseteq \mathbf{B}
        \cup \mathbf{C}$.
    \end{proposition}

    \begin{proof}
        \pf\ Easy. \qed
    \end{proof}

    \begin{definition}[Intersection]
        The \emph{intersection} of classes $\mathbf{A}$ and $\mathbf{B}$, 
        $\mathbf{A} \cap \mathbf{B}$, is the set whose elements are exactly the
        things that are members of both $\mathbf{A}$ and $\mathbf{B}$.
    \end{definition}

    \begin{proposition}
        For any classes $\mathbf{A}$, $\mathbf{B}$ and $\mathbf{C}$, we have:
        \begin{enumerate}
            \item $\mathbf{A} \cap \mathbf{B} \subseteq \mathbf{A}$
            \item $\mathbf{A} \cap \mathbf{B} \subseteq \mathbf{B}$
            \item If $\mathbf{C} \subseteq \mathbf{A}$ and $\mathbf{C} \subseteq \mathbf{B}$
            then $\mathbf{C} \subseteq \mathbf{A} \cap \mathbf{B}$
        \end{enumerate}
    \end{proposition}

    \begin{proof}
        \pf\ Immediate from definitions. \qed
    \end{proof}

    \begin{proposition}
        For any classes $\mathbf{A}$ and $\mathbf{B}$ we have $\mathbf{A} \cap \mathbf{B}
        = \mathbf{B} \cap \mathbf{A}$.
    \end{proposition}

    \begin{proof}
        \pf\ They are each the class of objects that belong to both $\mathbf{A}$ and $\mathbf{B}$. \qed
    \end{proof}

    \begin{proposition}
        For any classes $\mathbf{A}$, $\mathbf{B}$ and $\mathbf{C}$ we have
        $\mathbf{A} \cap (\mathbf{B} \cap \mathbf{C}) = (\mathbf{A} \cap \mathbf{B}) \cap \mathbf{C}$.
    \end{proposition}

    \begin{proof}
        \pf\ They are each the class of objects that belong to all of $\mathbf{A}$,
        $\mathbf{B}$ and $\mathbf{C}$. \qed
    \end{proof}

    \begin{proposition}
        For any classes $\mathbf{A}$, $\mathbf{B}$ and $\mathbf{C}$, we have
        $\mathbf{A} \cup (\mathbf{B} \cap \mathbf{C}) = (\mathbf{A} \cup \mathbf{B}) \cap
        (\mathbf{A} \cup \mathbf{C})$.
    \end{proposition}

    \begin{proof}
        \pf
        \begin{align*}
            x \in \mathbf{A} \cup (\mathbf{B} \cap \mathbf{C})
            & \Leftrightarrow x \in \mathbf{A} \vee (x \in \mathbf{B} \wedge x \in \mathbf{C}) \\
            & \Leftrightarrow (x \in \mathbf{A} \vee x \in \mathbf{B}) \wedge (x \in \mathbf{A} \vee x \in \mathbf{C}) \\
            & \Leftrightarrow x \in (\mathbf{A} \cup \mathbf{B}) \cap (\mathbf{A} \cup \mathbf{B})
        \end{align*}
    \end{proof}

    \begin{proposition}
        For any classes $\mathbf{A}$, $\mathbf{B}$ and $\mathbf{C}$, we have
        $\mathbf{A} \cap (\mathbf{B} \cup \mathbf{C}) = (\mathbf{A} \cap \mathbf{B}) \cup
        (\mathbf{A} \cap \mathbf{C})$.
    \end{proposition}

    \begin{proof}
        \pf\ Dual. \qed
    \end{proof}

    \begin{proposition}
        For any class $\mathbf{A}$ we have $\mathbf{A} \cap \emptyset = \emptyset$.
    \end{proposition}

    \begin{proof}
        \pf\ Immediate from definitions. \qed
    \end{proof}

    \begin{proposition}
        If $\mathbf{A} \subseteq \mathbf{B}$ then $\mathbf{A} \cap \mathbf{C} \subseteq \mathbf{B}
        \cap \mathbf{C}$.
    \end{proposition}

    \begin{proof}
        \pf\ Easy. \qed
    \end{proof}

    \begin{definition}[Disjoint]
        Two classes $\mathbf{A}$ and $\mathbf{B}$ are \emph{disjoint} iff they have no common members.
    \end{definition}

    \begin{proposition}
        Two classes $\mathbf{A}$ and $\mathbf{B}$ are \emph{disjoint} iff $\mathbf{A} \cap \mathbf{B}
        = \emptyset$.
    \end{proposition}

    \begin{proof}
        \pf\ Immediate from definitions. \qed
    \end{proof}

    \begin{definition}[Relative Complement]
        Given classes $\mathbf{A}$ and $\mathbf{B}$, the \emph{relative complement}
        $\mathbf{A} - \mathbf{B}$ is $\{ x \in \mathbf{A} : x \notin \mathbf{B} \}$.
    \end{definition}
    
    \begin{proposition}[De Morgan's Law]
        For any classes $\mathbf{A}$, $\mathbf{B}$ and $\mathbf{C}$, we have
        $\mathbf{C} - (\mathbf{A} \cup \mathbf{B}) = (\mathbf{C} - \mathbf{A}) \cap
        (\mathbf{C} - \mathbf{B})$.
    \end{proposition}

    \begin{proof}
        \pf\ They are each the set of objects that belong to $\mathbf{C}$ and not to $\mathbf{A}$
        nor $\mathbf{B}$. \qed
    \end{proof}

    \begin{proposition}[De Morgan's Law]
        For any classes $\mathbf{A}$, $\mathbf{B}$ and $\mathbf{C}$, we have
        $\mathbf{C} - (\mathbf{A} \cap \mathbf{B}) = (\mathbf{C} - \mathbf{A}) \cup
        (\mathbf{C} - \mathbf{B})$.
    \end{proposition}

    \begin{proof}
        \pf\ Dual. \qed
    \end{proof}

    \begin{proposition}
        For any classes $\mathbf{A}$ and $\mathbf{C}$ we have $\mathbf{A} \cap (\mathbf{C} - \mathbf{A})
        = \emptyset$.
    \end{proposition}

    \begin{proof}
        \pf\ Immediate from definitions. \qed
    \end{proof}

    \begin{proposition}
        If $\mathbf{A} \subseteq \mathbf{B}$ then $\mathbf{C} - \mathbf{B} \subseteq \mathbf{C} - \mathbf{A}$.
    \end{proposition}

    \begin{proof}
        \pf\ Easy. \qed
    \end{proof}

\end{document}
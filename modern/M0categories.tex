\documentclass{article}

\title{M0 Categories}
\author{Robin Adams}

\usepackage{amsmath}
\usepackage{amssymb}
\usepackage{amsthm}
\let\proof\relax
\let\endproof\relax
\let\qed\relax
\usepackage{pf2}
\usepackage[all]{xy}

\newtheorem{axiom}{Axiom}
\newtheorem{axs}[axiom]{Axiom Schema}
\newtheorem{lm}[axiom]{Lemma}
\newtheorem{proposition}[axiom]{Proposition}
\newtheorem{props}[axiom]{Proposition Schema}
\newtheorem{thm}[axiom]{Theorem}
\newtheorem{cor}{Corollary}[axiom]
\theoremstyle{definition}
\newtheorem{definition}[axiom]{Definition}

\newcommand{\inv}[1]{\ensuremath{{#1}^{-1}}}

\begin{document}

\maketitle
\tableofcontents

\section{Categories}

\begin{definition}[Category]
    A \emph{category} consists of:
    \begin{itemize}
        \item a collection of \emph{objects}.
        \item for any objects $A$ and $B$, a collection of \emph{maps} from $A$ to $B$. We write $f : A \rightarrow B$
        iff $f$ is a map from $A$ to $B$.
        \item for any object $A$, an \emph{identity map} $1_A : A \rightarrow A$
        \item for any maps $f : A \rightarrow B$ and $g : B \rightarrow C$, a map $g \circ f : A \rightarrow C$
    \end{itemize}
    such that:
    \begin{description}
        \item[Identity Laws]
        For any map $f : A \rightarrow B$, we have $1_B \circ f = f \circ 1_A = f : A \rightarrow B$
        \item[Associative Law]
        For any maps $f : A \rightarrow B$, $g : B \rightarrow C$ and $h : C \rightarrow D$, we hav
        $h \circ (g \circ f) = (h \circ g) \circ f : A \rightarrow D$ 
    \end{description}
\end{definition}

\begin{definition}
    A map $f : A \rightarrow B$ is \emph{monic} or a \emph{monomorphism}, $f : A \rightarrowtail B$,
    iff, for every object $T$ and morphisms $x_1, x_2 : T \rightarrow B$, if $f \circ x_1 = f \circ x_2$
    then $x_1 = x_2$.
\end{definition}

\begin{definition}
    A map $f : A \rightarrow B$ is \emph{epi} or an \emph{epimorphism}, $f : A \twoheadrightarrow B$,
    iff, for every object $T$ and morphisms $x_1, x_2 : B \rightarrow T$, if $x_1 \circ f = x_2 \circ f$
    then $x_1 = x_2$.
\end{definition}

\begin{definition}[Retraction, Section]
    Let $r : A \rightarrow B$ and $s : B \rightarrow A$. Then $r$ is a \emph{retraction} for $s$, and $s$ is a
    \emph{section} for $r$, iff $r \circ s = 1_B$.

    The object $A$ is a \emph{retract} of $B$ iff there exists a retraction $r : B \rightarrow A$,
    i.e. there exist maps $s : A \rightarrow B$ and $r : B \rightarrow A$ such that $r \circ s = 1_A$.
\end{definition}

\begin{proposition}
    If a map $f : A \rightarrow B$ has a section, then for any object $T$ and any map $y : T \rightarrow B$,
    there exists a map $x : T \rightarrow A$ such that $f \circ x = y$.
\end{proposition}

\begin{proof}
    \pf\ If $s : B \rightarrow A$ is a section of $f$, then we take $x = s \circ y$. We have
    $f \circ x = f \circ s \circ y = 1_B \circ y = y$. \qed
\end{proof}

\begin{proposition}
    If a map $f : A \rightarrow B$ has a retraction, then for any object $T$ and any map $g : A \rightarrow T$,
    there exists a map $t : B \rightarrow T$ such that $t \circ f = g$.
\end{proposition}

\begin{proof}
    \pf\ If $r : B \rightarrow A$ is a section for $f$, then we take $t = g \circ r$. We have
    $t \circ f = g \circ r \circ f = g \circ 1_A = g$. \qed
\end{proof}

\begin{proposition}
    Every section is monic.
\end{proposition}

\begin{proof}
    \pf\ Let $r : B \rightarrow A$ be a retraction for $f$. Then, if $f \circ x_1 = f \circ x_2$, then
    \begin{align*}
        r \circ f \circ x_1 & = r \circ f \circ x_2 \\
        \therefore 1_A \circ x_1 & = 1_A \circ x_2 \\
        \therefore x_1 & = x_2 & \qed
    \end{align*}
\end{proof}

\begin{proposition}
    Every retraction is epi.
\end{proposition}

\begin{proof}
    \pf\     Let $s : B \rightarrow A$ be a section for $f : A \rightarrow B$. Let $T$ be any set and $t_1, t_2 : T \rightarrow B$.
    Suppose $t_1 \circ f = t_2 \circ f$. Then
    \begin{align*}
        t_1 \circ f \circ s & = t_2 \circ f \circ s \\
        \therefore t_1 \circ 1_B & = t_2 \circ 1_B \\
        \therefore t_1 & = t_2
    \end{align*}
\end{proof}

\begin{proposition}
    For any object $A$, the identity map $1_A$ is a section and a retraction of itself.
\end{proposition}

\begin{proof}
    \pf\ The Unit Laws give $1_A \circ 1_A = 1_A$. \qed
\end{proof}

\begin{cor}
    Every object is a retract of itself.
\end{cor}

\begin{proposition}
    \label{proposition:retraction_comp}
    If $r_1 : B \rightarrow A$ is a retraction of $s_1 : A \rightarrow B$ and $r_2 : C \rightarrow B$
    is a retraction of $s_2 : B \rightarrow C$ then $r_1 \circ r_2$ is a retraction of $s_2 \circ s_1$.
\end{proposition}

\begin{proof}
    \pf
    \begin{align*}
        r_1 \circ r_2 \circ s_2 \circ s_1 & = r_1 \circ 1_B \circ s_1 \\
        & = r_1 \circ s_1 \\
        & = 1_A & \qed
    \end{align*}
\end{proof}

\begin{cor}
    If the object $A$ is a retract of $B$ and $B$ is a retract of $C$ then $A$ is a retract of $C$.
\end{cor}

\begin{thm}
    \label{thm:inverse_unique}
    If $r$ is a retraction of $f$ and $s$ is a section of $f$ then $r = s$.
\end{thm}

\begin{proof}
    \pf\ Let $f : A \rightarrow B$ and $r, s : B \rightarrow A$. Then
    \begin{align*}
        r & = r \circ 1_B \\
        & = r \circ f \circ s \\
        & = 1_A \circ s \\
        & = s & \qed
    \end{align*}
\end{proof}

\begin{definition}[Isomorphism]
    A map $f : A \rightarrow B$ is an \emph{isomorphism} or \emph{invertible}, $f : A \cong B$, iff there
    exists a map $f^{-1} : B \rightarrow A$, the \emph{inverse} for $f$, such that $f^{-1} \circ f = 1_A$ and
    $f \circ f^{-1} = 1_B$.
    
    Two objects $A$ and $B$ are \emph{isomorphic}, $A \cong B$, iff there exists an isomorphism between them.
\end{definition}

\begin{thm}
    The inverse of an isomorphism is unique.
\end{thm}

\begin{proof}
    \pf\ From Theorem \ref{thm:inverse_unique}. \qed
\end{proof}

\begin{thm}
    For any object $A$, the identity map $1_A : A \cong A$ is an isomorphism with $1_A^{-1} = 1_A$.
\end{thm}

\begin{proof}
    \pf\ We have $1_A \circ 1_A = 1_A$ by the Identity Laws. \qed
\end{proof}

\begin{thm}
    If $f : A \cong B$ then $\inv{f} : B \cong A$ and $\inv{(\inv{f})} = f$.
\end{thm}

\begin{proof}
    \pf\ Since $f \circ \inv{f} = 1_B$ and $\inv{f} \circ f = 1_A$ by the definition of inverse. \qed
\end{proof}

\begin{thm}
    If $f : A \cong B$ and $g : B \cong C$ then $g \circ f : A \cong C$ and
    $\inv{(g \circ f)} = \inv{f} \circ \inv{g}$.
\end{thm}

\begin{proof}
    \pf\ From Proposition \ref{proposition:retraction_comp}. \qed
\end{proof}

\begin{proposition}
    Every monomorphic retraction is an isomorphism.
\end{proposition}

\begin{proof}
    \pf\ Let $f : A \rightarrowtail B$ be a monomorphism with section $s : B \rightarrow A$. Then
    \begin{align*}
        f \circ s \circ f & = f \\
        \therefore s \circ f & = 1_A
    \end{align*}
    Thus $s$ is also a retraction for $f$, hence an inverse. \qed
\end{proof}

\begin{proposition}
    Every epimorphic section is an isomorphism.
\end{proposition}

\begin{proof}
    \pf\ Dual. \qed
\end{proof}

\begin{definition}[Idempotent]
    A map $e : A \rightarrow A$ is \emph{idempotent} iff $e \circ e = e$.
\end{definition}

\begin{definition}[Split Idempotent]
    Let $e : A \rightarrow A$ be idempotent. A \emph{splitting} of $e$ consists of an object $B$ and maps
    $s : B \rightarrow A$, $r : A \rightarrow B$ such that $r \circ s = 1_B$ and $s \circ r = e$.
\end{definition}

\begin{definition}[Automorphism]
    An \emph{automorphism} on an object $A$ is an isomorphism $A \cong A$.
\end{definition}

\section{Terminal Objects}

\begin{definition}[Terminal Object]
    An object $1$ is \emph{terminal} iff, for every object $X$, there exists exactly one morphism $X \rightarrow 1$.
\end{definition}

\begin{thm}
    If $T_1$ and $T_2$ are terminal objects, then the unique map $T_1 \rightarrow T_2$ is iso.
\end{thm}

\begin{proof}
    \pf\ Let $f : T_1 \rightarrow T_2$ be the unique such map, and $g : T_2 \rightarrow T_1$ the unique map
    in the other direction. The $g \circ f = 1_{T_1}$ since there is only one map $T_1 \rightarrow T_1$,
    and $f \circ g = 1_{T_2}$ since there is only one map $T_2 \rightarrow T_2$. \qed
\end{proof}

\section{Initial Objects}

\begin{definition}[Initial Object]
    An object $0$ is \emph{initial} iff, for every object $X$, there exists exactly one morphism 
    $0 \rightarrow X$.
\end{definition}

\section{Products}

\begin{definition}[Product]
    Let $A$ and $B$ be objects. A \emph{product} of $A$ and $B$ consists of an object $A \times B$ and
    morphisms $\pi_1 : A \times B \rightarrow A$, $\pi_2 : A \times B \rightarrow B$, the \emph{projections},
    such that, for any object $X$ and morphisms $f : X \rightarrow A$ and $g : X \rightarrow B$, there exists
    exactly one map $\langle f, g \rangle : X \rightarrow A \times B$ such that
    $\pi_1 \circ \langle f,g \rangle = f$ and $\pi_2 \circ \langle f, g \rangle = g$.
\end{definition}

\begin{thm}[Lawvere Diagonal Theorem]
    Let $\mathcal{C}$ be a category with finite products.

    Let $Y$ and $T$ be objects and $f : T \times T \rightarrow Y$. Suppose that, for every $g : T \rightarrow Y$,
    there exists a point $\overline{g} : 1 \rightarrow T$ such that $f \circ \langle 1_T, \overline{g} \circ ! \rangle = g$.
    Then, for every $\alpha : Y \rightarrow Y$, there exists $y : 1 \rightarrow Y$ such that $\alpha y = y$.
\end{thm}

\begin{proof}
    \pf
    \step{1}{\pflet{$\alpha : Y \rightarrow Y$}}
    \step{2}{\pflet{$g = \alpha \circ f \circ \langle 1_T, 1_T \rangle$}}
    \step{3}{\pick\ $t_0 : 1 \rightarrow T$ such that $f \circ \langle 1_T, t_0 \circ ! \rangle = \alpha \circ f
    \circ \langle 1_T, 1_T \rangle$}
    \step{4}{\pflet{$y = f \circ \langle t_0, t_0 \rangle$}}
    \step{5}{$\alpha y = y$}
    \begin{proof}
        \pf
        \begin{align*}
            \alpha y & = \alpha f \langle t_0, t_0 \rangle \\
            & = \alpha f \langle 1_T, 1_T \rangle t_0 \\
            & = f \langle 1_T, t ! \rangle t_0 \\
            & = f \langle t_0, t_0 \rangle \\
            & = y
        \end{align*}
    \end{proof}
    \qed
\end{proof}

\section{Sums}

\begin{definition}[Sum]
    Let $A$ and $B$ be objects. A \emph{sum} of $A$ and $B$ consists of an object $A + B$ and
    morphisms $\kappa_1 : A \rightarrow A + B$, $\kappa_2 : B \rightarrow A + B$, the \emph{injections},
    such that, for any object $X$ and morphisms $f : A \rightarrow X$ and $g : B \rightarrow X$, there exists
    exactly one map $[ f, g ] : A + B \rightarrow X$ such that
    $[f,g] \circ \kappa_1 = f$ and $[f,g] \circ \kappa_2 = g$.
\end{definition}

\section{Distributive Categories}

\begin{definition}[Distributive Category]
    Let $\mathcal{C}$ be a category with binary products and binary coproducts. Then $\mathcal{C}$ is
    \emph{distributive} iff, for any objects $A$, $B$, $C$, the map
    \[ [1_A \times \kappa_1, 1_A \times \kappa_2] : (A \times B) + (A \times C) \rightarrow A \times (B + C) \]
    is an isomorphism.
\end{definition}

\section{Equalizers}

\begin{definition}[Equalizer]
    Let $f, g : A \rightarrow B$. An \emph{equalizer} of $f$ and $g$ consists of an object $E$ and a morphism
    $e : E \rightarrow A$ such that $f \circ e = g \circ e$ and, for any object $X$ and morphism $x : X
    \rightarrow A$ such that $fx = gx$, there exists a unique $\overline{x} : X \rightarrow E$ such that
    $x = e \overline{x}$.
\end{definition}

\section{Map Objects}

\begin{definition}[Map Objects]
    Let $\mathcal{C}$ be a category with binary products.

    Let $A$ and $B$ be objects. A \emph{map object} from $A$ to $B$ consists of an object $B^A$ and a morphism
    $e : B^A \times A \rightarrow B$ such that, for any object $X$ and morphism $f : X \times A \rightarrow B$,
    there exists a unique morphism $\lambda f : X \rightarrow B^A$ such that $e \circ (\lambda f \times 1_A)
    = f$.
\end{definition}

\begin{proposition}
    Let $\mathcal{C}$ be a category with binary products and coproducts. Let $T$ be an object such that,
    for every object $A$, the map object $A^T$ exists. Then binary products with $T$ distribute over binary
    coproducts.
\end{proposition}

\begin{proof}
    \pf
    \step{1}{\pflet{$A, B \in \mathcal{C}$}}
    \step{2}{\pflet{$c = [\kappa_1 \times 1_T, \kappa_2 \times 1_T]: 
        (A \times T) + (B \times T) \rightarrow (A + B) \times T$}}
    \step{4}{$\lambda \kappa_1 : A \rightarrow ((A \times T) + (B \times T))^T$,
    $\lambda \kappa_2 : B \rightarrow ((A \times T) + (B \times T))^T$,
    and they are unique such that $e \circ (\lambda \kappa_1 \times 1_T) = \kappa_1$
    and $e \circ (\lambda \kappa_2 \times 1_T) = \kappa_2$}
    \step{5}{$[\lambda \kappa_1, \lambda \kappa_2] : A + B \rightarrow ((A \times T) + (B \times T))^T$
    is unique such that $e \circ ([\lambda \kappa_1, \lambda \kappa_2] \kappa_1 \times 1_T) = \kappa_1$
    and $e \circ ([\lambda \kappa_1, \lambda \kappa_2] \kappa_2 \times 1_T) = \kappa_2$}
    \step{7}{\pflet{$\inv{c} = e \circ ([\lambda \kappa_1, \lambda \kappa_2] \times 1_T)
    : (A + B) \times T \rightarrow (A \times T) + (B \times T)$}}
    \step{8}{$c \inv{c} = 1_{(A + B) \times T}$}
    \begin{proof}
        \step{a}{$\pi_1 c \inv{c} = \pi_1 : (A + B) \times T \rightarrow A + B$}
        \begin{proof}
            \step{i}{$\lambda(\pi_1 c \inv{c}) = \lambda \pi_1 : A + B \rightarrow (A + B)^T$}
            \begin{proof}
                \step{one}{$\lambda (\pi_1 c \inv{c}) \kappa_1 = (\lambda \pi_1) \kappa_1 : A \rightarrow (A + B)^T$}
                \begin{proof}
                    \step{I}{$e \circ (\lambda (\pi_1 c \inv{c}) \kappa_1 \times 1_T) =
                    e \circ ((\lambda \pi_1) \kappa_1 \times 1_T) : A \times T \rightarrow A + B$}
                    \begin{proof}
                        \pf
                        \begin{align*}
                            e \circ (\lambda (\pi_1 c \inv{c}) \kappa_1 \times 1_T)
                            & = (e \circ (\lambda (\pi_1 c \inv{c}) \times 1_T)(\kappa_1 \times 1_T) \\
                            & = \pi_1 c \inv{c} (\kappa_1 \times 1_T) \\
                            & = \pi_1 c \kappa_1 & (\text{\stepref{5}}) \\
                            & = \pi_1 (\kappa_1 \times 1_T) \\
                            & = \kappa_1 \pi_1 \\
                            & = \pi_1 (\kappa_1 \times 1_T) \\
                            & = e (\lambda \pi_1 \times 1_T) (\kappa_1 \times 1_T) \\
                            & = e ((\lambda \pi_1) \kappa_1 \times 1_T))
                        \end{align*}
                    \end{proof}
                \end{proof}
                \step{two}{$\lambda (\pi_1 c \inv{c}) \kappa_2 = (\lambda \pi_1) \kappa_2 : B \rightarrow (A + B)^T$}
                \begin{proof}
                    \pf\ Similar.
                \end{proof}
            \end{proof}
        \end{proof}
        \step{b}{$\pi_2 c \inv{c} = \pi_2$}
        \begin{proof}
            \pf\ Similar
        \end{proof}
    \end{proof}
    \step{9}{$\inv{c} c = 1_{(A \times T) + (B \times T)}$}
    \begin{proof}
        \pf\ $\inv{c} c = [\kappa_1, \kappa_2] = 1$
    \end{proof}
    \qed
\end{proof}

\begin{definition}[Cartesian closed category]
    A \emph{Cartesian closed category} is a category with finite products and map objects.
\end{definition}

\begin{thm}[Cantor's Diagonal Argument]
    Let $\mathcal{C}$ be a Cartesian closed category. Let $T, Y \in \mathcal{C}$ and $f : T \rightarrow Y^T$.
    Suppose that, for every map $g : T \rightarrow Y$, there exists $x : 1 \rightarrow T$ such that
    $\lambda g = fx$. Then every endomorphism on $Y$ has a fixed point.
\end{thm}

\begin{proof}
    \pf
    \step{1}{\pflet{$\alpha : Y \rightarrow Y$}}
    \step{2}{\pflet{$g = \alpha \circ e \circ \langle f, 1_T \rangle : T \rightarrow Y$}}
    \step{3}{\pick\ $x : 1 \rightarrow T$ such that $\lambda (g \pi_2) = fx$}
    \step{4}{\pflet{$y = e \circ \langle fx, x \rangle : 1 \rightarrow Y$}}
    \step{5}{$\alpha y = y$}
    \begin{proof}
        \pf
        \begin{align*}
            \alpha y & = \alpha e \langle f, 1_T \rangle x \\
            & = gx \\
            & = g \pi_2 \langle 1_1, x \rangle \\
            & = e (\lambda(g \pi_1) \times 1) \langle 1_1, x \rangle \\
            & = e (fx \times 1) \langle 1_1, x \rangle \\
            & = e \langle fx, x \rangle \\
            & = y
        \end{align*}
    \end{proof}
    \qed
\end{proof}

\section{Pullbacks}

\begin{definition}[Pullback]
    The diagram below is a \emph{pullback} iff $fp = gq$ and, for any object $X$ and morphisms $x : X \rightarrow B$
    and $y : X \rightarrow C$ such that $fx = gy$, there exists a unique morphism $m : X \rightarrow A$ such that $pm = x$
    and $qm = y$.
    \[ \xymatrix{
        A \ar[r]^p \ar[d]_q & B \ar[d]^{f} \\
        C \ar[r]_g & D
    } \]
\end{definition}

\section{Subobject Classifier}

\begin{definition}[Subobject Classifier]
    Let $\mathcal{C}$ be a category with a terminal object 1.
    A \emph{subobject classifier} consists of an object $\Omega$ and morphism $\top : 1 \rightarrow \Omega$
    such that, for every monomorphism $m : A \rightarrowtail B$, there exists a unique morphism
    $\chi_m : B \rightarrow \Omega$, the \emph{characterisetic morphism} of $m$, such that the following
    diagram is a pullback
    \[ \xymatrix{
        A \ar[r]^{!} \ar[d]_{m} & 1 \ar[d]^{\top} \\
        B \ar[r]_{\chi_m} & \Omega
    } \]
\end{definition}

\section{Toposes}

\begin{definition}[Topos]
    A \emph{topos} is a Cartesian closed category $\mathcal{C}$ with finite coproducts and a subobject
    classifier such that, for every object $X$, the slice category $\mathcal{C} / X$ has finite products.
\end{definition}

\end{document}
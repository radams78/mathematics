\documentclass{article}

\title{M0 Categories}
\author{Robin Adams}

\usepackage{amsmath}
\usepackage{amssymb}
\usepackage{amsthm}
\let\proof\relax
\let\endproof\relax
\let\qed\relax
\usepackage{pf2}
\usepackage[all]{xy}

\newtheorem{axiom}{Axiom}
\newtheorem{axs}[axiom]{Axiom Schema}
\newtheorem{lm}[axiom]{Lemma}
\newtheorem{proposition}[axiom]{Proposition}
\newtheorem{props}[axiom]{Proposition Schema}
\newtheorem{thm}[axiom]{Theorem}
\newtheorem{cor}{Corollary}[axiom]
\theoremstyle{definition}
\newtheorem{definition}[axiom]{Definition}

\newcommand{\inv}[1]{\ensuremath{{#1}^{-1}}}

\begin{document}

\maketitle
\tableofcontents

\section{Categories}

\begin{definition}[Category]
    A \emph{category} consists of:
    \begin{itemize}
        \item a collection of \emph{objects}.
        \item for any objects $A$ and $B$, a collection of \emph{maps} from $A$ to $B$. We write $f : A \rightarrow B$
        iff $f$ is a map from $A$ to $B$.
        \item for any object $A$, an \emph{identity map} $1_A : A \rightarrow A$
        \item for any maps $f : A \rightarrow B$ and $g : B \rightarrow C$, a map $g \circ f : A \rightarrow C$
    \end{itemize}
    such that:
    \begin{description}
        \item[Identity Laws]
        For any map $f : A \rightarrow B$, we have $1_B \circ f = f \circ 1_A = f : A \rightarrow B$
        \item[Associative Law]
        For any maps $f : A \rightarrow B$, $g : B \rightarrow C$ and $h : C \rightarrow D$, we hav
        $h \circ (g \circ f) = (h \circ g) \circ f : A \rightarrow D$ 
    \end{description}
\end{definition}

\begin{definition}
    A map $f : A \rightarrow B$ is \emph{monic} or a \emph{monomorphism}, $f : A \rightarrowtail B$,
    iff, for every object $T$ and morphisms $x_1, x_2 : T \rightarrow B$, if $f \circ x_1 = f \circ x_2$
    then $x_1 = x_2$.
\end{definition}

\begin{definition}
    A map $f : A \rightarrow B$ is \emph{epi} or an \emph{epimorphism}, $f : A \twoheadrightarrow B$,
    iff, for every object $T$ and morphisms $x_1, x_2 : B \rightarrow T$, if $x_1 \circ f = x_2 \circ f$
    then $x_1 = x_2$.
\end{definition}

\begin{definition}[Retraction, Section]
    Let $r : A \rightarrow B$ and $s : B \rightarrow A$. Then $r$ is a \emph{retraction} for $s$, and $s$ is a
    \emph{section} for $r$, iff $r \circ s = 1_B$.

    The object $A$ is a \emph{retract} of $B$ iff there exists a retraction $r : B \rightarrow A$,
    i.e. there exist maps $s : A \rightarrow B$ and $r : B \rightarrow A$ such that $r \circ s = 1_A$.
\end{definition}

\begin{proposition}
    If a map $f : A \rightarrow B$ has a section, then for any object $T$ and any map $y : T \rightarrow B$,
    there exists a map $x : T \rightarrow A$ such that $f \circ x = y$.
\end{proposition}

\begin{proof}
    \pf\ If $s : B \rightarrow A$ is a section of $f$, then we take $x = s \circ y$. We have
    $f \circ x = f \circ s \circ y = 1_B \circ y = y$. \qed
\end{proof}

\begin{proposition}
    If a map $f : A \rightarrow B$ has a retraction, then for any object $T$ and any map $g : A \rightarrow T$,
    there exists a map $t : B \rightarrow T$ such that $t \circ f = g$.
\end{proposition}

\begin{proof}
    \pf\ If $r : B \rightarrow A$ is a section for $f$, then we take $t = g \circ r$. We have
    $t \circ f = g \circ r \circ f = g \circ 1_A = g$. \qed
\end{proof}

\begin{proposition}
    Every section is monic.
\end{proposition}

\begin{proof}
    \pf\ Let $r : B \rightarrow A$ be a retraction for $f$. Then, if $f \circ x_1 = f \circ x_2$, then
    \begin{align*}
        r \circ f \circ x_1 & = r \circ f \circ x_2 \\
        \therefore 1_A \circ x_1 & = 1_A \circ x_2 \\
        \therefore x_1 & = x_2 & \qed
    \end{align*}
\end{proof}

\begin{proposition}
    Every retraction is epi.
\end{proposition}

\begin{proof}
    \pf\     Let $s : B \rightarrow A$ be a section for $f : A \rightarrow B$. Let $T$ be any set and $t_1, t_2 : T \rightarrow B$.
    Suppose $t_1 \circ f = t_2 \circ f$. Then
    \begin{align*}
        t_1 \circ f \circ s & = t_2 \circ f \circ s \\
        \therefore t_1 \circ 1_B & = t_2 \circ 1_B \\
        \therefore t_1 & = t_2
    \end{align*}
\end{proof}

\begin{proposition}
    For any object $A$, the identity map $1_A$ is a section and a retraction of itself.
\end{proposition}

\begin{proof}
    \pf\ The Unit Laws give $1_A \circ 1_A = 1_A$. \qed
\end{proof}

\begin{cor}
    Every object is a retract of itself.
\end{cor}

\begin{proposition}
    \label{proposition:retraction_comp}
    If $r_1 : B \rightarrow A$ is a retraction of $s_1 : A \rightarrow B$ and $r_2 : C \rightarrow B$
    is a retraction of $s_2 : B \rightarrow C$ then $r_1 \circ r_2$ is a retraction of $s_2 \circ s_1$.
\end{proposition}

\begin{proof}
    \pf
    \begin{align*}
        r_1 \circ r_2 \circ s_2 \circ s_1 & = r_1 \circ 1_B \circ s_1 \\
        & = r_1 \circ s_1 \\
        & = 1_A & \qed
    \end{align*}
\end{proof}

\begin{cor}
    If the object $A$ is a retract of $B$ and $B$ is a retract of $C$ then $A$ is a retract of $C$.
\end{cor}

\begin{thm}
    \label{thm:inverse_unique}
    If $r$ is a retraction of $f$ and $s$ is a section of $f$ then $r = s$.
\end{thm}

\begin{proof}
    \pf\ Let $f : A \rightarrow B$ and $r, s : B \rightarrow A$. Then
    \begin{align*}
        r & = r \circ 1_B \\
        & = r \circ f \circ s \\
        & = 1_A \circ s \\
        & = s & \qed
    \end{align*}
\end{proof}

\begin{definition}[Isomorphism]
    A map $f : A \rightarrow B$ is an \emph{isomorphism} or \emph{invertible}, $f : A \cong B$, iff there
    exists a map $f^{-1} : B \rightarrow A$, the \emph{inverse} for $f$, such that $f^{-1} \circ f = 1_A$ and
    $f \circ f^{-1} = 1_B$.
    
    Two objects $A$ and $B$ are \emph{isomorphic}, $A \cong B$, iff there exists an isomorphism between them.
\end{definition}

\begin{thm}
    The inverse of an isomorphism is unique.
\end{thm}

\begin{proof}
    \pf\ From Theorem \ref{thm:inverse_unique}. \qed
\end{proof}

\begin{thm}
    For any object $A$, the identity map $1_A : A \cong A$ is an isomorphism with $1_A^{-1} = 1_A$.
\end{thm}

\begin{proof}
    \pf\ We have $1_A \circ 1_A = 1_A$ by the Identity Laws. \qed
\end{proof}

\begin{thm}
    If $f : A \cong B$ then $\inv{f} : B \cong A$ and $\inv{(\inv{f})} = f$.
\end{thm}

\begin{proof}
    \pf\ Since $f \circ \inv{f} = 1_B$ and $\inv{f} \circ f = 1_A$ by the definition of inverse. \qed
\end{proof}

\begin{thm}
    If $f : A \cong B$ and $g : B \cong C$ then $g \circ f : A \cong C$ and
    $\inv{(g \circ f)} = \inv{f} \circ \inv{g}$.
\end{thm}

\begin{proof}
    \pf\ From Proposition \ref{proposition:retraction_comp}. \qed
\end{proof}

\begin{proposition}
    Every monomorphic retraction is an isomorphism.
\end{proposition}

\begin{proof}
    \pf\ Let $f : A \rightarrowtail B$ be a monomorphism with section $s : B \rightarrow A$. Then
    \begin{align*}
        f \circ s \circ f & = f \\
        \therefore s \circ f & = 1_A
    \end{align*}
    Thus $s$ is also a retraction for $f$, hence an inverse. \qed
\end{proof}

\begin{proposition}
    Every epimorphic section is an isomorphism.
\end{proposition}

\begin{proof}
    \pf\ Dual. \qed
\end{proof}

\begin{definition}[Idempotent]
    A map $e : A \rightarrow A$ is \emph{idempotent} iff $e \circ e = e$.
\end{definition}

\begin{definition}[Split Idempotent]
    Let $e : A \rightarrow A$ be idempotent. A \emph{splitting} of $e$ consists of an object $B$ and maps
    $s : B \rightarrow A$, $r : A \rightarrow B$ such that $r \circ s = 1_B$ and $s \circ r = e$.
\end{definition}

\begin{definition}[Automorphism]
    An \emph{automorphism} on an object $A$ is an isomorphism $A \cong A$.
\end{definition}
\end{document}
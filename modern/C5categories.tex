\documentclass{article}

\title{M5 Categories}
\author{Robin Adams}

\usepackage{amsmath}
\usepackage{amssymb}
\usepackage{amsthm}
\let\proof\relax
\let\endproof\relax
\let\qed\relax
\usepackage{pf2}
\usepackage[all]{xy}

\newtheorem{axiom}{Axiom}[section]
\newtheorem{axs}[axiom]{Axiom Schema}
\newtheorem{lemma}[axiom]{Lemma}
\newtheorem{proposition}[axiom]{Proposition}
\newtheorem{props}[axiom]{Proposition Schema}
\newtheorem{theorem}[axiom]{Theorem}
\newtheorem{corollary}{Corollary}[axiom]
\theoremstyle{definition}
\newtheorem{definition}[axiom]{Definition}
\newtheorem{example}[axiom]{Example}

\newcommand{\inv}[1]{\ensuremath{{#1}^{-1}}}
\newcommand{\id}[1]{\ensuremath{\mathrm{id}_{#1}}}
\newcommand{\dom}{\ensuremath{\operatorname{dom}}}
\newcommand{\cod}{\ensuremath{\operatorname{cod}}}

\begin{document}

\maketitle

\begin{definition}[Category]
    A \emph{category} $\mathcal{C}$ is a sextuple $(Ar, Ob, \dom, \cod, \id, m)$ such
    that:
    \begin{itemize}
        \item $\dom : Ar \rightarrow Ob$
        \item $\cod : Ar \rightarrow Ob$
        \item $\id : Ob \rightarrow Ar$
        \item $m : \{ (f,g) \in Ar^2 : \dom f = \cod g \} \rightarrow Ar$
    \end{itemize}
    such that:
    \begin{itemize}
        \item $\forall A \in Ob. \dom (\id A) = A$
        \item $\forall A \in Ob. \cod (\id A) = A$
        \item $\forall f,g \in Ar. (\dom f = \cod g \Rightarrow \dom m(f,g) = \dom g)$
        \item $\forall f,g \in Ar. (\dom f = \cod g \Rightarrow \cod m(f,g) = \cod f)$
        \item $\forall f \in Ar. m(\id (\cod f), f) = f$
        \item $\forall f \in Ar. m(f, \id (\dom f)) = f$
        \item $\forall f,g,h \in Ar. (\dom f = \cod g \wedge \dom g = \cod h
        \Rightarrow m(f,m(g,h)) = m(m(f,g),h))$
    \end{itemize}
    We call $Ar$ the \emph{arrows} of $\mathcal{C}$ and $Ob$ the \emph{objects}.
    We call $\dom f$ the \emph{domain} of the arrow $f$, and $\cod f$ the
    \emph{codomain}. We write $f : A \rightarrow B$ for $\dom f = A \wedge \cod f = B$.

    We say arrows $f$ and $g$ are \emph{composable} iff $\dom f = \cod g$, in which
    case $m(f,g)$ is called their \emph{composite}, written $f \circ g$.

    We call $\id A$ the \emph{identity arrow} on $A$.
\end{definition}

\begin{definition}[Category of Sets]
    The \emph{category of sets} $\mathbf{Set}$ is the category with set of objects
    $\mathbf{V}$ and with $\mathbf{Set}[A,B]$ the set of all functions from $A$ to $B$.
\end{definition}

\begin{definition}[Preordered Sets as Categories]
    Identify any preordered set $(P,\leq)$ with the category $P$ with set of objects $P$
    and a morphism $(a,b) : a \rightarrow b$ iff $a \leq b$.
\end{definition}

\begin{definition}[Discrete Category]
    For any set $A$, the \emph{discrete category} $A$ is the preordered set $(A, =)$
    considered as a category.
\end{definition}

\begin{definition}[Slice Category]
    Let $\mathbb{C}$ be a category and $A \in \mathbb{C}$. The \emph{slice category}
    $\mathbb{C} / A$ is the category with:
    \begin{itemize}
        \item objects all pairs $(B,f)$ where $B \in \mathbb{C}$ and $f : B \rightarrow A$
        in $\mathbb{C}$
        \item morphisms $(B,f) \rightarrow (C,g)$ all morphisms $h : B \rightarrow C$
        in $\mathbb{C}$ such that $g \circ h = f$.
    \end{itemize}
\end{definition}

\begin{definition}[Slice Category]
    Let $\mathbb{C}$ be a category and $A \in \mathbb{C}$. The \emph{coslice category}
    $\mathbb{C} \backslash A$ is the category with:
    \begin{itemize}
        \item objects all pairs $(B,f)$ where $B \in \mathbb{C}$ and $f : A \rightarrow B$
        in $\mathbb{C}$
        \item morphisms $(B,f) \rightarrow (C,g)$ all morphisms $h : B \rightarrow C$
        in $\mathbb{C}$ such that $g = h \circ f$.
    \end{itemize}
\end{definition}

\begin{definition}[Pointed Sets]
    The category $\mathbf{Set}_*$ of \emph{pointed sets} is the coslice category
    $\mathbf{Set} \backslash 1$.
\end{definition}

\begin{definition}[Subcategory]
    A category $\mathbb{C}$ is a \emph{subcategory} of $\mathbb{D}$ iff every object
    of $\mathbb{C}$ is an object of $\mathbb{D}$, and every morphism $A \rightarrow B$
    in $\mathbb{C}$ is a morphism $A \rightarrow B$ in $\mathbb{D}$.

    It is \emph{full} iff, for all $A, B \in \mathbb{C}$, every morphism $A \rightarrow B$
    in $\mathbb{D}$ is a morphism $A \rightarrow B$ in $\mathbb{C}$.
\end{definition}

\begin{proposition}
    Let $f, g : A \rightarrow B$. Then $f = g$ if and only if, for every object $X$
    and arrow $x : X \rightarrow A$, we have $f \circ x = g \circ x$.
\end{proposition}

\begin{proof}
    \pf\ If the right-hand side holds then $f = f \circ \id{A} = g \circ \id{A} = g$. \qed
\end{proof}

\begin{definition}[Monic]
    An arrow $f : A \rightarrow B$ is \emph{monic}, $f : A \rightarrowtail B$, iff,
    for every object $X$ and morphisms $x, y : X \rightarrow A$, if $f \circ x =
    f \circ y$ then $x = y$.
\end{definition}

\begin{definition}[Epic]
    An arrow $f : A \rightarrow B$ is \emph{epic}, $f : A \twoheadrightarrow B$, iff,
    for every object $X$ and morphisms $x, y : B \rightarrow X$, if $x \circ f =
    y \circ f$ then $x = y$.
\end{definition}

\begin{definition}[Section, Retraction]
    Let $r : A \rightarrow B$ and $s : B \rightarrow A$. Then $r$ is a \emph{retraction}
    of $s$, and $s$ is a \emph{section} of $r$, iff $r \circ s = \id{B}$.

    We also call a retraction a \emph{split epi}, and a section a \emph{split monic}.
\end{definition}

\begin{proposition}
    \label{proposition:section_monic}
    Every section is monic.
\end{proposition}

\begin{proof}
    \pf
    \step{1}{\pflet{$s : A \rightarrow B$ be a section of $r : B \rightarrow A$}}
    \step{2}{\pflet{$X$ be an object and $x, y : X \rightarrow A$}}
    \step{3}{\assume{$s \circ x = s \circ y$}}
    \step{4}{$x = y$}
    \begin{proof}
        \pf
        \begin{align*}
            x & = r \circ s \circ x \\
            & = r \circ s \circ y \\
            & = y
        \end{align*}
    \end{proof}
    \qed
\end{proof}

\begin{proposition}
    \label{proposition:retraction_epic}
    Every retraction is epic.
\end{proposition}

\begin{proof}
    \pf\ Dual.
\end{proof}

\begin{lemma}
    \label{lemma:left_inverse_right_inverse}
    Let $f : A \rightarrow B$, $g : B \rightarrow A$ and $h : B \rightarrow A$
    be arrows. If $g$ is a retraction of $f$ and $h$ is a section of $f$ then
    $g = h$.
\end{lemma}

\begin{proof}
    \pf
    \begin{align*}
        g & = g \circ \id{B} \\
        & = g \circ f \circ h \\
        & = \id{A} \circ h \\
        & = h & \qed
    \end{align*}
\end{proof}

\begin{proposition}
    Let $r : A \rightarrow B$. Then $r$ is a retraction if and only if, for every
    object $X$ and morphism $y : X \rightarrow B$, there exists $x : X \rightarrow A$
    such that $r \circ x = y$.
\end{proposition}

\begin{proof}
    \pf
    \step{1}{If $r$ is a retraction then, for every
    object $X$ and morphism $y : X \rightarrow B$, there exists $x : X \rightarrow A$
    such that $r \circ x = y$.}
    \begin{proof}
        \step{a}{\pflet{$s : B \rightarrow A$ be a section of $r$.}}
        \step{b}{\pflet{$X$ be an object and $y : X \rightarrow B$}}
        \step{c}{\pflet{$x = s \circ y$}}
        \step{d}{$r \circ x = y$}
    \end{proof}
    \step{2}{If, for every
    object $X$ and morphism $y : X \rightarrow B$, there exists $x : X \rightarrow A$
    such that $r \circ x = y$, then $r$ is a retraction.}
    \begin{proof}
        \pf\ Simply take $x = \id{A}$.
    \end{proof}
    \qed
\end{proof}

\begin{definition}[Isomorphism]
    An arrow $f : A \rightarrow B$ is an \emph{isomorphism}, $f : A \cong B$, iff there exists an arrow
    $\inv{f} : B \rightarrow A$, its \emph{inverse}, such that $\inv{f}$ is both a
    section and a retraction of $f$.
\end{definition}

\begin{proposition}
    The inverse of an isomorphism is unique.
\end{proposition}

\begin{proof}
    \pf\ Lemma \ref{lemma:left_inverse_right_inverse}. \qed
\end{proof}

\begin{proposition}
    Every isomorphism is monic.
\end{proposition}

\begin{proof}
    \pf\ Proposition \ref{proposition:section_monic}. \qed
\end{proof}

\begin{proposition}
    Every isomorphism is epic.
\end{proposition}

\begin{proof}
    \pf\ Proposition \ref{proposition:retraction_epic}. \qed
\end{proof}

\begin{proposition}
    If a morphism is monic and split epi then it is an isomorphism.
\end{proposition}

\begin{proof}
    \pf
    \step{1}{\pflet{$f : A \rightarrow B$ be monic and have section $s : B \rightarrow A$}}
    \step{2}{$f \circ s \circ f = f$}
    \step{3}{$s \circ f = \id{A}$}
    \step{4}{$f$ is iso with inverse $s$.}
    \qed
\end{proof}

\begin{proposition}
    If a morphism is epic and split monic then it is an isomorphism.
\end{proposition}

\begin{proof}
    \pf\ Dual. \qed
\end{proof}

\begin{proposition}
    For any object $A$, we have $\id{A} : A \cong A$ with $\inv{\id{A}} = \id{A}$.
\end{proposition}

\begin{proof}
    \pf\ Since $\id{A} \circ \id{A} = \id{A}$. \qed
\end{proof}

\begin{proposition}
    If $f : A \cong B$ then $\inv{f} : B \cong A$ and $\inv{(\inv{f})} = f$.
\end{proposition}

\begin{proposition}
    If $f : A \cong B$ and $g : B \cong C$ then $g \circ f : A \cong C$ and
    $\inv{(g \circ f)} = \inv{f} \circ \inv{g}$.
\end{proposition}

\begin{proof}
    \pf
    \begin{align*}
        \inv{f} \circ \inv{g} \circ g \circ f & = \inv{f} \circ f \\
        & = \id{A} \\
        g \circ f \circ \inv{f} \circ \inv{g} & = g \circ \inv{g} \\
        & = \id{C} & \qed
    \end{align*}
\end{proof}

\begin{definition}[Balanced]
    A category is \emph{balanced} iff every arrow that is both monic and epic is an
    isomorphism.
\end{definition}

%TODO Every topos is balanced.
%TODO The category of categories is not balanced.

\section{Terminal Objects}

\begin{definition}[Terminal]
    An object $T$ is \emph{terminal} iff, for every object $X$, there exists a
    unique morphism $X \rightarrow T$.
\end{definition}

\begin{example}
    In $\mathbf{Set}$, $1$ is terminal.
\end{example}

\begin{example}
    $* \in 1$ is terminal in $\mathbf{Set}_*$.
\end{example}

\begin{proposition}
    \label{proposition:terminal_unique}
    Any two terminal objects are isomorphic, and the isomorphism between them is unique.
\end{proposition}

\begin{proof}
    \pf
    \step{1}{\pflet{$S$ and $T$ be terminal objects.}}
    \step{2}{\pflet{$\phi$ be the unique arrow $S \rightarrow T$}}
    \step{3}{\pflet{$\inv{\phi}$ be the unique arrow $T \rightarrow S$}}
    \step{4}{$\phi \circ \inv{\phi} = \id{T}$}
    \begin{proof}
        \pf\ Each is the unique arrow $T \rightarrow T$.
    \end{proof}
    \step{5}{$\inv{\phi} \circ \phi = \id{S}$}
    \begin{proof}
        \pf\ Each is the unique arrow $S \rightarrow S$.
    \end{proof}
    \qed
\end{proof}

\section{Initial Objects}

\begin{definition}[Initial]
    An object $I$ is \emph{initial} iff, for every object $X$, there exists a
    unique morphism $I \rightarrow X$.
\end{definition}

\begin{example}
    In a preorder, an initial object is the same as a least element.
\end{example}

\begin{example}
    The only initial object in $\mathbf{Set}$ is $\emptyset$.
\end{example}

\begin{example}
    $* \in 1$ is initial in $\mathbf{Set}_*$.
\end{example}

\begin{proposition}
    Any two initial objects are isomorphic, and the isomorphism between them is unique.
\end{proposition}

\begin{proof}
    \pf\ Dual to Proposition \ref{proposition:terminal_unique}. \qed
\end{proof}

\section{Zero Objects}

\begin{definition}[Zero Object]
    A \emph{zero object} is an object that is both initial and terminal.
\end{definition}

\begin{example}
    $* \in 1$ is the zero object in $\mathbf{Set}_*$.
\end{example}
%TODO Isomorphism of categories

\begin{proposition}
    Let $(P, \leq)$ be a preorder and $a \in P$. The slice category $P/a$ is isomorphic
    to the preorder $(\{ x \in P : x \leq a \}, \leq)$ considered as a category.
\end{proposition}

\begin{proposition}
    Let $(P, \leq)$ be a preorder and $a \in P$. The coslice category $P \backslash a$ is isomorphic
    to the preorder $(\{ x \in P : a \leq x \}, \leq)$ considered as a category.
\end{proposition}

\section{Opposite Category}

\begin{definition}[Opposite Category]
    Let $\mathbb{C}$ be any category. The \emph{opposite category} $\mathbb{C}^{\mathrm{op}}$
    has objects the objects of $\mathbb{C}$ and morphisms $A \rightarrow B$ the morphisms
    $B \rightarrow A$ in $\mathbb{C}$.
\end{definition}

\begin{proposition}
    An initial object in $\mathbb{C}$ 
\end{proposition}
\section{Groupoids}

\begin{definition}[Groupoid]
    A \emph{groupoid} is a category in which every morphism is an isomorphism.
\end{definition}

\section{Automorphisms}

\begin{definition}[Automorphism]
    Let $A$ be an object. An \emph{automorphism} on $A$ is an isomorphism $A \cong A$.
\end{definition}

\section{Quotient Sets}

\begin{proposition}
    Let $A$ be a set and let $\sim$ be an equivalence relation on $A$. Let $\mathbb{C}$
    be the subcategory of $\mathbf{Set} \backslash A$ consisting of pairs $(Z, \phi)$
    such that, for all $x,y \in A$, if $x \sim y$ then $\phi(x) = \phi(y)$.

    Then $(A/\sim, \pi)$ is the initial object in $\mathbb{C}$, where $\pi : A
    \rightarrow A / \sim$ is the canonical projection.
\end{proposition}

\begin{proof}
    \pf\ For any object $\phi : A \rightarrow Z$ in the category, the only morphism
    $(A/\sim, \pi) \rightarrow (Z, \phi)$ is the function $f : A/\sim \rightarrow Z$
    such that $f([a]) = \phi(a)$ for all $a \in A$. \qed
\end{proof}

\section{Products}

\begin{definition}[Product]
    Let $\mathbb{C}$ be a category and $\{ A_i \}_{i \in I}$ a family of objects in $\mathbb{C}$.
    A \emph{product} of $\{ A_i \}_{i \in I}$ is a terminal object in the category with:
    \begin{itemize}
        \item objects all pairs $(C,\{ f_i \}_{i \in I})$ where $C \in \mathbb{C}$
        and $f_i : C \rightarrow A_i$ for all $i \in I$;
        \item morphisms $(C,\{ f_i \}_{i \in I}) \rightarrow (D,\{ g_i \}_{i \in I})$ 
        all morphisms $x : C \rightarrow D$ such that, for all $i \in I$, we have $g_i \circ x = f_i$.
    \end{itemize}
\end{definition}

\begin{example}
    The Cartesian product $\prod_{i \in I} A_i$ with projections is the product of $\{ A_i \}_{i \in I}$ in $\mathbf{Set}$.
\end{example}

\begin{example}
    Products in a preorder are meets.
\end{example}

\begin{proposition}
    If $A \times B$ and $B \times A$ exist then they are isomorphic.
\end{proposition}

\begin{proposition}
    If $A \times (B \times C)$ and $(A \times B) \times C$ exist then they are isomorphic.
\end{proposition}
\section{Coproducts}

\begin{definition}[Coproduct]
    Let $\mathbb{C}$ be a category and $A, B \in \mathbb{C}$. A \emph{coproduct} of $A$
    and $B$ is an initial object in the category with:
    \begin{itemize}
        \item objects all triples $(C,f,g)$ where $C \in \mathbb{C}$, $f : A \rightarrow
        C$ and $g : B \rightarrow C$
        \item morphisms $(C,f,g) \rightarrow (D,h,k)$ all morphisms $x : C \rightarrow D$
        such that $h = x \circ f$ and $k = x \circ g$
    \end{itemize}
\end{definition}

\begin{example}
    Disjoint unions are coproducts in $\mathbf{Set}$.
\end{example}

\begin{example}
    Products in a preorder are joins.
\end{example}

\end{document}